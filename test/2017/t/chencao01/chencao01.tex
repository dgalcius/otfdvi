%BeginFileInfo
%%Publisher=DGBOOK
%%Project=CHENCAO
%%Manuscript=CHENCAO01
%%MS position=
%%Stage=308
%%TID=ispudulyte
%%Pages=
%%Format=2016
%%Distribution=vtex
%%Destination=PDF
%%PDF type=print
%%PDF.Maker=pdfwf_luatex
%%Spelled=Dictionary: American, Computer: 1GSRED478, 2018.02.22 08:31
%%History1=2018.02.21 12:37
%EndFileInfo
\documentclass[]{chencao}
\usepackage{book-aux-toc}
\makeindex

\startlocaldefs
%author defs here:
\allowdisplaybreaks
\endlocaldefs

%\HPROOF\SETGRID
%\PROOF
\CRC

\pubyear{2018}
\firstpage{1}
\lastpage{5}
%\openaccess

\begin{document}

\thisischapter{1}
\chapter{Introduction}


\begin{frontmatter}
\abstract{Problems in modeling and control of crowds of pedestrians are
firstly presented. Short comments on previous research and the
organization of this book are also presented in this chapter.}
\end{frontmatter}
%\setcounter{section}{1}

%s1 #&#
\section{Motivation}

Humans are the most socially complex animals on this planet. It is not
surprising that research related to crowds of pedestrian has received
a lot of attention in recent years. A lot of work has been
conducted for particles, vehicles, robots, animals, and
even human beings from the perspectives of behavior, psychology,
cognitive, and network theory (see
\cite{1.Couzin2008,1.Couzin2005,1.Czirok1999,1.Spieser2009,1.Moussaid2011}).
{Among the previous work, problems related to crowds of pedestrians are
the most challenging due to difficulties in modeling of human beings,
as there are no universal tools to characterize the complex temporal and
spatial features of crowds.}

On the other side, more and more tragedies due to people's stampede have
been reported in recent years. The most tragic stampede occurred in
Mecca in 1990, where 1426 pilgrims were trampled to death or suffocated.
In evacuation, people got injured or lost their lives due to panic motion or running in every direction without aim. The catastrophic events
have demonstrated the need to reanalyze and
reexamine current evacuation policies and procedures for crowds of
pedestrians. {Thus, policy makers urgently need better crowd
management or evacuation strategies.}

The problems confronted in research of crowds of pedestrians can be
listed as follows:
%
\begin{enumerate}
\item[(1)] How to characterize or obtain a satisfactory social-dynamic model for
crowds of pedestrians that is much closer to reality compared to the
previous model.
\item[(2)] How to enforce and stabilize the desired pattern formation of crowds of
pedestrians and how to avoid some rare or dangerous formation pattern
in evacuation of crowds.
\end{enumerate}

%s2 #&#
\section{Current status of research}

%s2.1 #&#
\subsection{Modeling of crowds}

According to the differences in scales, the models for crowds of
pedestrians can be categorized by microscopic model,macroscopic
model and mesoscopic model (see
\cite{1.Cristiani2011,1.Bellomo2012,1.DirkHelbing2005,1.Bellomo2011b,1.Dogbe2012a}).
%
\begin{enumerate}
\item[(1)] The Newton principle is a powerful tool to describe the motion of
particles at microscopic scale but the heavy burden of computation will
not make it a better choice with the increase of the number of
particles.
\item[(2)] Conservation of mass and momentum is the basic principle employed in
obtaining the macroscopic model. Although the computation burden has
been reduced greatly, the individual character of each pedestrian has
been ignored when using this kind of method and the heterogeneity of
different pedestrians can't be easily characterized at the macro-scale.
\item[(3)] For a mesoscopic model, not only the computation burden has been
reduced, but also the heterogeneity of different pedestrians can be
guaranteed. However, {qualitative and quantitative results are
not easy to obtain for integral-differential equations obtained at this
scale.}
\end{enumerate}
Besides the problem of multiple scales in modeling crowds of
pedestrians, {there are a lot of other effects that influence the
pattern of motion of crowds, such as imitating behavior of neighbors,
following external signals, psychological unity, emotional intensity,
and level of violence, as shown in \cite{1.Berlonghi1995}.} The present
status is that there is no common agreement on which model is the best
one in describing this kind of complex social dynamics. A new
methodology and theory are required for better approximating and
characterizing this complex social-dynamic system.

%s2.2 #&#
\subsection{Control of crowds}

Compared to the modeling of crowds mentioned above, control of crowds
is much more challenging as shown in recent work; see
\cite{1.DirkHelbing2005,1.Bellomo2011a,1.Cristiani2011,
1.ChristianDogbe2012,1.Bogdan2011a,1.DanielStuart2013}.
%
\begin{itemize}
\item
A lot of evacuation procedures or policies have been designed using
computer simulations. Considering the adaptability and robustness to the
environment, the method of simulation is not a good choice as the
obtained modeling results or evacuation policies are not effective
anymore in different buildings or different scenarios.
\item
In some of the previous research based on the mathematical model,
pedestrians have been treated as particles and some of the
characteristics of human beings have been neglected in control of the
crowds, such as short-range and long-range interactions, effects of
memory, and statistical characters at the temporal or spatial scale.
\item
In large crowds of pedestrians, self-organization or cooperative
movement have been observed a long time ago. But there is little
research on how to realize the desired formation patterns and prevent
dangerous patterns so that a stampeding tragedy can be avoided.
\end{itemize}
As there is no perfect model for all kinds of scenarios, there are no
universal controllers that can solve all evacuation problems. Based on
models that are much closer to reality, many efforts have been made as
regards control of crowds for the purpose of better management and
efficient evacuation of crowds.

%s2.3 #&#
\subsection{Comments}

Some manuscripts have been published in recent years concerning the
modeling and control problems of crowds of pedestrians, such as
\cite{1.AlfioQuarteroni2003,1.Haken2006,1.Thalmann2007,1.WolframW.F2008,1.Bellomo2008a,1.Kachroo2008b,1.Pelechano2008,1.Timmermans2009,1.Kachroo2009,1.Barcelo2010}.
But the authors found that some important characteristics of the crowds
have been neglected in previous research and their effects should be
reconsidered and reexamined in both the modeling and the control stages
for crowds of pedestrians.
%
\begin{enumerate}
\item[(1)] \textit{Integer order versus fractional order at temporal scale.}

The movements of each pedestrian are the results of a complex
interaction between physical and psychological issues. Inter-event time
has been proved to play an important role in analyzing the movement of
crowds, as shown in \cite{1.BruceJ.West2014}. Contrary to the fact that
the distribution of inter-event time satisfies a power law distribution
in most cases, an exponential law distribution has been assumed in most
of the previous research within the framework of calculus of integer
order. The \index{Calculus of fractional order}calculus of fractional
order has been introduced at the temporal scale as a remedy for this gap
in this book.
\item[(2)] \textit{Integer order versus fractional order in spatial scale.}

Another important thing should be pointed out: the spatial scale is
assumed to be uniform and the dimensions of space are restricted to one
dimension, two dimensions, and three dimensions in the previous
research. But these assumptions are only reasonable if the crowds of
pedestrians can fill space like particles of gases or fluids, while this
is not the case as is clear from observations. Theoretically, only a
normal diffusive process has been considered in the previous research
and few results have been reported for sub-diffusive or super-diffusive
processes for modeling of crowds.
\item[(3)] \textit{Short-range interactions versus long-range interactions}

Short-range interactions have been extensively considered in the
schooling of fish and flocking of birds and in the control of
multi-agent systems, while long-range interactions dominating a system's
phase transition only has received attention recently. Based on the
results obtained in \cite{1.Ishiwata2012}, long-range interactions at
the micro-scale have been proved to be closely connected to the dynamic
model of fractional order at the macro-scale. Not only short-range
interactions but also long-range interactions can easily be manipulated
using the framework of the \index{Calculus of fractional order} calculus
of fractional order.
\end{enumerate}

%s3 #&#
\section{Organization of this book}

In the first part of this book, a dynamic model of fractional order for
crowds of pedestrians is studied at the micro-scale, macro-scale, and
meso-scale. Ordinary differential equations (ODEs) of fractional order,
partial differential equations (PDEs) of fractional order and coupled
ODE-PDEs of fractional order have been obtained for modeling of crowds
where the characteristics of temporal, spatial, and long-range
interactions mentioned above have been embedded. Based on the obtained
models, control or evacuation of crowds is considered in the second part
of this book. An intelligent evacuation system based on FO-Diff-MAS2D
is also introduced to illustrate or show the effectiveness of the
theoretical results. The organization of this book is shown in Figure~\ref{1.Fig:organization}.

%f1 #&#
\begin{figure}
\begin{centering}
\includegraphics{01f01}
\end{centering}
\caption{Organization of this book.}
\label{1.Fig:organization}
\end{figure}

\def\refname{References}
\begin{thebibliography}{00}

%%% bbsrt2.pl, ver. 2.5.8, 2017.08.09
%b1 ###bbsrt2
%b1 #&#
\bibitem{1.Bellomo2008a}
\begin{bbook}
\bauthor{\binits{N.} \bsnm{Bellomo}}.
\emph{\bbtitle{Modeling Complex Living Systems:
A Kinetic Theory and Stochastic Game Approach}}.
\bpublisher{Birkh\"{a}user}, \blocation{Boston},
\byear{2008}.
\end{bbook}
\endbibitem

%b2 ###bbsrt2
%b3 #&#
\bibitem{1.Bellomo2011b}
\begin{barticle}
\bauthor{\binits{N.} \bsnm{Bellomo}} and
\bauthor{\binits{C.} \bsnm{Dogbe}}.
\batitle{On the modeling of traffic and crowds a survey of models, speculations, and perspectives}.
\emph{\bjtitle{SIAM Review}},
\bvolume{53}(\bissue{3}):\bfpage{409}--\blpage{463},
\byear{2011}.
\end{barticle}
\endbibitem

%b3 ###bbsrt2
%b4 #&#
\bibitem{1.Bellomo2012}
\begin{barticle}
\bauthor{\binits{N.} \bsnm{Bellomo}},
\bauthor{\binits{B.} \bsnm{Piccoli}}, and
\bauthor{\binits{A.} \bsnm{Tosin}}.
\batitle{Modeling crowd dynamics from a complex system viewpoint}.
\emph{\bjtitle{Mathematical Models and Methods in Applied Sciences}},
\bvolume{22}:\bfpage{1}--\blpage{29},
\byear{2012}.
\end{barticle}
\endbibitem

%b4 ###bbsrt2
%b2 #&#
\bibitem{1.Bellomo2011a}
\begin{barticle}
\bauthor{\binits{N.} \bsnm{Bellomo}},
\bauthor{\binits{C.} \bsnm{Bianca}}, and
\bauthor{\binits{V.} \bsnm{Coscia}}.
\batitle{On the modeling of crowd dynamics:
an overview and research perspectives}.
\emph{\bjtitle{S$\vec{e}$MA Journal}},
\bvolume{54}(\bissue{1}):\bfpage{25}--\blpage{46},
\byear{2013}.
\end{barticle}
\endbibitem

%b5 ###bbsrt2
%b11 #&#
\bibitem{1.Berlonghi1995}
\begin{barticle}
\bauthor{\binits{A.\,E.} \bsnm{Berlonghi}}.
\batitle{Understanding and planning for different spectator crowds}.
\emph{\bjtitle{Safety Science}},
\bvolume{18}(\bissue{4}):\bfpage{239}--\blpage{247},
\byear{1995}.
\end{barticle}
\endbibitem

%b6 ###bbsrt2
%b5 #&#
\bibitem{1.Bogdan2011a}
\begin{bchapter}
\bauthor{\binits{P.} \bsnm{Bogdan}} and
\bauthor{\binits{R.} \bsnm{Marculescu}}.
\bctitle{A fractional calculus approach to modeling fractal dynamic games}.
In \emph{\bbtitle{Proceedings of the IEEE Conference on Decision and Control and
European Control Conference}}, pages \bfpage{255}--\blpage{260},
\byear{2011}.
\end{bchapter}
\endbibitem

%b7 ###bbsrt2
%b6 #&#
\bibitem{1.Dogbe2012a}
\begin{barticle}
\bauthor{\binits{D.} \bsnm{Christian}}.
\batitle{On the modelling of crowd dynamics by generalized kinetic models}.
\emph{\bjtitle{Journal of Mathematical Analysis and Applications}},
\bvolume{387}(\bissue{2}):\bfpage{512}--\blpage{532},
\byear{2012}.
\end{barticle}
\endbibitem

%b8 ###bbsrt2
%b7 #&#
\bibitem{1.Couzin2008}
\begin{barticle}
\bauthor{\binits{I.\,D.} \bsnm{Couzin}}.
\batitle{Collective cognition in animal groups}.
\emph{\bjtitle{Trends in Cognitive Sciences}},
\bvolume{13}(\bissue{1}):\bfpage{36}--\blpage{43},
\byear{2008}.
\end{barticle}
\endbibitem

%b9 ###bbsrt2
%b8 #&#
\bibitem{1.Couzin2005}
\begin{barticle}
\bauthor{\binits{I.\,D.} \bsnm{Couzin}},
\bauthor{\binits{J.} \bsnm{Krause}},
\bauthor{\binits{N.\,R.} \bsnm{Franks}}, and
\bauthor{\binits{S.\,A.} \bsnm{Levin}}.
\batitle{Effective leadership and decision-making in animal groups on the move}.
\emph{\bjtitle{Nature}},
\bvolume{433}:\bfpage{513}--\blpage{516},
\byear{2005}.
\end{barticle}
\endbibitem

%b10 ###bbsrt2
%b12 #&#
\bibitem{1.Cristiani2011}
\begin{barticle}
\bauthor{\binits{E.} \bsnm{Cristiani}},
\bauthor{\binits{B.} \bsnm{Piccoli}}, and
\bauthor{\binits{A.} \bsnm{Tosin}}.
\batitle{Multiscale modeling of granular flows with application to crowd dynamics}.
\emph{\bjtitle{Multiscale Modelling and Simulation}},
\bvolume{9}(\bissue{1}):\bfpage{155}--\blpage{182},
\byear{2011}.
\end{barticle}
\endbibitem


%vBOOK0111G21

%b11 ###bbsrt2
%b9 #&#
\bibitem{1.Czirok1999}
\begin{barticle}
\bauthor{\binits{A.} \bsnm{Czirok}}.
\batitle{Collective motion of self-propelled particles kinetic phase transition in one dimension}.
\emph{\bjtitle{Physical Review Letters}},
\bvolume{82}(\bissue{1}):\bfpage{209}--\blpage{212},
\byear{1999}.
\end{barticle}
\endbibitem

%b12 ###bbsrt2
%b10 #&#
\bibitem{1.ChristianDogbe2012}
\begin{barticle}
\bauthor{\binits{C.} \bsnm{Dogbe}}.
\batitle{Applicable thermostatted models to crowd dynamics: Comment on ``thermostatted kinetic equations as models for complex systems in physics and life sciences'' by Carlo Bianca}.
\emph{\bjtitle{Physics of Life Reviews}},
\bvolume{9}(\bissue{4}):\bfpage{410}--\blpage{412},
\byear{2012}.
\end{barticle}
\endbibitem

%b13 ###bbsrt2
%b13 #&#
\bibitem{1.Haken2006}
\begin{bbook}
\bauthor{\binits{H.} \bsnm{Haken}}.
\emph{\bbtitle{Information and Self-Organization A Macroscopic Approach to Complex Systems}}.
\bpublisher{Springer}, \blocation{Berlin, Heidelberg},
\byear{2006}.
\end{bbook}
\endbibitem

%b14 ###bbsrt2
%b14 #&#
\bibitem{1.DirkHelbing2005}
\begin{barticle}
\bauthor{\binits{D.} \bsnm{Helbing}},
\bauthor{\binits{L.} \bsnm{Buzna}},
\bauthor{\binits{A.} \bsnm{Johansson}}, and
\bauthor{\binits{T.} \bsnm{Werner}}.
\batitle{Self-organized pedestrian crowd dynamics: Experiments, simulations, and design solutions}.
\emph{\bjtitle{Transportation Science}},
\bvolume{39}(\bissue{1}):\bfpage{1}--\blpage{24},
\byear{2005}.
\end{barticle}
\endbibitem

%b15 ###bbsrt2
%b15 #&#
\bibitem{1.Ishiwata2012}
\begin{barticle}
\bauthor{\binits{R.} \bsnm{Ishiwata}} and
\bauthor{\binits{Y.} \bsnm{Sugiyama}}.
\batitle{Relationships between power-law long-range interactions and fractional mechanics}.
\emph{\bjtitle{Physica A}},
\bvolume{391}(\bissue{23}):\bfpage{5827}--\blpage{5838},
\byear{2012}.
\end{barticle}
\endbibitem


%vBOOK0111G21

%b16 ###bbsrt2
%b16 #&#
\bibitem{1.Barcelo2010}
\begin{bbook}
\bauthor{\binits{B.} \bsnm{Jaume}}.
\emph{\bbtitle{Fundamentals of Traffic Simulation}}.
\bpublisher{Springer Science and Business Media}, \blocation{Berlin},
\byear{2010}.
\end{bbook}
\endbibitem

%b17 ###bbsrt2
%b18 #&#
\bibitem{1.Kachroo2009}
\begin{bbook}
\bauthor{\binits{P.} \bsnm{Kachroo}}.
\emph{\bbtitle{Pedestrian Dynamics: Mathematical Theory and Evacuation Control}}.
\bpublisher{CRC Press, Taylor \& Francis Group}, \blocation{London},
\byear{2009}.
\end{bbook}
\endbibitem


%vBOOK0127G24

%b18 ###bbsrt2
%b19 #&#
\bibitem{1.Kachroo2008b}
\begin{bbook}
\bauthor{\binits{P.} \bsnm{Kachroo}},
\bauthor{\binits{S.\,J.} \bsnm{Al-nasur}},
\bauthor{\binits{S.\,A.} \bsnm{Wadoo}}, and
\bauthor{\binits{A.} \bsnm{Shende}}.
\emph{\bbtitle{Pedestrian Dynamics Feedback Control of Crowd Evacuation}}.
\bpublisher{Springer-Verlag}, \blocation{Berlin, Heidelberg},
\byear{2008}.
\end{bbook}
\endbibitem


%vBOOK0111G21

%b19 ###bbsrt2
%b20 #&#
\bibitem{1.WolframW.F2008}
\begin{bbook}
\bauthor{\binits{W.\,W.\,F.} \bsnm{Klingsch}},
\bauthor{\binits{C.} \bsnm{Rogsch}},
\bauthor{\binits{A.} \bsnm{Schadschneider}}, and
\bauthor{\binits{M.} \bsnm{Schreckenberg}}.
\emph{\bbtitle{Pedestrian and Evacuation Dynamics}}.
\bpublisher{Springer-Verlag}, \blocation{Berlin, Heidelberg},
\byear{2010}.
\end{bbook}
\endbibitem

%b20 ###bbsrt2
%b21 #&#
\bibitem{1.Moussaid2011}
\begin{barticle}
\bauthor{\binits{M.} \bsnm{Moussaid}},
\bauthor{\binits{D.} \bsnm{Helbing}}, and
\bauthor{\binits{G.} \bsnm{Theraulaz}}.
\batitle{How simple rules determine pedestrian behavior and crowd disasters}.
\emph{\bjtitle{Proceedings of the National Academy of Sciences of the United States of America}},
\bvolume{108}(\bissue{17}):\bfpage{6884}--\blpage{6888},
\byear{2011}.
\end{barticle}
\endbibitem


%vBOOK0111G21

%b21 ###bbsrt2
%b22 #&#
\bibitem{1.Pelechano2008}
\begin{bbook}
\bauthor{\binits{N.} \bsnm{Pelechano}},
\bauthor{\binits{J.\,M} \bsnm{Allbeck}}, and
\bauthor{\binits{N.\,I.} \bsnm{Badler}}.
\emph{\bbtitle{Virtual Crowds Methods, Simulation, and Control}}.
\bpublisher{Morgan \& Claypool}, \blocation{Williston},
\byear{2008}.
\end{bbook}
\endbibitem

%b22 ###bbsrt2
%b23 #&#
\bibitem{1.AlfioQuarteroni2003}
\begin{barticle}
\bauthor{\binits{A.} \bsnm{Quarteroni}} and
\bauthor{\binits{A.} \bsnm{Veneziani}}.
\batitle{Analysis of a geometrical multiscale model based on the coupling
of ODE and PDE for blood flow simulations}.
\emph{\bjtitle{Multiscale Modeling and Simulation}},
\bvolume{1}:\bfpage{173}--\blpage{195},
\byear{2003}.
\end{barticle}
\endbibitem

%b23 ###bbsrt2
%b17 #&#
\bibitem{1.Spieser2009}
\begin{bchapter}
\bauthor{\binits{K.} \bsnm{Spieser}} and
\bauthor{\binits{D.\,E.} \bsnm{Davison}}.
\bctitle{A cooperative multi-agent approach for stabilizing the
psychological dynamics of a two-dimensional crowd}.
In \emph{\bbtitle{Proceedings of the American Control Conference}},
pages \bfpage{5737}--\blpage{5742},
\byear{2009}.
\end{bchapter}
\endbibitem


%vBOOK0111G21

%b24 ###bbsrt2
%b24 #&#
\bibitem{1.DanielStuart2013}
\begin{bchapter}
\bauthor{\binits{D.} \bsnm{Stuart}},
\bauthor{\binits{K.} \bsnm{Christensen}},
\bauthor{\binits{A.} \bsnm{Chen}},
\bauthor{\binits{K.-C.} \bsnm{Cao}},
\bauthor{\binits{C.} \bsnm{Zeng}}, and
\bauthor{\binits{Y.\,Q.} \bsnm{Chen}}.
\bctitle{A framework for modeling and managing mass pedestrian evacuations involving
individuals with disabilities: Networked segways as mobile sensors \& actuators}.
In \emph{\bbtitle{Proceedings of the ASME
2013 International Design Engineering Technical Conferences and Computers and
Information in Engineering Conference}},
\bcomment{DETC2013-12652},
\byear{2013}.
\end{bchapter}
\endbibitem


%vBOOK0111G21

%b25 ###bbsrt2
%b25 #&#
\bibitem{1.Thalmann2007}
\begin{bbook}
\bauthor{\binits{D.} \bsnm{Thalmann}} and
\bauthor{\binits{S.} \bsnm{Raupp Musse}}.
\emph{\bbtitle{Crowd Simulation}}.
\bpublisher{Springer}, \blocation{Berlin},
\byear{2007}.
\end{bbook}
\endbibitem


%vBOOK0111G21

%b26 ###bbsrt2
%b26 #&#
\bibitem{1.Timmermans2009}
\begin{bbook}
\bauthor{\binits{H.} \bsnm{Timmermans}}.
\emph{\bbtitle{Pedestrian Behavior: Models, Data Collection and Applications}}.
\bpublisher{Emerald Group Publishing Limited}, \blocation{Bingley},
\byear{2009}.
\end{bbook}
\endbibitem


%vBOOK0111G21

%b27 ###bbsrt2
%b27 #&#
\bibitem{1.BruceJ.West2014}
\begin{bbook}
\bauthor{\binits{B.\,J.} \bsnm{West}},
\bauthor{\binits{M.} \bsnm{Turalska}}, and
\bauthor{\binits{P.} \bsnm{Grigolini}}.
\emph{\bbtitle{Networks of Echoes Imitation, Innovation and Invisible Leaders, volume Computatio}}.
\bpublisher{Springer International Publishing}, \blocation{Switzerland},
\byear{2014}.
\end{bbook}
\endbibitem



\end{thebibliography}
%% BODY
\end{document}
