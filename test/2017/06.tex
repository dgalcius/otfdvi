%BeginFileInfo
%%Publisher=DGBOOK
%%Project=STEFAN1
%%Manuscript=STEFAN101
%%MS position=
%%Stage=307
%%TID=juraten
%%Pages=0
%%Format=2016
%%Distribution=vtex
%%Destination=PDF
%%PDF type=print
%%PDF.Maker=pdfwf_luatex
%%PS.Maker=xdvipsk
%%Spelled=Dictionary: American (with accents), Computer: 1GSMPR490, 2018.03.27 10:23
%%History1=2018.03.27 09:19
%EndFileInfo
\documentclass[nochecklpage]{stefan1}
%\usepackage{book-aux-toc}
\usepackage{mathrsfs}
\usepackage[left]{vmkcol}
\makeindex

\startlocaldefs
\newcommand{\rrvert}{\vert}
\newcommand{\rrVert}{\Vert}
\newcommand{\llvert}{\vert}
\newcommand{\llVert}{\Vert}
%author defs here:
\allowdisplaybreaks

\newtheorem{theorem}{Theorem}[chapter]
\newtheorem{assertion}[theorem]{Assertion}
\newtheorem{statement}[theorem]{Statement}
\theoremstyle{definition}
\newtheorem{postulate}[theorem]{Postulate}
\endlocaldefs

%\HPROOF\SETGRID
%\PROOF
\CRC

\pubyear{2019}
\firstpage{1}
\lastpage{36}
%\openaccess

\begin{document}

\thisischapter{1}
\chapter{Quantum logic}\label{ch:QM}

\begin{epigraph}
The nature of light is a subject of no material importance to
the concerns of life or to the practice of the arts, but it is in many
other respects extremely interesting.
%
\author{Thomas Young}
\end{epigraph}
%
%% BODY
%
In this chapter, we will continue our discussion of the connections
between preparation devices, physical systems and measuring instruments,
started in the Introduction. In particular, we will try to understand
what is actually measured by the instruments, and how these results
depend on the state of the observed system.

Until the end of the 19th century such questions could only raise
eyebrows. In classical mechanics and in all pre-quantum physics, it was
tacitly assumed that in each state the physical system possesses a set
of quantities (position, momentum, mass, etc.). These quantities simply
show up in measurements. They can be measured simultaneously, accurately
and reproducibly. Yes, of course, every measurement is limited by a
certain imprecision, but this is only a technical difficulty that can
and should be neglected in a fundamental theory. All this was considered
so obvious that it was not even mentioned in textbooks.

However, since the end of the 19th century, these traditional ideas
began to disintegrate under the onset of new discoveries such as the
radiation spectrum of heated bodies, the discrete spectrum of atoms and
the photoelectric effect. Solutions of all these problems have been
found within the framework of
\emph{quantum mechanics}\index{quantum mechanics}~--~a~completely new approach to physics that
emerged in the first third of the 20th century as a result of joint
efforts and passionate debates of such outstanding scientists as Bohr,
Born, de Broglie, Dirac, Einstein, Fermi, Fock, Heisenberg, Pauli,
Planck, Schr\"{o}dinger, Wigner and many
others.\index{Bohr, N.}\index{Born, M.}\index{de Broglie, L.}\index{Einstein, A.}\index{Fermi, E.}\index{Fock, V.\,A.}\index{Heisenberg, W.}\index{Pauli, W.}\index{Planck, M.}\index{Schr\"{o}dinger, E.}\index{Wigner, E.\,P.}\index{Dirac, P.\,A.\,M.}
Out of all these
studies, a completely unexpected and paradoxical picture of the physical
world has emerged that was completely unlike the orderly and transparent
classical picture. In spite of their counterintuitive strangeness,
predictions of quantum mechanics are extraordinarily accurate: they are
checked daily in countless physical and chemical laboratories around the
world, and have never been refuted. This makes quantum mechanics the
most successful physical theory of all time.

There are dozens of good textbooks explaining the laws of quantum
mechanics and how to use them to analyze systems and predict
observations in each particular case. We assume that the reader is
fairly familiar with these laws. We will be more interested in the
deeper meaning and interpretation of the quantum formalism, which still
generates bitter controversies. Why does nature behave in a random way?
Or is this randomness only apparent, but in fact there is a deeper level
of reality, where quantum uncertainty gives way to some new laws? How
can different states (alive and dead Schr\"{o}dinger's cats) exist in
a superposition? Is it possible to change the rules of quantum mechanics
(for example, by adding some nonlinearity to the Hilbert space)\index{Hilbert, D.} without
being in contradiction with experiments? People are increasingly asking
such questions recently, when the search for the quantum theory of
gravitation has intensified, and one popular trend is to look for
alternative formulations of quantum mechanics in order to ``harmonize''
it with the general theory of relativity \cite{Khrennikov}.

In this chapter we will present a rather old, but not well-known point
of view on the origin of quantum laws. This approach seeks
explanation of the quantum behavior in the fundamental logical structure
of physics. In particular, this approach asserts that the true logical
relationships between results of measurements are different from the
classical laws of Aristotle and Boole. The usual classical logic needs
to be replaced by the so-called \emph{quantum logic}.\index{quantum logic}\index{Aristotle}\index{Boole, G.}

Since ancient Greece, logic has been considered the queen of sciences,
perhaps not even a science as such, but something even more fundamental:
a metascience, a framework for our perception of the world and for the
construction of all other sciences. Therefore, it is difficult to
imagine anything more revolutionary and provocative than an encroachment
on the laws of logic. Nevertheless, there are enough convincing reasons
to make just this step.
%
\begin{quote}
In introductory quantum physics classes (especially in the United
States), students are informed \emph{ex cathedra} that the state of a
physical system is represented by a complex-valued wave function~$\psi $,
that observables correspond to self-adjoint operators, that the
temporal evolution of the system is governed by a Schr\"{o}dinger
equation and so on. Students are expected to accept all this
uncritically, as their professors probably did before them. Any question
of why is dismissed with an appeal to authority and an injunction to
wait and see how well it all works. Those students whose curiosity
precludes blind compliance with the gospel according to Dirac and von
Neumann are told that they have no feeling for physics and that they
would be better off studying mathematics or philosophy. A happy
alternative to teaching by dogma is provided by basic quantum logic,
which furnishes a sound and intellectually satisfying background for the
introduction of the standard notions of elementary quantum mechanics --
D.\,J. Foulis \cite{Foulis}.\index{Foulis, D.\,J.}
\end{quote}
%
The idea that the most fundamental difference between classical and
quantum mechanics lies in their different logical structures belongs to
Birkhoff and von Neumann. In this chapter, we briefly outline their
ideas of \emph{quantum logic} \cite{Birkhoff} as well as later
contributions made especially by Mackey \cite{Mackey} and Piron
\cite{Piron2, Piron}; see also \cite{Curcuraci}.
\index{Birkhoff, G.}\index{von Neumann, J.}\index{Mackey, G.\,W.}\index{Piron, C.}

We will argue that the formalism of quantum mechanics (including the
algebras of state vectors and Hermitian operators in the Hilbert space)
follows almost inevitably from the simplest properties of measurements
and logical relationships between them. These properties and
relationships are so simple and fundamental that it seems impossible to
modify them, and therefore it would be almost impossible to change
quantum laws without violating their internal consistency and agreement
with experiment. The practical conclusion is that the unification of
quantum mechanics and relativity will not be achieved by changing or
modifying quantum laws.\footnote{In Volume~3 we will
explain how one should change the formalism of special relativity to
make it compatible with quantum mechanics.}

In Section \ref{sc:thought} we will examine limitations of classical
approaches by analyzing the two-holes (two-slits) interference
experiment from the points of view of the wave and corpuscular theories
of light.

The logical structure of classical physics will be presented in Sections
\ref{sc:1.1} and \ref{sc:measurements-and-probabilities}. In particular,
we will discuss the close relationship between the classical Boolean
logic and the phase space formalism. In Section \ref{sc:1.3} we will
note the remarkable fact that the only difference between classical and
quantum logics (and, therefore, between classical and quantum physics
in general) lies in two inconspicuous axioms of \emph{distributivity}. This
postulate of classical logic should be replaced by the \emph{orthomodular}
postulate of quantum logic. In Section \ref{ss:quant-obs} this will lead us (via Piron's theorem) to the standard formalism
of quantum mechanics with its Hilbert spaces, Hermitian operators, wave
functions, etc. In Section \ref{sc:complete} we will add some thoughts
to the endless philosophical debate about interpretations of quantum
mechanics.

%s1 #&#
\section{Why do we need quantum mechanics?}\label{sc:thought}
The inadequacy of the classical concepts becomes clear if we analyze the
dispute between corpuscular and wave theories of light. Let us
illustrate the essence of this, without exaggeration, centuries-old
debate by the example of a thought experiment with the \emph{camera}
\emph{obscura}.\index{camera obscura}

%s1.1 #&#
\subsection{Corpuscular theory of light}\label{ss:corpuscular}
You may have seen or heard about a simple optical device called
\emph{camera obscura} or pinhole camera.\index{pinhole camera} It is easy to make this device
yourself. Take a lightproof box, make a small hole in one of its walls
and place a photographic plate at the opposite wall, as shown in
Figure~\ref{fig:3.1}. The light entering the inside of the box through the hole
will create a clear inverted image of the outside world on the
photographic plate.

\begin{SCfigure}[50][b!]
\includegraphics{01f01}
\caption{The image in the pinhole camera is created by rectilinear beams (rays)
of light.}
\label{fig:3.1}
\end{SCfigure}

You can achieve even greater clarity by reducing the size of the hole.
But this, of course, will reduce the brightness of the image. This
behavior of light has been known for centuries. The first scientific
explanation for this and many other properties of light (reflection,
refraction, etc.) was suggested by Newton.\index{Newton, I.} In a slightly modernized
language, his \emph{corpuscular theory}\index{corpuscular theory}
explained the formation of the image as follows:
%
\begin{quote}
\textbf{\textit{Corpuscular theory:}} Light is a stream of tiny particles
(photons) flying along straight classical trajectories (light rays).
[For example, the ray that lands at the point $ A'$ in Figure~\ref{fig:3.1} was emitted from the point $ A $ and passed right through
the hole.] Each such particle carries a certain amount of energy. When
the particle collides with the photographic plate, this energy is
released within one grain of the emulsion and creates a single image
point. Bright light contains so many photons that their individual spots
flood the photographic plate. All these points
merge into one continuous image, and the density of the image is
proportional to the number of particles hitting the plate during the
time of exposure.
\end{quote}
%
Let us continue our experiment with the pinhole camera, making the hole
size smaller and smaller. Corpuscular theory asserts that shrinking
holes will produce a clearer, but dimmer image. However, the experiment
shows something completely different! At some point, as the size of the
hole is reduced, the image will begin to blur; and in the limit of a
very small hole all the details will disappear, and the picture will
turn into one circular diffuse spot, as in Figure~\ref{fig:3.2}\,(a). The
shape and size of this blur are no longer dependent on the light source
outside the camera. It would seem that the light rays, passing through
the small hole, are randomly scattered in all directions. This effect
was discovered by Grimaldi in the middle of the 17th century and was
subsequently dubbed \emph{diffraction}.\index{diffraction}\index{Grimaldi, F.\,M.}

%f1 #&#
\begin{figure}[h]
\vspace*{-6pt}
\includegraphics{01f02}
\caption{(a) The image in the pinhole camera with a very small hole. (b) Image
density along the line $AB$.}
\label{fig:3.2}
\vspace*{-6pt}
\end{figure}

Diffraction does not fit in the corpuscular theory. Why on earth do
light corpuscles deviate from straight-line trajectories? Maybe this is
due to their interaction with the material of the walls surrounding the
hole? However, this explanation should be rejected, if only because the
diffraction pattern does not depend on the material -- paper or steel
-- from which the walls of the box are made.

The most striking evidence of the fallacy of the na\"{\i }ve corpuscular
theory of light is the \emph{interference} effect,\index{interference}
discovered by Young in 1802.\index{Young, T.} To see the interference, we can slightly
modify our pinhole camera: instead of one hole, make two holes that are
close to each other so that their diffraction blurs on the photographic
plate overlap. We already know that if we leave open the left hole and
close the right one, then we get a diffuse blur $ L $ (the left dashed
line in Figure~\ref{fig:3.3}\,(a)). If, on the contrary, we close the left
hole and open the right one, we get another diffuse blur $ R $. Let us
now try to predict what would happen if both holes are opened.

\begin{SCfigure}[50][h]
\includegraphics{01f03}
\caption{Image density in a two-hole camera. (a) In the na\"{\i }ve corpuscular
theory. (b) In reality.}
\label{fig:3.3}
\end{SCfigure}

Following the logic of the corpuscular theory, we could conclude that
the photons reaching the photographic plate are of two kinds: those that
have passed through the left and the right hole, respectively. If the
two holes are open simultaneously, then the density of ``left'' photons
should add with the density of ``right'' photons, and the resulting
image $ L + R $ must be a superposition of the two images (solid line
in Figure~\ref{fig:3.3}\,(a)). Right? No, wrong! This seemingly logical
reasoning is at odds with the experiment. The actual image on the
photographic plate has an additional structure (brighter and darker
areas shown by solid line in Figure~\ref{fig:3.3}\,(b)), called the \emph{interference pattern}. There are regions where the image density is higher
than $ L + R $ (constructive interference) and regions with the density
lower than $ L + R $ (destructive interference).

How can corpuscular theory explain this strange interference pattern?
For example, we could assume that there is some interaction between
light corpuscles, so that the passage of particles through the left and
right holes are not independent events, and the law of addition of
probabilities is not applicable to them. However, this idea should be
rejected, because the interference pattern does not disappear even if
we release the photons one by one, so that their interaction is excluded.

For example, in a two-hole interference experiment performed by Taylor
in 1909~\cite{Taylor},\index{Taylor, G.\,I.} the light intensity was so low that no more
than one photon was present in the camera at any given time. This
removed any possibility of interaction between photons and any effect
of this interaction on the interference pattern. Does this mean that a
single photon can interfere with itself? Maybe the photon somehow splits
apart, passes through both holes and then reconnects before colliding
with the photographic plate? This explanation also does not stand up to
criticism, because one photon can blacken only one grain of the
emulsion. Nobody has ever seen a ``half-photon''.

Perhaps, a particle passing through the right hole somehow knows whether
the left hole is open or closed, and adjusts its trajectory,
accordingly? This just does not make sense, and we have to admit that
our simple corpuscular theory does not have a logical explanation for all these observations.

\vspace*{-3pt}
%s1.2 #&#
\subsection{Wave theory of light}\label{ss:wave}
\vspace*{-3pt}

The inability to explain such fundamental properties of light as
diffraction and interference was a heavy blow to the Newtonian
corpuscular theory. These effects, like other light properties known in
the pre-quantum era (reflection, refraction, polarization, etc.) were
brilliantly explained by the \emph{wave theory}\index{wave theory} of
light developed by Huygens, Young, Fresnel and others.\index{Huygens, C.}\index{Young, T.}\index{Fresnel, A.-J.} During the 19th
century, the wave theory gradually supplanted the Newtonian corpuscles.
The idea that light is a wave process received its strongest support
from the Maxwell theory,\index{Maxwell, J.\,C.} which combined optics with electromagnetic
phenomena. This theory explained that light is made of oscillating
electric $ \boldsymbol{E}(t, \boldsymbol{r}) $ and magnetic
$ \boldsymbol{B}(t, \boldsymbol{r}) $ fields -- sinusoidal waves
propagating with the speed of light $ c $. According to Maxwell, the
energy of this wave and, accordingly, the intensity of light $ I $ is
proportional to the square of the amplitude of the field vectors:
$ I \propto \boldsymbol{E}^{2} $. Then, from the point of view of the
wave theory, the formation of the photographic image can be explained
as follows.
%
\begin{quote}
\textbf{\textit{Wave theory:}} Light is a continuous oscillating wave or
field propagating through space. When a light wave meets with molecules
of the photographic emulsion, charged parts of the molecules begin to
oscillate under the action of the electric and magnetic vectors in the
light field. In those places where the amplitude of the electromagnetic
oscillations is maximal, the charges of the molecules are subjected to
the strongest force, and the density of the photographic image is the
highest.
\end{quote}
%
This model explains both diffraction and interference in a fairly
natural way: diffraction simply means that light waves are capable of
going around obstacles, just like other types of waves (sea waves, sound
waves, etc.) do.\footnote{The wavelength of the visible light varies
between 0.4 micron for violet light and 0.7 microns for red light. So,
for large obstacles or holes, the effect of diffraction is very small,
and the corpuscular theory of light works quite well.} To explain the
interference at two holes, it is sufficient to note that when two parts
of a monochromatic wave pass through different apertures and meet on a photographic
plate, their electric (and magnetic) vectors add up. However, the wave
intensities are proportional to the squares of the vectors and,
therefore, are not additive: $ I \propto (\boldsymbol{E}_{1} +
\boldsymbol{E}_{2}) ^{2} = \boldsymbol{E}_{1} ^{2} + 2 \boldsymbol{E}
_{1} \cdot \boldsymbol{E} _{2} + \boldsymbol{E} _{2} ^{2} \neq
\boldsymbol{E}_{1} ^{2} + \boldsymbol{E}_{2} ^{2} \propto I_{1} + I
_{2} $. From simple\vadjust{\vspace*{-6pt}\eject} geometric considerations it follows that there are
places where these two waves always come with the same phase ($
\boldsymbol{E}_{1} \uparrow \uparrow \boldsymbol{E}_{2} $ and
$ \boldsymbol{E}_{1} \cdot \boldsymbol{E}_{2}> 0 $), which means
\emph{constructive interference}, and there are other places where the
waves come out of phase ($ \boldsymbol{E}_{1} \uparrow \downarrow
\boldsymbol{E}_{2} $ and $ \boldsymbol{E}_{1} \cdot \boldsymbol{E}
_{2} <0 $), i.\,e., \emph{destructive interference}.

\vspace*{-3pt}
%s1.3 #&#
\subsection{Light of low intensity and other experiments}\label{ss:low-intensity}
\vspace*{-3pt}

In the 19th century physics, the particle--wave dispute was
resolved in favor of the wave theory of light. However, further
experiments showed that the victory was declared prematurely. To
understand the problems of the wave theory, let us continue our thought
experiment with the interference pattern. This time, we will gradually
reduce the intensity of the light source. At first we will not notice
anything unusual: quite predictably the image density on the
photographic plate will decrease. However, from a certain point we will
notice that the image ceases to be uniform and continuous, as before.
We will see that it consists of separate dots, as if light were incident
on some grains of the emulsion and did not touch others. This
observation is difficult to explain from the point of view of the wave
theory. How can a continuous wave create this dotted image? But the
corpuscular theory copes easily: obviously, these dots are created by
separate particles (photons), which bombard the surface of the
photographic plate.

In the late 19th and early 20th centuries, other experiments appeared
that challenged the wave theory of light. The most famous of them was
the photoelectric effect:\index{photoelectric effect} it was found out that when light falls on a
piece of metal, it can knock electrons out of the metal into the vacuum.
In itself, this discovery was not surprising. However, it was surprising
how the number of knocked-out electrons depended on the frequency of
light and its intensity. It was found out that only light with a
frequency above a certain threshold $ \omega_{0} $ could knock out
electrons from the metal. Light of a lower frequency was unable to do
this, even if its intensity was very high. Why was this observation so
surprising? From the point of view of the wave theory, it could be
assumed that the electrons are emitted from the metal by the forces
originated from the electric $ \boldsymbol{E} $ and magnetic
$ \boldsymbol{B} $ fields in the light wave. The higher intensity of
light ($=$~the larger magnitudes of the vectors $ \boldsymbol{E} $ and
$ \boldsymbol{B} $) naturally means a higher force acting on the
electrons and a greater probability of the electron emission. So why
could not intensive low-frequency light cope with this work?

In 1905, Einstein\index{Einstein, A.} explained the photoelectric effect by returning to the
long-forgotten Newtonian corpuscles in the form of light quanta, later
called \emph{photons}.\index{photon} Einstein described light as
``$\ldots$ consisting of finite number of energy quanta which are localized at
points in space, which move without dividing and which can only be
produced and absorbed as complete units.'' \cite{Arons}. According
to Einstein, each photon carries the energy $ \hbar \omega $, where
$ \omega $ is the light frequency\footnote{$ \omega $ is the so-called
\emph{cyclic frequency} (measured in radians per second), which is related
to the ordinary frequency $ \nu $ (measured in cycles per second) by
formula $ \omega = 2 \pi \nu $.} and $ \hbar $ is the \emph{Planck}
\emph{constant}.\index{Planck constant}\index{Planck, M.} Each photon has a chance to
encounter only one electron in the metal and transfer its energy to it.
Only high-energy photons (i.\,e., photons present in high-frequency light)
can transmit enough energy to the electron to overcome the energy
barrier $ E_{b} $ between the volume of the metal and the vacuum.
Low-frequency light has low-energy photons $ \hbar \omega <E_{b}
\approx \hbar \omega_{0} $. Hence, regardless of the intensity ($=$~the
number of photons) of such light, its photons are simply too weak and
unable to kick the electrons strong enough to overcome the
barrier.\footnote{In fact, low-frequency light can lead to the
electron emission when two low-energy photons collide simultaneously
with the same electron. But such events are unlikely and become
noticeable only at very high light intensities.}

In the Compton experiments (1923), the\index{Compton scattering}\index{Compton, A.}
interaction of X-ray radiation with free electrons was studied in much
detail, and indeed, this interaction was more like a collision of two
particles than a shaking of the electron by a periodic electromagnetic
wave.

These observations should confirm our conclusion that light is a stream
of corpuscles, as Newton said. But how about interference? We have
already established that corpuscular theory is unable to give a logical
explanation for this effect!

So, the young quantum theory faced the seemingly impossible task of
reconciling two classes of experiments with light. Some experiments
(diffraction, interference) were easily explained within the framework
of the wave theory of light, but did not agree with the corpuscles --
photons. Other experiments (photoelectric effect, Compton scattering)
contradicted the wave properties and clearly indicated that light
consists of particles. To all this confusion, in 1924 de Broglie added\index{de Broglie, L.}
the hypothesis that the \emph{particle}--\emph{wave dualism}\index{particle-wave dualism}
 is characteristic not only of photons.
He argued that all material particles -- for example, electrons -- have
wave properties. This ``crazy'' idea was soon confirmed by Davisson and
Germer, who observed the interference of electron beams in 1927.\index{Davisson, C.}\index{Germer, L.}

Without a doubt, in the first quarter of the 20th century, physics
approached the greatest crisis in its history. Heisenberg described this
situation as follows.\index{Heisenberg, W.}
%
\begin{quote}
I remember discussions with Bohr which went through many hours till very
late at night and ended almost in despair; and when at the end of the
discussion I went alone for a walk in the neighboring park I repeated
to myself again and again the question: Can nature possibly be as absurd
as it seemed to us in those atomic experiments? -- W. Heisenberg
\cite{Heisenberg}.
\end{quote}

%s2 #&#
\section{Classical logic}\label{sc:1.1}
In order to advance in our understanding of the paradoxes mentioned
above, we need to go beyond the framework of classical physics.
Therefore, to begin with, we are going to outline this framework, i.\,e.,
to look at classical mechanics. For simplicity, we consider the
classical description of a single particle in a one-dimensional space.

%s2.1 #&#
\subsection{Phase space of one classical particle}\label{sc:1.1.1}
The \emph{state} $\phi $
\label{lb:state}\index{state}
of a classical particle is completely and uniquely determined by
specifying the particle's position $x$ and momentum $p$.\footnote{It
might seem more natural to work with the particle's velocity $v$ instead
of its momentum. However, we will see later that our choice of primary
observables has a special meaning, because $x$ and $p$ are ``canonically
conjugated'' variables.} Such states will be called\index{pure classical state} \emph{pure classical}.

Thus, all possible states of a one-particle system are labeled by a pair
of numbers $(x,p)$ and can be represented by points on a plane, which
will be called the \emph{phase space}\index{phase space} of the system and
denoted $ \mathcal{S} $ (see Figure~\ref{fig:4.1}).
\label{lb:phase-space}
Then particle dynamics is represented by lines ($=$~trajectories) in the
phase space $ \mathcal{S} $.\index{trajectory}

\begin{SCfigure}[50][h]
\includegraphics{01f04}
\caption{Phase space of a particle in one spatial dimension.}
\label{fig:4.1}
\end{SCfigure}


%s2.2 #&#
\subsection{Propositions in phase space}\label{sc:1.2.1}
In order to make it easier to switch to the quantum description in the
future, let us introduce the concept of experimental (or logical)
\emph{proposition}, sometimes also called
\emph{yes}--\emph{no question}.\index{proposition}\index{yes-no experiment}\label{lb:proposition}
The most obvious are propositions about individual observables. For
example, the proposition $ \mathcal{A} = $ ``the particle position is
in the interval $a$'' is meaningful. In each pure state of the system,
this proposition can be either true or false (the question answered
``yes'' or ``no'').

Experimentally, such a proposition can be realized using a
one-dimensional ``Geiger counter'',\index{Geiger, H.} which occupies the region $a$ of
space. The counter clicks ($=$~the proposition is true) if the particle
passes through the counter's discharge chamber and does not click ($=$~the
proposition is false) if the particle is outside the region $a$.

Propositions can be represented in the phase space. For example, the
above proposition $ \mathcal{A} $ is associated with the strip $ A $ in
Figure~\ref{fig:4.1}. The proposition is true if the point representing
the state $ \phi $ is inside the strip $ A $. Otherwise, the proposition
is false.%\label{lb:state}

Similarly, propositions about the momentum are represented by strips
parallel to the $x$ axis. For example, the strip $ B $ in Figure~\ref{fig:4.1} corresponds to the logical proposition $ \mathcal{B} = $
``the particle momentum belongs to the interval $ b $.''

We will denote by $\mathcal{L}$ the
\label{lb:set_of_allp}
set of all propositions about the physical system.\footnote{$
\mathcal{L} $ is also called the \emph{propositional system}
or \emph{logic}.\index{logic}}\index{propositional system} In the rest of this section we will study the
structure of this set and establish its connection with the classical
\emph{Boolean logic}.\index{Boolean logic}\index{Boole, G.} The set of all possible states
of the system will be denoted $\mathfrak{S}$.
\label{lb:set_of_alls}
In this chapter, our goal is to study the mathematical relationships
between the elements $ \mathcal{X} \in \mathcal{L} $ and $ \phi
\in \mathfrak{S} $ in these two sets.

%s2.3 #&#
\subsection{Operations with propositions}\label{sc:1.2.2}
In classical theory, in addition to the above propositions $
\mathcal{A} $ and $\mathcal{B} $ about single observables, we can also
associate a proposition with each region of the phase space. For
example, in Figure~\ref{fig:4.1} we showed a rectangle $C$, which is an
intersection of the two strips $ C = A \cap B $. Apparently, this
rectangle also corresponds to an admissible proposition $ \mathcal{C} = \mathcal{A}
\wedge \mathcal{B} = $ `the particle position is in the interval $a$
\emph{and} its momentum is in the interval $b$.''
\label{lb:wedge}
In other words, this proposition is obtained by applying the logical
operation ``AND'' to the two elementary propositions $ \mathcal{A}$ and
$ \mathcal{B}$. We denote this logical operation (\emph{meet}) by the symbol
\label{lb:vee}\index{meet}
$\mathcal{A} \wedge \mathcal{B} $.\footnote{This symbol differs from
the symbol $ A \cap B $ for the intersection of two regions in the phase
space, thereby emphasizing that we are dealing with the logical
operation ``AND'', which relates specifically to propositions. In
classical logic, there is an equivalence between propositions and
regions in the phase space $ \mathcal{S} $, so having two different
notations may seem superfluous. However, in the quantum case, such an
equivalence is lost, the idea of the phase space is not applicable and
only the logical notation $\mathcal{A} \wedge \mathcal{B} $ makes sense.}

%t1 #&#
\begin{table}[b!]
\def\arraystretch{1.05}
%\vspace*{12pt}
\tabcolsep=0pt
\caption{Five operations and two special elements in the theory of subsets of
the phase space $\mathcal{S}$, in the classical logic and in the lattice
theory.\index{negation}\index{implication}}
\label{table:2.2}
\begin{tabular*}{\textwidth}{@{\extracolsep{4in minus 4in}}lllll}
\multicolumn{1}{L{1.7cm}}{Symbol for subsets in $\mathcal{S}$}
& \multicolumn{1}{L{1cm}}{Name in logic}
& \multicolumn{1}{L{1.8cm}}{Meaning in classical logic}
& \multicolumn{1}{L{1.7cm}}{Name in lattice theory}
& \multicolumn{1}{L{1.67cm}@{}}{Symbol in lattice theory} \\
\midrule\starttabularbody
\multicolumn{5}{l}{\textit{Operations with subsets/propositions}} \\
$X \subseteq Y$ & implication & $\mathcal{X}$ IMPLIES $\mathcal{Y}$& less or equal & $\mathcal{X} \leq \mathcal{Y}$ \\
$X \subseteq Y$, $X \neq Y$ & implication & $\mathcal{X}$ IMPLIES $\mathcal{Y}$& less & $\mathcal{X} < \mathcal{Y}$ \\
$X \cap Y$ & conjunction & $\mathcal{X} $ AND $\mathcal{Y}$ & meet & $\mathcal{X} \wedge \mathcal{Y}$ \\
$X \cup Y$ & disjunction & $\mathcal{X} $ OR $\mathcal{Y}$ & join & $\mathcal{X} \vee \mathcal{Y}$ \\
$\mathcal{S} \setminus X$ & negation & NOT $\mathcal{X}$ & orthocomplement & $\mathcal{X}^{\perp }$ \\
\hline
\multicolumn{5}{l}{\textit{Special subsets/propositions}} \\
$\mathcal{S}$ & tautology & always true & maximal element & $\mathcal{I}$ \\
$\emptyset _{\mathcal{S}}$ & absurdity & always false & minimal element & $\emptyset $ \\
\end{tabular*}
\end{table}

The four other logical operations listed in Table \ref{table:2.2} are
also naturally defined in the language of propositions--regions. For
example, the rectangle $ C = A \cap B $ lies entirely inside the strip
$ A $. From the point of view of logic, we can say that proposition~$ \mathcal{C} $
 ``IMPLIES'' proposition $ \mathcal{A} $. Indeed, in any
state where $ \mathcal{C} $ is true, the proposition $ \mathcal{A} $ is
also true. This logical connection will be denoted by $ \mathcal{C}
\leq \mathcal{A} $.\footnote{If $ \mathcal{C} $ ``IMPLIES''
$ \mathcal{A} $ and definitely $ \mathcal{A} \neq \mathcal{C} $, then
we will use the symbol $ \mathcal{C} < \mathcal{A} $.\label{lb:lessless}}\label{lb:lessthan}

The proposition $ \mathcal{A} $ ``OR'' $ \mathcal{B} $ corresponds to
the union ($ A \cup B $) of two regions in the phase space. This
proposition will be written as $ \mathcal{A} \vee \mathcal{B} $. If
either $ \mathcal{A} $ or $ \mathcal{B} $ is true, then the
\emph{join} $ \mathcal{A} \vee \mathcal{B} $ is definitely true.\index{join}

The last operation is the complement of a phase-space region $ A
$.\footnote{That is, the region consisting of phase-space points not
belonging to $A$.} Obviously its logical equivalent is the negation of
the proposition $ \mathcal{A} $, which we denote by $ \mathcal{A}^{
\perp } $ ($=$~``NOT''$ \mathcal{A} $, \emph{orthocomplement}).\label{lb:orthocomplement}\index{orthocomplement}

In addition to these four operations, we will need two special
propositions, listed in Table \ref{table:2.2}.

\emph{Maximal}\index{maximal proposition}\index{tautology} proposition (or
\emph{tautology}) $ \mathcal{I} \in \mathcal{L} $ corresponds to the whole
phase space, i.\,e., the maximal subset of $ \mathcal{S} $. This
proposition can be expressed in different verbal forms. For example:
$ \mathcal{I} = $ ``particle position is somewhere on the real axis''
or $ \mathcal{I} = $ ``particle momentum is somewhere on the real
axis.'' Both these propositions are always true for any
state.\footnote{Measurements of observables always yield \emph{some}
result, because we agreed in the Introduction that an ideal measuring
device never misfires.}\label{lb:tautology}

Propositions like ``the value of the observable is not on the real
axis'' or ``the value of the observable lies in the empty subset of the
real axis'' are always false and equal to the single \emph{minimal} (or
\emph{absurd}) proposition\index{minimal proposition}\index{absurd proposition}
$ \emptyset $ in the set $ \mathcal{L} $.\label{lb:absurd}

%t2 #&#
\begin{table}
\tabcolsep=0pt
\caption{Basic axioms of classical and quantum logics.}
\label{table:2.1}
\begin{tabular*}{.7\textwidth}{@{\extracolsep{4in minus 4in}}rll}
& Name & Formula \\
\midrule\starttabularbody
 \multicolumn{3}{l}{\textit{Axioms  of  orthocomplemented lattices}} \\
1 & Reflectivity & $\mathcal{X} \leq \mathcal{X}$ \\
2 & Symmetry & $(\mathcal{X} \leq \mathcal{Y})  \ \&\   (\mathcal{Y} \leq \mathcal{X}) \Rightarrow  \mathcal{X}= \mathcal{Y} $ \\
3 & Transitivity & $(\mathcal{X} \leq \mathcal{Y})  \ \&\   (\mathcal{Y} \leq \mathcal{Z})  \Rightarrow  \mathcal{X}\leq \mathcal{Z}$ \\
4 & Definition of $\mathcal{I}$ & $\mathcal{X} \leq \mathcal{I}$ \\
5 & Definition of $\emptyset $ & $\emptyset \leq \mathcal{X}$ \\
6 & Definition of $\wedge $ & $\mathcal{X} \wedge \mathcal{Y} \leq \mathcal{X}$ \\
7 & Definition of $\wedge $ & $(\mathcal{Z} \leq \mathcal{X})  \ \&\   (\mathcal{Z} \leq \mathcal{Y}) \Rightarrow \mathcal{Z} \leq (\mathcal{X} \wedge \mathcal{Y})$ \\
8 & Definition of $\vee $ & $ \mathcal{X} \leq \mathcal{X} \vee \mathcal{Y} $ \\
9 & Definition of $\vee $ & $(\mathcal{X} \leq \mathcal{Z})  \ \&\  (\mathcal{Y} \leq \mathcal{Z}) \Rightarrow (\mathcal{X} \vee \mathcal{Y}) \leq \mathcal{Z} $ \\
10 & Commutativity & $\mathcal{X} \vee \mathcal{Y} = \mathcal{Y} \vee \mathcal{X}$ \\
11 & Commutativity & $\mathcal{X} \wedge \mathcal{Y} = \mathcal{Y} \wedge \mathcal{X}$ \\
12 & Associativity & $(\mathcal{X} \vee \mathcal{Y}) \vee \mathcal{Z} = \mathcal{X} \vee (\mathcal{Y} \vee \mathcal{Z})$ \\
13 & Associativity &$(\mathcal{X} \wedge \mathcal{Y}) \wedge \mathcal{Z} = \mathcal{X} \wedge (\mathcal{Y} \wedge \mathcal{Z})$ \\
14 & Noncontradiction & $\mathcal{X} \wedge \mathcal{X}^{\perp } = \emptyset $ \\
15 & Noncontradiction & $\mathcal{X} \vee \mathcal{X}^{\perp } = \mathcal{I}$ \\
16 & Double negation & $(\mathcal{X}^{\perp })^{\perp } = \mathcal{X}$ \\
17 & Contraposition & $\mathcal{X} \leq \mathcal{Y} \Rightarrow \mathcal{Y}^{\perp } \leq \mathcal{X}^{\perp }$ \\
\hline
 \multicolumn{3}{l}{\textit{Additional assertions of classical logic}} \\
18 & Distributivity & $ \mathcal{X} \vee (\mathcal{Y} \wedge \mathcal{Z}) = (\mathcal{X} \vee \mathcal{Y}) \wedge (\mathcal{X} \vee \mathcal{Z})$ \\
19 & Distributivity & $ \mathcal{X} \wedge (\mathcal{Y} \vee \mathcal{Z}) = (\mathcal{X} \wedge \mathcal{Y}) \vee (\mathcal{X} \wedge \mathcal{Z}) $\\
\hline
 \multicolumn{3}{l}{\textit{Additional postulate of quantum logic}}
\\
20 & Orthomodularity & $ \mathcal{X} \leq \mathcal{Y} \Rightarrow \mathcal{X}\leftrightarrow \mathcal{Y} $ \\
\end{tabular*}
\end{table}

%s2.4 #&#
\subsection{Axioms of logic}\label{sc:1.2.3}

Five operations and two special propositions, presented above, define
a rich mathematical structure. To work with these objects, it is
necessary to establish their mutual relations, i.\,e., laws (or axioms)
of logic.

The easiest way to establish these laws is to use the equivalence
between logical propositions and subsets of the phase space. This means
that the properties of logical operations (``IMPLIES,'' ``AND,'' ``OR,''
``NOT'') coincide with the properties of operations on subsets
(``inclusion,'' ``intersection,'' ``union,'' ``complement''). From this
analogy, it is not difficult to obtain the laws of classical logic
listed in lines 1 through 19 of Table \ref{table:2.1}.\footnote{Actually,
the choice of the axioms of logic is rather arbitrary. There are
different approaches to the axiomatization of logic, and our approach
is not the most economical. We tried to select our axioms so that they
had the most transparent meaning.}\vadjust{\vspace*{12pt}\eject}

\nn
For example, the transitivity property 3 from Table \ref{table:2.1}
allows us to build syllogisms, such as the one analyzed by Aristotle:\index{Aristotle}
%
\begin{quote}
\emph{If all humans are mortal},

\emph{and all Greeks are humans},

\emph{then all Greeks are mortal}.
\end{quote}
%
Indeed, we have three propositions: $ \mathcal{X} = $ ``this is a
Greek,'' $ \mathcal{Y} = $ ``this is a human being'' and $
\mathcal{Z} = $ ``this is mortal.'' We know that $ \mathcal{X}$ implies
$\mathcal{Y} $ (i.\,e., $ \mathcal{X} \leq \mathcal{Y} $). We also know
that $ \mathcal{Y} $ implies $ \mathcal{Z}$ ($ \mathcal{Y} \leq
\mathcal{Z} $). Then the transitivity property tells us that
$ \mathcal{X}$ implies~$ \mathcal{Z}$ ($ \mathcal{X} \leq \mathcal{Z}$,
i.\,e., ``every Greek is mortal'').

Property 14 says that a proposition $ \mathcal{X} $ and its negation
$ \mathcal{X}^{\perp } $ cannot be true at the same time, i.\,e., their
meet $ \mathcal{X} \wedge \mathcal{X}^{\perp } $ is equal to the absurd
proposition $\emptyset $. Property 15 is the famous \emph{tertium non}
\emph{datur} law of logic: either the proposition $ \mathcal{X} $ or its
negation $ \mathcal{X}^{\perp } $ is true, and \emph{the third} \emph{is}
\emph{not given}.

A set of objects with operations and special elements from Table
\ref{table:2.2}, subject to properties 1--17 from Table
\ref{table:2.1}, is referred to as the \emph{orthocomplemented
lattice}\index{orthocomplemented lattice} by mathematicians.

Many useful logical relationships can be derived from the axioms of
orthocomplemented lattices. Some of them are formulated in the form of
lemmas and theorems in Appendix \ref{ss:theorems}. However, these axioms
1--17 are still not enough to describe the classical logic of
propositions unequivocally. Subsets of the phase space and propositions
of classical logic are subject to additional \emph{distributive}
\emph{laws}\index{distributive law} 18 and 19 in Table \ref{table:2.1}.

Like other properties in the upper portion of Table \ref{table:2.1}, the
distributive laws are easily derived from our analogy ``proposition''
$\leftrightarrow $ ``region of the phase space.'' Nevertheless, we put
these laws in a separate category. As we shall see in Section
\ref{sc:1.3}, it is these laws that determine the difference between
classical and quantum logics. In Table \ref{table:2.1} we call them
``assertions,'' because we do not consider them to be true in
fundamental quantum theory.\footnote{In our book we distinguish
\emph{postulates},\index{postulate} \emph{statements} and \emph{assertions}.\index{assertion}
Postulates form the basis of our theory. In many
cases, they undoubtedly follow from experiments, and we do not question
their validity. Statements follow logically from the Postulates, and we
consider them to be correct. Assertions refer to
claims\index{statement} that are made in other theories, but do not have place
in our approach (RQD).}

%s2.5 #&#
\subsection{Phase space from axioms of classical logic}\label{sc:1.2.4}
Thus, we have shown that in the phase space of classical mechanics the
set of all propositions $\mathcal{L} $ is an orthocomplemented lattice
with distributive laws 18 and 19. Such lattices will be called
\emph{Boolean algebras} or \emph{classical logics}.\footnote{Strictly
speaking, the definition of classical logic involves also a technical
condition of the lattice \emph{atomicity}.\index{atomic lattice} In our case
this means the existence of ``minimal nonzero'' propositions --
\emph{atoms},\index{atomic proposition} which correspond to points in the
phase space.}\index{classical logic}\index{Boolean algebra}

For us, it is very important that one can prove the converse statement,
which is the following.

%tchapter.1 #&#
\begin{theorem}[representation of classical logic]
\label{Theorem4.13}
For each classical logic $ \mathcal{L} $ defined by properties $1$--$19$
from Table \ref{table:2.1}, there exist a set $ \mathcal{S}
$\footnote{In classical mechanics, the set $ \mathcal{S} $ is called
the \emph{phase space}.} and an isomorphism $ h(\mathcal{X}) $ between
logical propositions $\mathcal{X} \in \mathcal{L} $ and subsets of
$ \mathcal{S} $, such that logical operations in $ \mathcal{L} $ match
with set-theoretical operations in $ \mathcal{S}$, as follows:
%
\begin{align*}
\mathcal{X} \leq \mathcal{Y} &\Leftrightarrow h(\mathcal{X}) \subseteq h(
\mathcal{Y}),
\\
h(\mathcal{X} \wedge \mathcal{Y}) &= h(\mathcal{X}) \cap h( \mathcal{Y}),
\\
h(\mathcal{X} \vee \mathcal{Y}) &= h(\mathcal{X}) \cup h(\mathcal{Y}),
\\
h \bigl( \mathcal{X}^{\perp } \bigr) &= \mathcal{S} \setminus h(
\mathcal{X}),
\\
h(\mathcal{I}) &= \mathcal{S,}
\\
h(\emptyset ) &= \emptyset_{\mathcal{S}};
\end{align*}
%
see Table \ref{table:2.2}.
\end{theorem}

The importance of this theorem lies in the possibility to derive
foundations of classical physics (e.\,g., the structure of the phase
space) from axioms of logic. Starting with the Boolean logic, we come
to the idea of the phase space, where states are represented by points.
From here it is not far to other elements of classical mechanics, such
as, for example, the description of dynamics by trajectories.

\vspace*{-3pt}
%s2.6 #&#
\subsection{Classical observables}\label{ss:observables}
\vspace*{-3pt}

In classical mechanics, an observable ($=$~physical quantity) $ F $ is
represented by a real function $ f: \mathcal{S} \to \mathbb{R} $ on the
phase space. To each point ($=$~state) of the phase space the function
$f$ associates a single number -- the value of the observable in this
state. Three examples of such observables/functions are shown in Figure~\ref{fig:4.12}. They are the position $ x $, the momentum $ p $ and the
energy $ H $ of a one-dimensional oscillator (a pendulum) with a
quadratic Hamiltonian. The values taken by the corresponding functions
$f$ are from the spectra of the observables. In the case of $ x $ and
$ p $, the spectrum is the entire real axis $ \mathbb{R} = (- \infty
, + \infty ) $, and the spectrum of $ H $ is the set of nonnegative
numbers $ [0, + \infty ) $.\looseness=-1

\begin{SCfigure}[50][h]
\includegraphics{01f05}
\caption{Observables in the language of propositions in the phase space: (a)
position $x$, (b) momentum $p$, (c) energy of the harmonic oscillator $H(x,p) =
p^{2}/(2m) + \alpha x^{2}$.}
\label{fig:4.12}
\end{SCfigure}

Each such function-observable $f$ defines a set of constant-value lines
$ x, p, H = \ldots , 1,2,\allowbreak 3,4, \ldots $ in $\mathcal{S}$ (shown in Figure~\ref{fig:4.12}), which in turn
can be interpreted as subsets $ \mathcal{S}_{f} $ or logical
propositions in the phase space. The proposition $ \mathcal{S}_{f}
\in \mathcal{L} $ is pronounced ``the observable $ F $ has the value
$ f $.'' Thus, each observable can be equivalently described as a map
$ \mathcal{F} $ from the spectrum of the observable into the set of all
propositions $ \mathcal{L} $. This
map\footnote{It is also called the \emph{proposition}-\emph{valued}
\emph{measure}.} has the following properties:
%
%
%3 items enumerated
%the first item: 1
%the last item: 3
% the (last) widest item: 3
\begin{enumerate}
\item[(1)]
The function $ \mathcal{F} $ associates to each point $ f $ of the
spectrum of the observable $ F $ one and only one logical proposition
$ \mathcal{S}_{f} \in \mathcal{L} $.
\item[(2)]
Propositions corresponding to different points $  ( f \neq f'
 ) $ of the spectrum are disjoint.\footnote{Two propositions
$ \mathcal{X} $ and $ \mathcal{Y} $ are called \emph{disjoint},\index{disjoint propositions}
 if $ \mathcal{X} \leq \mathcal{Y}^{
\perp } $ (or, equivalently, $ \mathcal{Y} \leq \mathcal{X}^{\perp }
$).} On the phase plane, such disjoint propositions correspond to
nonintersecting regions, otherwise we would have absurd states
possessing two different values of the same observable simultaneously.
\item[(3)]
The join (union) of the propositions $ \mathcal{S}_{f} $ over all
spectrum points is equal to the maximal (trivial) proposition ($ \vee
_{f} \mathcal{S}_{f} = \mathcal{I} $), which is equivalent to the entire
phase space. This condition indicates that in each state it is possible
to measure some value of the observable. There are no states ($=$~points
in the phase space) where the observable is not measurable.
\vspace*{-3pt}
\end{enumerate}
%
\begin{SCfigure}
\includegraphics{01f06}
\caption{Observable $H$ as a mapping $\mathcal{F}$ from the spectrum of $H$ into
the set of propositions in the phase space $S$. The function $\mathcal{F}$ maps
the spectrum interval $a=[2,4]$ to the subset-proposition $\mathcal{A}$.}
\label{fig:4.12x}
\end{SCfigure}
%
So, with each observable $ F $ and with each subset $ a $ of the real
axis $ \mathbb{R} $ we associate an experimental \emph{proposition}
$ \mathcal{A} = $ ``the value of the observable $ F $ is inside the
subset \mbox{$ a \subseteq \mathbb{R} $}.'' Obviously, $ \mathcal{A} $ is equal
to the join of elementary propositions $ \mathcal{S}_{f} $ over all
points of the spectrum lying inside the interval $ a $.  The mapping subset$ \to $proposition is illustrated in Figure~\ref{fig:4.12x}.

\vspace*{-3pt}
%s3 #&#
\section{Measurements and probabilities}\label{sc:measurements-and-probabilities}
\vspace*{-3pt}

In the previous section, we developed the classical logic of strictly
deterministic states in which the answer to any yes--no question could
be either definite ``yes'' or definite ``no.'' However, such states are
rarely found in real experiments. As a rule, measurements are associated
with randomness, uncertainties, errors, etc. To describe such
unpredictable outcomes we need the concepts of an ensemble and a
probability measure.\looseness=-1

\vspace*{-3pt}
%s3.1 #&#
\subsection{Ensembles and measurements}\label{ss:quantum-case}
\vspace*{-3pt}

We will call \emph{experiment}\index{experiment} a procedure for preparing
an \emph{ensemble}\footnote{Ensemble is a set of identical copies of the physical
system, made in -- as much as possible -- the same conditions.}\index{ensemble}
 and measuring the same observable in each member of the
ensemble.\footnote{It is important to note that in this book we do not
consider repeated measurements performed on the same copy of the
physical system. We will assume that after the measurement has been
made, the used copy of the system is discarded. A fresh copy is required
for each new measurement. In particular, this means that we are not
interested in the state of the system after the measurement. The
description of successive measurements in one instance of a physical
system is an interesting task, but it is beyond the scope of our book.\label{foot: repeat}}

So, let us prepare many copies of the system, all in one state
$\phi $ ($=$~ensemble) and perform measurements of the same proposition
$ \mathcal{X} $ in all these copies. As we already know, there is no
guarantee that the outcomes of these measurements will be the same.
Hence, for some members of the ensemble the proposition $ \mathcal{X}
$ will be found true, and for other members it will be false. Using
these data, we can introduce a function $ (\phi | \mathcal{X}) $, which
we call \emph{probability measure}\index{probability measure} and which
associates to each state $ \phi $ and each proposition $ \mathcal{X} $
the probability that $ \mathcal{X} $ is true in the state (ensemble)
$ \phi $. The value of this function (a real number in the interval
between 0 and 1) is obtained as a result of the following steps:
%
\begin{enumerate}
\item[(i)] prepare an instance of the system in the state $\phi $;
\item[(ii)] make a measurement and determine whether the proposition
$\mathcal{X}$ is true or false;
\item[(iii)] repeat steps (i) and (ii) $N$ times and calculate the
probability by the formula
%
\begin{align*}
(\phi |\mathcal{X}) = \lim_{N \to \infty } \frac{M}{N},
\end{align*}
%
where $M$ is the number of times the proposition $\mathcal{X}$ was found
true.
\end{enumerate}
%
In order to obtain the most complete description of the physical system,
it is necessary to perform such experiments with all possible
propositions $ \mathcal{X} \in \mathcal{L} $ for all possible ensembles
($=$~states) $ \phi \in \mathfrak{S} $.

%s3.2 #&#
\subsection{States as probability measures}\label{sc:1.2.33}
If we are not too lazy to complete all such measurements, we will notice
that the probability measure $ (\phi | \mathcal{X}) $ has the following
properties:\label{lb:measure}\index{state}
%
\begin{itemize}
\item  The probability corresponding to the maximal (trivial)
proposition is 1 in all states, so\looseness=-1
%e1 #&#
\begin{align}
(\phi |\mathcal{I}) = 1. \label{eq:prob1}
\end{align}

\item  The probability corresponding to the minimal (absurd)
proposition is 0 in all states, so
%e2 #&#
\begin{align}
(\phi |\emptyset ) = 0. \label{eq:prob2}
\end{align}

\item  The probability corresponding to the join of disjoint
propositions is the sum of individual probabilities, so
%e3 #&#
\begin{align}
(\phi |\mathcal{X} \vee \mathcal{Y}) = (\phi |\mathcal{X}) + (\phi |
\mathcal{Y}), \quad \mbox{if } \mathcal{X} \leq \mathcal{Y}^{\perp }.
\label{eq:prob3}
\end{align}
\end{itemize}
%
The first two statements follow directly from definitions of special
logic elements $ \mathcal{I} $ and $ \emptyset $. The third statement
is known as the \emph{third Kolmogorov} \emph{probability axiom}: the
probability of observing either one of the two (or several) mutually
exclusive events is equal to the sum of event probabilities.\index{Kolmogorov axiom}\index{Kolmogorov, A.\,N.}

%s3.3 #&#
\subsection{Probability distributions and statistical mechanics}\label{sc:1.1.3}
In classical physics, the description of random events is handled by
\emph{statistical mechanics}.\index{statistical mechanics} In this
discipline, states that have an element of randomness are called
\emph{mixed classical} states.\index{mixed state} Mathematically, they are
represented by \emph{probability distributions},\index{probability distribution}
 which are functions $ \rho (x,p) $ on
the phase space that
%
%
%3 items enumerated
%the first item: 1
%the last item: 3
% the (last) widest item: 3
\begin{enumerate}
\item[(1)]
are nonnegative: $ \rho (x,p) \geq 0$;
\item[(2)]
normalized (their integral over the entire phase space is equal to 1),
%
\begin{align*}
\int_{-\infty }^{+\infty } dx \int_{-\infty }^{+\infty
} dp \rho (x,p) =
1;
\end{align*}
\item[(3)]
express the probability of the answer \emph{yes} to the question $ \mathcal{X} $
 by the formula
%e4 #&#
\begin{align}
(\phi | \mathcal{X}) = \int_{X} dx dp \rho (x,p), \label{eq:prob-sum}
\end{align}
%
where $ X $ is the region of the phase space corresponding to the
question ($=$~proposition) $\mathcal{X}$.
\end{enumerate}
%
By combining the notion of probability distribution with the laws of
logic 1--19 from Table \ref{table:2.1}, we can arrive at the classical
theory of probability. But we will not dwell on it here. In the next two
sections, we will be more interested in quantum logic and quantum
probability theory. Here, we will finish our discussion of classical
probabilities with a few remarks about determinism.

The randomness present in mixed classical states is usually associated
with our inability to provide identical preparation conditions for all
members in the ensemble. For example, when we throw a die, it falls in
an accidental, unpredictable manner. However, we believe that this
unpredictability is simply due to our inability to strictly control the
movement of our hand. Thus, classical randomness and probabilities are
technical in nature rather than fundamental.

Therefore, classical physics is based on one tacitly assumed axiom,
which we formulate here as an assertion.

%achapter.2 #&#
\begin{assertion}[full determinism]
\label{assertionJ_c}It is possible to prepare such ensembles (states) of the physical system
where measurements of all observables produce the same result every time.
In other words, we assume the existence of pure classical states
representable by points in the phase space. \end{assertion}

Pure states are also representable in the language of probability
distributions. They correspond to delta-like functions $ \rho (x, p) =
\delta (x-x_{0}) \delta (p-p_{0}) $ on the phase space. Then formula
(\ref{eq:prob-sum}) confirms that in such states all experimental
results are deterministic. A proposition $\mathcal{X} $ is either true
($ (\phi | \mathcal{X}) = 1 $), if the point $(x_{0}, p_{0}) $ of the
phase space belongs to the subset $ X $, or false ($ (\phi |
\mathcal{X}) = 0 $) otherwise, without any intermediate possibility.

%s4 #&#
\section{Logic of quantum mechanics}\label{sc:1.3}
To clarify the basic ideas of quantum mechanics, let us return to the
experiment with photons passing through one hole (see Subection
\ref{ss:low-intensity}). We found out that in the low intensity regime, when
the photons are emitted one by one, the image on the screen (or
photographic plate) consists of individual dots, which are randomly
distributed sites of particle (photon) hits. This means that results of
measuring the photon position are not reproducible, even if the state
preparation conditions are controlled in the most careful way!

From this we conclude that the behavior of photons involves some
\emph{random} element. This is the most fundamental statement of quantum
mechanics.\footnote{In Subection \ref{ss:states}, we will see that this
statement is, in fact, a consequence of the even more fundamental
Postulate \ref{postulateK13}.}

%schapter.3 #&#
\begin{statement}[fundamental randomness]
\label{postulateJ2}Measurements in the microworld have an element of randomness. This
randomness is fundamental and cannot be explained or reduced \textup{(}as we did
in the classical case\textup{)} to some inaccuracies in the preparation of
initial states or experimental errors.
\end{statement}

By adopting this Statement, we conclude that classical Assertion
\ref{assertionJ_c} (complete determinism) is incorrect. In the pinhole
camera setup, it is impossible to prepare such an ensemble of photons,
in which all of them hit the same point on the screen. What is the
reason for this scatter? Honestly, no one knows. This is one of the
greatest mysteries of nature. Quantum theory does not even attempt to
explain the physical causes of such a random behavior of microsystems.
This theory takes randomness as a given and simply tries to find its
mathematical description. To proceed, we have to move beyond the simple
declaration of randomness and introduce more precise statements and
definitions.

%s4.1 #&#
\subsection{Partial determinism of quantum mechanics}\label{sc:1.3.2}
We begin our construction of the formalism from the following postulate.

%pchapter.4 #&#
\begin{postulate}[connection between states and propositions]
\label{postulateJ}For each yes--no question there is an ensemble in which the answer
``yes'' is found with certainty (the probability $=$ 1).
\end{postulate}


Indeed, it makes no sense to talk about an experimental proposition, if there is not a single ensemble in which this proposition
can be unambiguously measured.

In Subsection \ref{ss:observables}, we identified observables with mappings
from $\mathbb{R}$ into the set of yes--no questions. Thus, each value
$f$ of the observable $F$ maps to a proposition $S_{f}$. According to
Postulate \ref{postulateJ}, we can always prepare an ensemble in which
this proposition is 100\,\% true.

%schapter.5 #&#
\begin{statement}[partial determinism]
\label{postulateJ1}For each observable $ F $ and each value $ f $ from its spectrum, an
ensemble can be prepared in which measurements of this observable are
reproducible, i.\,e., repeatedly yield the same value $ f $.
\end{statement}

Postulate \ref{postulateJ} and Statement \ref{postulateJ1} are weakened
versions of the classical Assertion~\ref{assertionJ_c}. Instead of
requiring the reproducibility of measurements for all observables and
propositions at once, we limit this property to single
observables.\footnote{Notice also that Statement \ref{postulateJ1}
does not forbid the existence of certain groups of (\emph{compatible})
observables,\index{compatible observables} whose measurements can be
reproducible within the same ensemble. For example, in Chapter
\ref{ch:operators}, we will see that three components $ (p_{x}, p_{y},
p_{z}) $ of the particle momentum are compatible observables. The same
is true for three components $ (r_{x}, r_{y}, r_{z}) $ of the particle
position. However, the pairs $ (p_{x}, x) $, $ (p_{y}, y) $ and
$ (p_{z}, z) $ are incompatible.} Hence the quantum postulate is a
softer requirement, and quantum mechanics is a more general theory than
classical mechanics. Moreover, we expect the quantum theory to include
classical mechanics as a special case.

So, in quantum mechanics we do not question the existence of
propositions about one observable. This means that propositions
represented by the strips $ A $ and $ B $ in Figure~\ref{fig:4.1} continue
to have well-defined meanings.\footnote{That is, there are ensemble
states in which these propositions are true.} However, the quantum and
classical approaches diverge when it comes to propositions involving
more than one observable. For example, in quantum mechanics we cannot
guarantee that the proposition corresponding to the rectangle
$ C = A \cap B $ in Figure~\ref{fig:4.1}\footnote{This is a proposition
about simultaneous measurement of both the position and the momentum of
one particle.} exists and can be realized in the form of an instrumental
setup.

Heisenberg\index{Heisenberg, W.} was the first to question the simultaneous measurability of
certain pairs of observables. He gave the following heuristic arguments.
Imagine that we want to accurately measure both the position and the
momentum of an electron. For this, we have to look through the
microscope.\index{Heisenberg microscope} To see the electron, we have
to illuminate it. For a more accurate determination of the position we
should use light with a short wavelength. However, photons of this light
have high energy (momentum). Colliding with the electron under study,
such photons will inevitably give it a kick, making it
impossible to accurately determine the electron's velocity or momentum.

So, we suspect that for sufficiently narrow strips $A$ and $B$ in Figure~\ref{fig:4.1} their intersection $ C = A \cap B $ may be just ``too small''
to correspond to any real experimental proposition. Thus, there are no
experimental propositions about points in the phase space. In other
words, in nature there is no device that could realize the proposition
``the particle's position is $ x_{0} $ and the particle's momentum is
$ p_{0} $.'' This also means that the true lattice of propositions
cannot coincide with the Boolean lattice of subsets in the phase space.
What can we do? What lattice should we take to build the logic of
questions in quantum physics?

Our plan for constructing quantum theory is as follows:
%
\begin{itemize}
\item  First, we will establish the logic of propositions in
our theory. We will call it \emph{quantum logic}. As we saw above, we
expect it to differ from the classical Boolean logic ($=$~orthocomplemented distributive lattice).
\item  Next, we will formulate the representation theorem of
Piron, which asserts that propositions of quantum logic can be
represented by subspaces in a Hilbert space.
\item  From this result it will be easy to derive all basic
properties of the quantum formalism: the superposition principle, the
probability interpretation of wave functions, observables as Hermitian
operators, etc.
\end{itemize}
%
%s4.2 #&#
\subsection{Axioms of quantum logic from probability measures}\label{ss:compatibility1}
Note that in our derivation of the axioms of classical logic we used the
equivalence between logical propositions and subsets of the phase space.
In the quantum case, this equivalence does not work, and we have to look
for other ways. To implement the first point of our plan, we will show
that many axioms from Table \ref{table:2.1} can be derived even without
reference to the phase space.\footnote{These axioms will be
transferred without changes from the classical logic to the quantum
one.} For these derivations we will need only the simplest properties
of probability measures $ (\phi | \mathcal{X}) $.

Suppose that we have prepared two state ensembles $ \phi $ and
$ \psi $ of our physical system and measured values of the probability
measures $ (\phi | \mathcal{X}) $ and $ (\psi | \mathcal{X}) $ by going
over all possible experimental propositions $ \mathcal{X} $. If, as a
result of this gigantic work, we find that $ (\phi |\mathcal{X}) = (\psi |\mathcal{X}) $
for all $ \mathcal{X} $, then the states $ \phi $ and $ \psi $ will be
regarded as equal ($ \phi = \psi $). Indeed, there is no physical
difference between these two states, where measurements give the same
results ($=$~probabilities).

For similar reasons, we will say that two propositions $ \mathcal{X} $
and $ \mathcal{Y} $ are equal ($ \mathcal{X} = \mathcal{Y} $) if for all
states $ \phi $\vspace*{-3.5pt}
%
%e5 #&#
\begin{align}
(\phi |\mathcal{X}) = (\phi |\mathcal{Y}). \label{eq:(4.1)}
\end{align}
%
It then follows that the probability measure $ (\phi | \mathcal{X}) $,
considered as a function on the set of all states $ \mathfrak{S} $, is
a unique representative of the proposition $ \mathcal{X}
$.\footnote{That is, different propositions define different functions
$ (\phi | \mathcal{X}) $ on the set of states $ \mathfrak{S} $.} Hence,
we can study properties of propositions by analyzing properties of
probability measures $ (\phi | \mathcal{X}) $. For this, there is no
need to deal with regions of the phase space, which is exactly what we
want.

For example, we will say that $ \mathcal{X} \leq \mathcal{Y} $ if
measurements for all states $ \phi \in \mathfrak{S} $ show that
$ (\phi | \mathcal{X}) \leq (\phi | \mathcal{Y}) $. The relation
$ \mathcal{X} \leq \mathcal{Y} $ defines the \emph{partial ordering}\index{partial ordering}
 on the propositional system $ \mathcal{L} $.

After the partial ordering $ \leq $ is established on the entire set
$ \mathcal{L} $, it is not difficult to define the \emph{meet} operation
$ \mathcal{X} \wedge \mathcal{Y} $ for all pairs $ \mathcal{X} $,
$ \mathcal{Y} $. For example, if $ \mathcal{X} $ and $\mathcal{Y} $ are
given, we should be able to find a set of all propositions $ \mathcal{Z}'
$ that are ``less than or equal to'' both $ \mathcal{X} $ and
$ \mathcal{Y,} $\footnote{At least one such proposition $ \emptyset
$ always exists.} $ \mathcal{Z}' \leq \mathcal{X} $ and $ \mathcal{Z}' \leq \mathcal{Y} $.
It is reasonable to assume that there is a single maximal proposition
$\mathcal{Z} $ in this set. We shall call it the \emph{meet}\index{meet} of
$\mathcal{X}$ and $\mathcal{Y}$: $\mathcal{Z} = \mathcal{X} \wedge
\mathcal{Y}$.

The \emph{join}\index{join} $ \mathcal{X} \vee \mathcal{Y} $ for all pairs
$ \mathcal{X}$, $ \mathcal{Y} $ is defined in a similar way: it is the
unique smallest proposition that is greater than or equal to both
$ \mathcal{X}$ and $ \mathcal{Y} $. These definitions are formalized as
properties 6--9 in Table \ref{table:2.1}. Properties 10--13 follow
naturally from these definitions.

Further, suppose that for some pair of propositions $ \mathcal{X} $,
$ \mathcal{Y} $ we notice that for all states $ (\phi | \mathcal{X}) =
1 - (\phi | \mathcal{Y}) $. Then we will say that the two propositions
are orthocomplemented: $ \mathcal{X} = \mathcal{Y}^{\perp } $ or,
equivalently, $ \mathcal{Y} = \mathcal{X}^{\perp } $.\index{orthocomplement}

Reasoning in this way, it is not difficult to derive (see Appendix
\ref{ss:meet_join2}) all axioms 1--17 of classical logic from Table
\ref{table:2.1}, except for the axioms of distributivity 18 and 19. The
latter two axioms cannot be justified using our approach with
probability measures $ (\phi | \mathcal{X}) $. For this reason, we
regard distributive laws as less valid and call them simply assertions.

So, in quantum mechanics, we are not allowed to use the distributive
laws of logic. However, in order to obtain a nontrivial theory, it is
necessary to find some kind of substitute for these two laws. On the one
hand, this new postulate must be sufficiently specific, so that it can
be used to develop a nontrivial logical and physical theory. On the
other hand, it must be general enough and include distributive laws as
a special case, because we want to have the classical theory as a
limiting case of the quantum one.\looseness=-1

So, how should we formulate this new quantum axiom?

%s4.3 #&#
\vspace*{-3.5pt}\subsection{Compatibility of propositions}\label{ss:compatibility}
\vspace*{-3.5pt}
To answer this question, let us turn to the important concept of
compatibility. We will say that two propositions $ \mathcal{X} $ and
$ \mathcal{Y} $ are \emph{compatible}\index{compatible propositions} (denoted
$ \mathcal{X}\leftrightarrow \mathcal{Y} $)
\label{lb:compatibility}
if
%
%e6 #&#
\begin{align}
\mathcal{X} = (\mathcal{X} \wedge \mathcal{Y}) \vee \bigl( \mathcal{X} \wedge
\mathcal{Y}^{\perp } \bigr) \label{eq:compat1}
\end{align}
%
and\vspace*{-6pt}
%
%e7 #&#
\begin{align}
\mathcal{Y} = (\mathcal{X} \wedge \mathcal{Y}) \vee \bigl( \mathcal{X}
^{\perp } \wedge \mathcal{Y} \bigr) . \label{eq:compat2}
\end{align}
%
In Subection \ref{ss:compatible} we will see that two experimental
propositions can be measured simultaneously if and only if they are
compatible. Therefore, we should not be surprised by Theorem
\ref{Theorem4.17}, which states that the compatibility ($=$~simultaneous
measurability) of all propositions is a characteristic property of
classical Boolean lattices.

%s4.4 #&#
\subsection{Logic of quantum mechanics}\label{ss:qm-logic}
From the Heisenberg microscope example it should be clear that, unlike
in the classical case, in quantum mechanics not all propositions are
measurable simultaneously ($=$~compatible). Then, as the basic statement
of quantum logic, we postulate that two propositions are definitely
compatible if one follows from the other, and we leave it to mathematics
to decide on the compatibility of other pairs.\footnote{The author
does not know any deeper justification for this postulate. The strongest
argument is that this postulate really works, i.\,e., it leads to the
well-known mathematical structure of quantum mechanics, which has been
tested extensively in experiments.}

%pchapter.6 #&#
\begin{postulate}[orthomodularity]
\label{postulateK13}Propositions about physical systems obey the \emph{ortho}-\emph{modular law}:
If $ \mathcal{B} $ follows from $ \mathcal{A} $, then $ \mathcal{A} $
and $ \mathcal{B} $ are compatible, i.\,e.,\index{orthomodular law}
%
%e8 #&#
\begin{align}
\mathcal{A} \leq \mathcal{B} \quad \Rightarrow\quad  \mathcal{A} \leftrightarrow
\mathcal{B}.
\label{eq:orthomodularity}
\end{align}
 \end{postulate}

\nn
Orthocomplemented lattices\footnote{That is, described by properties
1--17 in Table \ref{table:2.1}.} with the additional orthomodular
Postulate \ref{postulateK13} (property 20 in Table \ref{table:2.1}) are
called \emph{orthomodular lattices}.\index{orthomodular lattices} With the
addition of technical conditions of atomicity and irreducibility, these
lattices become the so-called\index{quantum logic} \emph{quantum}
\emph{logics}. The relationships between different types of lattices and
logics are shown in Figure~\ref{fig:4.12y}.

\begin{SCfigure}[50][b!]
\includegraphics{01f07}
\caption{Relationships between different types of lattices and logics.}
\label{fig:4.12y}
\end{SCfigure}

%s4.5 #&#
\subsection{Quantum logic and Hilbert space}\label{ss:qm-logic-hilbert}
We have seen that in classical mechanics we do not have to use the
exotic lattice theory. Instead, we can apply Theorem
\ref{Theorem4.13} and go over to the physically transparent language of
the phase space. Is there a similar equivalence theorem in the quantum
case? The answer is ``yes.'' It is not difficult to notice close
analogies between the quantum system of propositions described above and
the algebra of projections on closed subspaces in a complex Hilbert\index{Hilbert, D.}
space $ \mathscr{H} $ (see Appendices \ref{sc:hilbert-space} and
\ref{sc:subspaces}). In particular, if operations between projections
(or subspaces) in the Hilbert space are translated into the language of
lattice operations in accordance with Table
\ref{table:2.5},\footnote{We have denoted by $ \mathscr{X} \uplus
\mathscr{Y} $ the \emph{linear span}\index{linear span}
\label{lb:span}
of two subspaces $ \mathscr{X} $ and $ \mathscr{Y} $ (see Appendix
\ref{ss:vector-space}). Here $ \mathscr{X} \cap \mathscr{Y} $ denotes
the intersection of these subspaces and
\label{lb:cap}
$ \mathscr{X}'$ is the orthogonal complement of $ \mathscr{X} $. In the
Hilbert space \emph{atoms}\index{atom} are one-dimensional subspaces. They
are also referred to as \emph{rays}.\index{ray}} then all axioms of quantum logic are
easily verified.

%t3 #&#
\begin{table}[h]
\tabcolsep=0pt
\caption{Translations of symbols between equivalent languages: (i) subspaces in
the Hilbert space~$\mathscr{H}$, (ii) projections on these subspaces and (iii)
propositions in quantum logic $\mathcal{L}$.}
\label{table:2.5}
\begin{tabular*}{.8\textwidth}{@{\extracolsep{4in minus 4in}}lll}
Subspaces in $\mathscr{H}$ & Projections in $\mathscr{H}$ & Propositions in $\mathcal{L}$ \\
\midrule\starttabularbody
$\mathscr{X}\subseteq \mathscr{Y}$ & $P_{\mathscr{X}}P_{\mathscr{Y}} = P_{\mathscr{Y}}P_{\mathscr{X}} = P_{\mathscr{X}}$ & $\mathcal{X} \leq \mathcal{Y}$ \\
$\mathscr{X}\cap \mathscr{Y}$ & $P_{\mathscr{X} \cap \mathscr{Y}} $& $\mathcal{X} \wedge \mathcal{Y}$ \\
$\mathscr{X} \uplus \mathscr{Y} $ & $P_{\mathscr{X} \uplus \mathscr{Y}}$ & $ \mathcal{X} \vee \mathcal{Y}$ \\
$\mathscr{X}'$ & $1 - P_{\mathscr{X}}$ & $\mathcal{X}^{\perp }$ \\
$\mathscr{X}$ and $\mathscr{Y}$ compatible & $[P_{\mathscr{X}}, P_{\mathscr{Y}}] = 0$ & $ \mathcal{X} \leftrightarrow \mathcal{Y}$ \\
$\mathscr{X} \perp \mathscr{Y}$ & $P_{\mathscr{X}}P_{\mathscr{Y}} = P_{\mathscr{Y}}P_{\mathscr{X}} =0$ & $\mathcal{X} \leq \mathcal{Y}^{\perp }$ \\
$\emptyset $ & $ 0$ & $\emptyset $ \\
$\mathscr{H}$ & $1$ & $\mathcal{I}$ \\
$\operatorname{ray}  \mathrm{x}$ & $|\mathrm{x}\rangle \langle \mathrm{x}|$ (1D projection) & $x$ is an atom \\
\end{tabular*}
\end{table}

For example, the violation of distributive laws follows from the fact
that in $ \mathscr{H} $ there are pairs of incompatible subspaces (see
Appendix \ref{sc:subspaces}). The validity of the orthomodular law in
$ \mathscr{H} $ is proved in Theorem \ref{TheoremG.7}.

%s4.6 #&#
\subsection{Piron's theorem}

Thus, we have established that the set of subspaces in a complex Hilbert
space $ \mathscr{H} $ is indeed a representative of some quantum logic.
Can we claim also the opposite, i.\,e., that for each quantum logic one
can construct a representation by subspaces in some Hilbert space? The
(positive) answer to this question is given by the famous \emph{Piron}
\emph{theorem}\index{Piron's theorem}\index{Piron, C.} \cite{Piron2, Piron}, which
forms the basis of the mathematical formalism of quantum mechanics and
is a quantum analog of the classical representation Theorem
\ref{Theorem4.13}.

%tchapter.7 #&#
\begin{theorem}[Piron's theorem]
\label{Theorem.Piron}
Each quantum logic $ \mathcal{L} $ is isomorphic to the lattice of
closed subspaces in a Hilbert space $ \mathscr{H} $. The correspondences
proposition $\leftrightarrow $ subspace are defined by the rules shown in
Table \ref{table:2.5}.
\end{theorem}


The proof of this theorem is beyond the scope of our
book.\footnote{Piron's theorem does not specify the nature of scalars
in the Hilbert space. It leaves the possibility of choosing any
\emph{division ring} \emph{with involutive antiautomorphism} as the
collection of scalars in $ \mathscr{H} $. We can substantially reduce
this unwanted freedom if we recall the important role played by real
numbers in physics (for example, the values of observables always lie
in $ \mathbb{R} $). Therefore, it makes sense to consider only those
rings that include $ \mathbb{R} $ as a subring. In 1877 Frobenius\index{Frobenius, F.\,G.} proved
that there are only three such rings. These are \emph{real numbers}
$ \mathbb{R} $, \emph{complex numbers}\label{lb:complex}\label{lb:real}
$\mathbb{C} $ and \emph{quaternions}\index{quaternions} $ \mathbb{H} $.
Although there is fairly extensive literature on the real and,
especially, quaternionic quantum mechanics
\cite{Stueckelberg,quaternionic,Moretti, Moretti2}, the significance of
these exotic theories for physics remains unclear. Therefore, in our
book we will adhere to the standard quantum mechanics in complex Hilbert
spaces.}

%s4.7 #&#
\subsection{Should we abandon classical logic?}\label{ss:shouldwe}
So, we came to the paradoxical conclusion that classical logic and
classical probability theory are not suitable for describing quantum
microscopic systems. How could this be? After all, classical logic is
the foundation of all mathematics and indeed of the whole scientific
method! All proofs of mathematical theorems use the laws of Boolean
logic, including the distributive laws that were discarded by
us.\footnote{Note, however, attempts \cite{Dejonghe} to develop
the so-called \emph{quantum mathematics}, which is based on the laws of
quantum logic.} Even theorems of quantum mechanics are being proved in
the framework of classical logic. Are we not entering into a
contradiction when we claim that the true logic of experimental
statements is not classical, but quantum \cite{Putnam}?

In everyday life, as in ordinary mathematics, we have the right to use
inaccurate classical logic, because we usually deal with fixed objects
that are not subject to quantum fluctuations. Theorems of Euclidean\index{Euclid}
geometry speak of well-defined circles and triangles, not of statistical
ensembles of figures with randomly distributed parameters. Therefore,
in the proofs of such theorems, it is perfectly acceptable to use the
laws of classical logic. However, when we go to the microworld, where
results of measurements are subject to randomness and observables may
be incompatible with each other, then we have to admit that classical
distributive laws are no longer valid and that quantum logic should take
over.

The theory of orthomodular lattices is well known to mathematicians. In
principle, we could make all constructions and calculations in quantum
theory, based on this formalism. Such an approach would have certain
advantages, since all its components would have a clear physical
meaning: the elements $ \mathcal{X} $ of the lattice are experimental
propositions implementable in the laboratory and the probabilities
$ (\phi | \mathcal{X}) $ can be measured directly in experiments.
However, such a theory would encounter insurmountable difficulties,
mainly because lattices are rather exotic mathematical objects; we lack
experience and intuition to work with them. In addition, this approach
would require us to abandon the familiar distributive laws of logic and
thus would greatly complicate our reasoning.

Historically, the development of quantum theory took another route.
Thanks to Piron's theorem,\footnote{Of course, the fathers of quantum
mechanics did not know about this theorem, which was formulated only in
the 1960s.} the physically transparent but mathematically cumbersome
lattices could be replaced by physically obscure but mathematically
convenient Hilbert spaces, wave functions and Hermitian operators. In
the next section we will briefly summarize this traditional formalism.

%s5 #&#
\section{Physics in Hilbert space}\label{ss:quant-obs}
In the previous section, we established a one-to-one correspondence
between experimental propositions and subspaces of the Hilbert space.\index{Hilbert, D.}
In this section, we will use this fact to construct the mathematical
formalism of quantum mechanics. In particular, we will see that, in
accordance with textbooks, observables are expressed by Hermitian
operators in $ \mathscr{H} $, and pure quantum states are unit length
vectors in the same space.%\vfill

%s5.1 #&#
\subsection{Quantum observables}\label{ss:quant-obs2}
In Subsection \ref{ss:observables}, we saw that, in the language of logic,
an observable $F$ is a mapping $\mathcal{F}$ associating a proposition in $ \mathcal{L} $ ($=$~a subspace
in~$ \mathscr{H} $) with each point
of the spectrum of $F$. The points $ f $ in the spectrum of the observable
$ F $ are called \emph{eigenvalues} of this observable.\index{eigenvalue}
The subspace $ \mathscr{F}_{f} \subseteq \mathscr{H} $ corresponding to
the eigenvalue $ f $ is called the \emph{eigensubspace},\index{eigensubspace}
 and the projection $ P_{f} $ on this subspace is
called the \emph{spectral projection}.\index{spectral projection} Each
vector in the eigensubspace $ \mathscr{F}_{f} $ will be called an
\emph{eigenvector} of the observable.

Let us consider two distinct eigenvalues $f$ and $g$ of one observable
$F$. According to the definition from Subsection \ref{ss:observables}, the
corresponding propositions $ \mathcal{F}_{f} $ and $ \mathcal{F}_{g} $
are disjoint, and their eigensubspaces are orthogonal. The linear span
of the subspaces $ \mathscr{F}_{f} $, where $ f $ runs through the
entire spectrum of the observable $ F $, coincides with the entire
Hilbert space $ \mathscr{H} $. Consequently, spectral projections $ P_f $ of any
observable form a \emph{resolution of} \emph{the identity} (see Appendix
\ref{sc:projections}). Thus, according to formula (\ref{eq:A.62}), we
can associate an Hermitian operator\index{Hermite, C.}
%
%e9 #&#
\begin{align}
F = \sum_{f} f P_{f} \label{eq:op-obs}
\end{align}
%
with each observable $ F $. In the following, we often use the terms
``observable'' and ``Hermitian operator'' as synonyms.

%s5.2 #&#
\subsection{States}\label{ss:states}
As we know from Subsection \ref{sc:1.2.33}, each state $ \phi $ defines a
probability measure $ (\phi | \mathcal{X}) $ on propositions in
$ \mathcal{L} $. In accordance with the quantum isomorphism
``proposition'' $ \leftrightarrow $ ``subspace,'' the state $ \phi $ also
defines a \emph{probability measure}\index{probability measure}
$(\phi | \mathscr{X}) $ on subspaces $ \mathscr{X} $ in the Hilbert
space $ \mathscr{H} $. This probability measure is a function that maps
subspaces into the interval $ [0,1] \subseteq \mathbb{R} $ and has the
following properties:
%
\begin{itemize}
\item  The probability corresponding to the entire space
$ \mathscr{H} $ is equal to 1 in all states,
%e10 #&#
\begin{align}
(\phi |\mathscr{H}) = 1. \label{eq:prob1x}
\end{align}

\item  The probability corresponding to the zero (empty)
subspace is 0 in all states,
%e11 #&#
\begin{align}
(\phi |\emptyset ) = 0. \label{eq:prob2x}
\end{align}

\item  The probability corresponding to the direct sum of
orthogonal subspaces is the sum of probabilities for each
subspace,\footnote{This is equivalent to the third Kolmogorov
probability axiom (\ref{eq:prob3}). By the symbol $ \oplus $ (direct
sum) we denote the linear span ($ \uplus $) of two subspaces in the case
where these subspaces are orthogonal.}
%e12 #&#
\begin{align}
(\phi |\mathscr{X} \oplus \mathscr{Y}) = (\phi |\mathscr{X}) + ( \phi |
\mathscr{Y}), \quad \mbox{if } \mathscr{X} \perp \mathscr{Y.} \label{eq:prob3x}
\end{align}
\end{itemize}
%
The following important theorem \cite{Gleason} provides a
classification of all such probability measures ($=$~all possible states
of a quantum system).

%tchapter.8 #&#
\begin{theorem}[Gleason's theorem]
\label{Theorem.Gleason}\index{Gleason's theorem}\index{state}\index{Gleason, A.}
If $ (\phi | \mathscr{X}) $ is a probability measure on subspaces in the
Hilbert space $ \mathscr{H} $ with properties
(\ref{eq:prob1x})--(\ref{eq:prob3x}), then there is a
nonnegative\footnote{A Hermitian operator is called \emph{nonnegative} if
all its eigenvalues are greater than or equal to zero.} Hermitian
operator $ \hat{\rho } $ in $ \mathscr{H} $ such that
%
%e13 #&#
\begin{align}
\operatorname{Tr} (\hat{\rho }) = 1 \label{eq:(4.28)}
\end{align}
%
and for any subspace $ \mathscr{X} $ and its projection $ P_{
\mathscr{X}} $, the value of the probability measure is\footnote{$ \operatorname{Tr}
$ means the \emph{trace} of the matrix of the operator $ \hat{\rho } $, i.\,e.,
the sum of its diagonal elements; see Appendix \ref{ss:functions}.}
%
%e14 #&#
\begin{align}
(\phi |\mathscr{X}) = \operatorname{Tr} (P_{\mathscr{X}} \hat{\rho }). \label{eq:(4.29)}
\end{align}
\end{theorem}



The operator $ \hat{\rho } $ is usually called the \emph{density}
\emph{operator}\index{density operator} or \emph{density}
\emph{matrix}.\index{density matrix}

Proving Gleason's theorem is not easy, and we refer the curious reader
to the original papers \cite{Gleason,
Richman_Bridges}. Here we will only touch upon the physical
interpretation of this result. First, in accordance with the spectral
Theorem \ref{spectral-th}, one can find an orthonormal basis
$ |e_{i} \rangle $ in $ \mathscr{H} $ where the density operator
$ \hat{\rho } $ reduces to the diagonal form
%
%e15 #&#
\begin{align}
\hat{\rho } = \sum_{i} \rho_{i} |
e_{i} \rangle \langle e_{i} |, \label{eq:rho}
\end{align}
%
where the eigenvalues $ \rho_{i} $ have the properties
%
%e16 #&#
%e17 #&#
\begin{align}
0 &\leq \rho_{i} \leq 1, \label{eq:(4.30)}
\\
\sum_{i} \rho_{i} & = 1. \label{eq:(4.31)}
\end{align}
%
Among all states with properties (\ref{eq:(4.30)})--(\ref{eq:(4.31)}),
one can select those in which only one coefficient $ \rho_{i} $ is
nonzero
and $ \rho_{j} = 0 $ for all other indices $ j \neq i $. In this case
the density operator reduces to the projection onto a one-dimensional
subspace
%
%e18 #&#
\begin{align}
\hat{\rho }= |e_{i} \rangle \langle e_{i}|. \label{eq:1.17a}
\end{align}
%
Such states will be referred to as \emph{pure quantum}\index{pure quantum state}
states. For pure states, the formula
(\ref{eq:(4.29)}) for calculating probabilities is simplified. Formally,
using Lemma \ref{LemmaA.4} and Theorem~\ref{TheoremA.13}, we find that
the probability for the proposition $ \mathcal{X} $ to be true in the
state (\ref{eq:1.17a}) is equal to the square of the modulus of the
projection of $ | e_{i} \rangle $ onto the subspace $ \mathscr{X} $,
i.\,e.,
%
%e19 #&#
\begin{align}
(\phi |\mathcal{X}) = \operatorname{Tr} \bigl(P_{\mathscr{X}} |e_{i} \rangle
\langle e _{i}|\bigr) = \operatorname{Tr} \bigl(\langle e_{i}|
P_{\mathscr{X}} |e_{i} \rangle \bigr) = \langle e_{i}|
P_{\mathscr{X}} P_{\mathscr{X}}|e_{i} \rangle = \Vert
P_{
\mathscr{X}}|e_{i}\rangle \Vert^{2}.
\label{eq:phix}
\end{align}
%
Therefore, it is customary to describe a pure state by a vector
$ | e_{i} \rangle $ of unit length chosen arbitrarily from the
corresponding one-dimensional subspace ($=$~ray).\footnote{Obviously,
the vector $ | e_{i} \rangle $ is defined only up to a \emph{phase}
\emph{factor} $ e^{i \alpha } $, which has a unit modulus ($ | e^{i \alpha
} | = 1 $, where $ \alpha \in \mathbb{R} $). Indeed, being substituted
in (\ref{eq:phix}), the vector $ e^{i \alpha } | e_{i} \rangle $ leads
to the same probability value,
%
\begin{align*}
\big\Vert P_{\mathscr{X}}\bigl(e^{i \alpha }|e_{i} \rangle \bigr)
\big\Vert^{2} = |e^{i
\alpha }|^{2} \Vert P_{\mathscr{X}}|e_{i}
\rangle \Vert^{2} = \Vert P_{\mathscr{X}}|e_{i} \rangle
\Vert^{2},
\end{align*}
%
so that both vectors $ | e_{i} \rangle $ and $ e^{i \alpha } | e_{i}
\rangle $ are legitimate representatives of the state $\phi $.}

In Subsection \ref{ss:quant-obs2} we introduced the notion of an
eigenvector of the observable $F$. Pure states corresponding to such
eigenvectors will be called \emph{eigenstates} of the observable $ F
$.\index{eigenvector}\index{eigenstate} Obviously, observables have
definite values ($=$~eigenvalues) in their eigenstates. This means that
the eigenstates are precisely those states ($=$~ensembles) whose existence
was guaranteed by Statement \ref{postulateJ1}.

Importantly, there are no quantum probability measures ($=$~states) that
give definite answers to all experimental questions. Thus, by assuming
the orthomodularity of the propositional lattice (Postulate
\ref{postulateK13}), we automatically explained the probabilistic nature
of quantum states (Statement \ref{postulateJ2}).

\emph{Mixed} quantum states\index{mixed state} are expressed as mixtures (\ref{eq:rho}) of
pure states. The coefficients $ \rho_{i} $ in this
formula reflect the probabilities of the pure states in the mixture.
Thus, in quantum mechanics there are two types of uncertainties. The
first type is present in mixed states. This is the same uncertainty
familiar to us from classical (statistical) physics. It appears in
situations where the experimenter does not have complete control over
the preparation of the system, for example, when he throws a die. The
second type of uncertainty is present even in pure quantum states
(\ref{eq:1.17a}). It has no analog in classical physics, it cannot be
gotten rid of by improved control of the initial conditions. This
uncertainty reflects the unavoidable presence of chance in microscopic
phenomena.

We will not discuss mixed quantum states in this book. Therefore, we
will only deal with uncertainties of the second fundamental type. Hence,
speaking of a quantum state $ \phi $, we always have in mind a certain
state vector $ | \phi \rangle $, determined up to a phase factor
$ e^{i \alpha } $. In the following, we will use the terms ``quantum
state'' and ``state vector'' as synonyms.

%s5.3 #&#
\subsection{Complete sets of commuting observables}\label{ss:compatible}
In Subsection \ref{ss:compatibility} we defined the idea of compatible
propositions, and in Lemma \ref{LemmaA.17} we showed that the
compatibility of propositions is equivalent to the commutativity of the
corresponding projections. For physics, these properties are important
because for a pair of compatible propositions ($=$~projections, $=$
subspaces), there are states in which both these propositions have
certain values, i.\,e., they are simultaneously measurable without any
statistical randomness. Similar claims can be made about two compatible
($=$~commuting) Hermitian operators ($=$~observables). In accordance with
Theorem \ref{TheoremA.20}, a pair of such operators has a common basis
of eigenvectors ($=$~eigenstates). In these eigenstates both observables
have definite (eigen)values.

We assume that for every physical system one can always find at least
one minimal and complete set of mutually commuting observables
$ K, L, M, \ldots $\,.\footnote{A set $ K, L, M, \ldots $ is called
\emph{minimal} if not one observable from this set can be expressed as a
function of other observables in the set. The set is \emph{complete} if no
new observable can be added to it without destroying the minimality
property. An example of a complete set of mutually commuting observables
for one massive particle is $ \{M, P_{x}, P_{y}, P_{z}, S_{z} \} $,
where $M$, $\boldsymbol{P}$ and $ \boldsymbol{S} $ are the operators of
mass, momentum and spin, respectively (see Section \ref{sc:massive}).}
Then we can construct an orthonormal basis $ |e_{i} \rangle $ of common
eigenvectors of these operators so that each such eigenvector is
uniquely marked by eigenvalues $ k_{i}, l_{i}, m_{i}, \ldots $ of the
operators $ K, L, M, \ldots $\,. That is, if $ | e_{i} \rangle $ and
$ | e_{j} \rangle $ are two different basis vectors, then their sets of
eigenvalues $ \{k_{i}, l_{i}, m_{i}, \ldots \} $ and $ \{k_{j}, l_{j},
m_{j}, \ldots \} $ are not the same.

%s5.4 #&#
\subsection{Wave functions}\label{ss:wave_function}
Each state vector $ | \phi \rangle $ can be represented as a linear
combination of the basis vectors constructed in the previous subsection,
%
%e20 #&#
\begin{align}
| \phi \rangle = \sum_{i} \phi_{i}
|e_{i} \rangle , \label{eq:(4.33)}
\end{align}
%
where in the bra-ket notation (see Appendix \ref{ss:bra-ket}) the
coefficients are expressed as
%
%e21 #&#
\begin{align}
\phi_{i} = \langle e_{i} | \phi \rangle . \label{eq:(4.33a)}
\end{align}

The set of complex numbers $ \phi_{i} $ can be considered as a function
$ \phi (k, l, m, \ldots ) $ on the common spectrum of the observables
$ K, L, M, \ldots $\,. This is called the \emph{wave function}\index{wave function}
of the state $ | \phi \rangle $ in the
\emph{representation} defined by the observables $ K, L, M, \ldots $\,. We will
discuss examples of one-particle wave functions in Sections
\ref{sc:representations}--\ref{ss:position-representation}.

%s5.5 #&#
\subsection{Expectation values}\label{ss:expectation}
Formula (\ref{eq:op-obs}) defines a spectral resolution of the
observable $ F $, where index $ f $ runs through all eigenvalues of
$ F $. The spectral projections $ P_{f} $ can be expanded through basis
eigenvectors, so we have
%
%e22 #&#
\begin{align}
P_{f} \equiv \sum_{i=1}^{m} \bigl\llvert
e_{i}^{f} \big\rangle \big\langle e_{i}^{f}
\bigr\rrvert . \label{eq:1.39a}
\end{align}
%
Here $  \llvert   e_{i}^{f}  \rangle  $ are orthogonal eigenvectors of
the operator $ F $ that are inside the eigensubspace $ \mathscr{F}
_{f} $, and $ m $ is the dimension of this subspace.\footnote{If
$ m> 1 $, then the eigenvalue $ f $ is called \emph{degenerate}.\index{degenerate eigenvalue}}
Then from (\ref{eq:phix}) one can find
the probability for measuring $ f $ in each pure state $\phi $,
%
%e23 #&#
\begin{align}
( \phi | P_{f} ) = \Biggl\llVert \sum_{i=1}^{m}
\big\llvert e_{i}^{f} \big\rangle \bigl\langle e_{i}^{f}
\big| \phi \bigr\rangle \Biggr\rrVert ^{2}= \sum_{i=1}^{m}
\bigl\llvert \bigl\langle e_{i}^{f} \big| \phi \big\rangle
\bigr\rrvert ^{2}. \label{eq:(4.34)}
\end{align}

Sometimes we need to know the weighted average, or the \emph{expectation}
\emph{value},\label{lb:expectation}\index{expectation value} $ \langle F \rangle $ of the observable
$ F $ in the state $ | \phi \rangle , $
%
\begin{align*}
\langle F \rangle \equiv \sum_{f} ( \phi |
P_{f} ) f.
\end{align*}
%
Substituting here equation (\ref{eq:(4.34)}), we obtain
%
\begin{align*}
\langle F \rangle = \sum_{j=1}^{n} |\langle
e_{j} | \phi \rangle |^{2} f_{j} \equiv \sum
_{j=1}^{n} |\phi_{j}|^{2}
f_{j},
\end{align*}
%
where the summation is carried out over the entire basis of eigenvectors
$ | e_{j} \rangle $. From expansions (\ref{eq:(4.33)}),
(\ref{eq:op-obs}) and (\ref{eq:1.39a}) it follows that the combination
$ \langle \phi | F | \phi \rangle $ is a more compact notation for the
expectation value $ \langle F \rangle $. Indeed
%
%e24 #&#
\begin{align}
\langle \phi |F| \phi \rangle &= \biggl( \sum_{i}
\phi_{i}^{*} \langle e_{i} | \biggr)
\biggl( \sum_{j} | e_{j} \rangle
f_{j} \langle e_{j} | \biggr) \biggl( \sum
_{k} \phi_{k} |e_{k} \rangle
\biggr)
\nonumber
\\
&= \sum_{ijk} \phi_{i}^{*}
f_{j} \phi_{k} \langle e_{i} | e_{j}
\rangle \langle e_{j} | e_{k} \rangle = \sum
_{ijk} \phi_{i}^{*} f_{j}
\phi_{k}\delta_{ij} \delta_{jk}
\nonumber
\\
&= \sum_{j} \llvert \phi_{j} \rrvert
^{2} f_{j} = \langle F \rangle . \label{eq:fff}
\end{align}

%s5.6 #&#
\subsection{Basic rules of classical and quantum mechanics}\label{ss:rules}
The results obtained in this chapter can be summarized as follows. If
the physical system is prepared in a pure state $ \phi $ and we want to
calculate the probability $ \omega $ to measure the observable $ F $
within the interval $ E \subseteq \mathbb{R} $, then we need to perform
the following steps.

\bigskip
\noindent
{In classical mechanics:}
%
%
%5 items enumerated
%the first item: 1
%the last item: 5
% the (last) widest item: 5
\begin{enumerate}
\item[(1)] Determine the phase space $ \mathcal{S} $ of the physical
system.
\item[(2)] Find the real function $ f: \mathcal{S} \to \mathbb{R} $
corresponding to our observable $ F $.
\item[(3)] Find the subset $ U \subseteq \mathcal{S} $ corresponding to
the spectral interval $ E $, where $ U $ is the collection of all points
$ s \in \mathcal{S} $ such that $ f(s) \in E $ (see Figure~\ref{fig:4.12x}).
\item[(4)] Find the point $ s_{\phi } \in \mathcal{S} $ representing the
pure classical state $ \phi $.
\item[(5)] The probability $ \omega $ is 1 if $ s_{\phi } \in U $ and
$ \omega = 0 $ otherwise.
\end{enumerate}
%
%
%\subsubsection*
%\bigskip
%\noindent
{In quantum mechanics:}
%
%
%6 items enumerated
%the first item: 1
%the last item: 6
% the (last) widest item: 6
\begin{enumerate}
\item[(1)] Determine the Hilbert space $ \mathscr{H} $ of the physical
system.
\item[(2)] Find the Hermitian operator $ F $ corresponding to our
observable in $ \mathscr{H} $.
\item[(3)] Find the eigenvalues and eigenvectors of $ F $.
\item[(4)] Find the spectral projection $ P_{E} $ corresponding to the
spectral interval $ E $.
\item[(5)] Find the unit vector $ | \phi \rangle $ (defined up to a phase
factor) representing the state $ \phi $ in the Hilbert space
$ \mathscr{H} $.
\item[(6)] Substitute all these ingredients in the probaility formula $ \omega =
\langle \phi | P_{E} | \phi \rangle $.
\end{enumerate}
%
At the moment, the classical and quantum recipes seem completely
unrelated to each other. Nevertheless, we are sure that such a
connection must exist, because we know that both these theories are
variants of the probability formalism on orthomodular lattices. In
Section \ref{sc:classical}, we will see that in the macroscopic world
with massive objects and poor resolution of measuring devices, the
classical recipe appears as a reasonable approximation to the quantum
one.

%s6 #&#
\section{Interpretations of quantum mechanics}\label{sc:complete}
So far in this chapter, we were occupied with the mathematical formalism
of quantum mechanics. Many details of this formalism (wave functions,
superpositions of states, Hermitian operators, nonstandard logic, etc.)
seem very abstract and detached from reality. This situation has
generated a lot of debates about the physical meaning and interpretation
of quantum laws. In this section, we will suggest our point of view on
these controversies.

%s6.1 #&#
\subsection{Quantum nonpredictability}\label{ss:unpredictability}
Experiments with quantum microsystems revealed one simple but
nonetheless mysterious fact: if we prepare $ N $ absolutely identical
physical systems under the same conditions and measure the value of the
same physical quantity, we can obtain $ N $ different results.

Let us illustrate this statement with two examples. From experience we
know that each photon passing through the aperture of the pinhole camera
will hit some point on the photographic plate. However, the next photon,
most likely, will hit another point. And, in general, the locations of
hits are randomly distributed over the surface. Quantum mechanics does
not even try to predict the fate of each individual photon. It only
knows how to calculate the probability density for the points of impact,
but the behavior of each individual photon remains completely random and
unpredictable.\looseness=1

Another example of this -- obviously random -- behavior is the decay of
radioactive nuclei. The \tsup{232}Th  nucleus has a half-life of 14
billion years. This means that in any sample containing thorium,
approximately half of all \tsup{232}Th  nuclei will decay during this
period. In principle, quantum physicists can calculate the decay
probability of a nucleus by solving the corresponding Schr\"{o}dinger
equation.\footnote{Although our current knowledge of the nature of
nuclear forces is completely inadequate
to perform this kind of calculation for thorium.} However, they cannot
even approximately guess when the given nucleus decays. It can happen
today or in 100 billion years.\index{Schr\"odinger equation}

It would be wrong to think that the probabilistic nature of microscopic
systems has little effect on our macroscopic world. Very often the
effects of random quantum processes can be amplified and lead to
macroscopic phenomena, which are equally random. One well-known example
of such amplification is the thought experiment with ``Schr\"{o}dinger's
cat'' \cite{cat}.\index{Schr\"{o}dinger's cat}\index{Schr\"{o}dinger, E.}\vfill

%\vspace*{-3pt}
%s6.2 #&#
\subsection{Collapse of wave function}\label{ss:unpredictability2}
%\vspace*{-3pt}
In the \emph{orthodox interpretation}\index{orthodox interpretation} of
quantum mechanics, the behavior described above is called the ``collapse
of the quantum probability distribution''\index{wave function collapse}  and is often surrounded with
a certain aura of mystery.\footnote{To emphasize the analogy with the
classical case, here we specifically talk about the collapse of the
``probability distribution,'' and not about the collapse of the ``wave
function,'' as in other works. It is precisely the probability
distribution that is subject to experimental observation, and the wave
function is a purely theoretical concept.} In this interpretation, the
most controversial point of quantum mechanics is its different attitude
to the physical system and the measuring device. The system is regarded
as a quantum object that can exist in strange
superpositions,\footnote{The electron in the previous example is
allegedly in a superposition of states smeared over the surface of the
photographic plate, and the thorium nucleus is in a superposition of the
decayed and undecayed states.} while the measuring device is a classical
object whose state (readout) is fully unambiguous. It is believed that
at the time of measurement, an uncontrolled interaction between the
system and the measuring device occurs, so that the superposition
collapses into one of its components, which is recorded by the
instrument. Inside the theory, this difference of attitudes is expressed
in the fact that the system is described by a wave function, but the
measuring device is described by an Hermitian operator. This leads to a number of
unpleasant questions.

Indeed, the measuring device consists of the same atoms as the physical
system and the rest of the universe. Therefore, it is rather strange
when such devices are put into a separate category of objects. But if
we decided to combine the device and the system into one wave function,
when would it collapse? Maybe this collapse would require the
participation of a conscious observer? Does this mean that by making
observations, we control the course of physical processes?

Sometimes a mystery is seen in the fact that the quantum-mechanical
probability distribution ($=$~wave function) has two mutually exclusive
laws of evolution. While we are not looking at the system, this
distribution develops smoothly and predictably (in accordance with the
Schr\"{o}dinger equation), and at the time of measurement it experiences
a random unpredictable collapse.

%s6.3 #&#
\subsection{Collapse of classical probability distribution}\label{ss:unpredictability3}
By itself, the collapse of the probability distribution is not something
strange. A similar collapse occurs in the classical world as well. For example,
when shooting from a rifle at a target, it is almost impossible to
predict the hit location of each specific bullet. Therefore, the state
of the bullet before it hits the target ($=$~before the measurement) is
conveniently described by the probability distribution. At the moment
of the hit, the bullet punches the target at a specific place, and the
probability is immediately replaced by certainty. The measurement leads
to the ``collapse of the probability distribution,'' exactly as in the
formalism of quantum mechanics.

The probability density for the bullet changes smoothly (spreads out)
from the moment of the shot and up to the time of impact. The
unpredictable collapse of this probability distribution occurs
instantaneously in the entire space. These behaviors are completely
analogous to the two (allegedly contradictory) variants of quantum
evolution, but the classical collapse does not raise any controversy among
theorists and philosophers.

We rightly believe that the collapse of classical probability is the
natural behavior of any probability distribution. Then, why does the
collapse of quantum probability still trouble theoreticians?

The fact is that in the case of the bullet and the target, we are sure
that the bullet was \emph{somewhere} at each time instant, even when we did
not see it. In all these moments the bullet had a definite position,
momentum, rotation speed about its axis (spin) and other properties. Our
description of the bullet had some element of randomness only because
of our laziness, unwillingness or inability to completely control the
act of shooting. By describing the state of the bullet by a probability
distribution, we simply admitted the level of our ignorance. When we
looked at the pierced target and thus ``collapsed'' the probability
distribution, we had absolutely no influence on the state of the bullet,
but simply improved (updated) our knowledge about it. The probability
distribution and its collapse are things that occur exclusively in our
heads and do not have actual physical existence.

%s6.4 #&#
\subsection{Hidden variables}\label{ss:hidden}
Einstein believed that the same logic should be applied to measurements
in the microworld. He wrote:\index{Einstein, A.}
%
\begin{quote}
I think that a particle must have a separate reality independent of the
measurements. That is an electron has spin, location and so forth even
when it is not being measured. I like to think that the moon is there
even if I am not looking at it.
\end{quote}
%
If we follow this logic blindly, we must admit that even at the
microscopic level, nature must be regular and deterministic. Then the
observed randomness of quantum processes should be explained by some yet
unknown ``hidden'' variables that cannot be observed and controlled
\emph{yet}. If we exaggerate somewhat, the theory of hidden variables reduces
to the idea that each electron has a navigation system that directs it
to the designated point on the photographic plate. Each nucleus has an
alarm clock inside it, and the nucleus decays at the call of this alarm
clock. The behavior of quantum systems only seems random to us, since
we have not yet penetrated the designs of these ``navigation systems''
and ``alarm clocks''.

According to the theory of ``hidden variables,'' the randomness in the
microworld has no special quantum-mechanical nature. This is the same
classic pseudo-ran\-domness that we see when shooting at a target or
throwing dice. Then we have to admit that modern quantum mechanics is
not the last word. Future theory will teach us how to fully describe the
properties of individual systems and to predict events without reference
to the quantum chance. Of course, such faith cannot be refuted, but so
far no one has succeeded in constructing a convincing theory of hidden
variables predicting (at least approximately, but beyond the limits of
quantum probabilities) the results of microscopic measurements.

%s6.5 #&#
\subsection{Quantum-logical interpretation}\label{ss:q-logic}
The most famous thought experiment in quantum mechanics is the two-hole
interference, which demonstrates the limits of classical probability
theory. Recall that in this experiment (see Section \ref{sc:thought}
and Subsection \ref{sc:scatter2}) we did not have the right to add the
probabilities for passing through alternative holes. Instead, quantum
mechanics recommended adding the so-called probability amplitudes and
then squaring the resulting sum \cite{Feynman-lecturesIII}.

This observation leads to the suspicion that the usual postulates of
probability (and logic) do not operate in microsystems. Thus, we
naturally approach the idea of quantum logic as the basis of quantum
mechanics. It turns out that both fundamental features of quantum
measurements -- the randomness of outcomes and the addition of
probability amplitudes for alternative events -- find a simple and
concise explanation in quantum logic (see Section \ref{sc:1.3}). Both
these laws of quantum mechanics follow directly from the orthomodular
logic of experimental propositions. As we know from Piron's theorem,
such logic is realized by a system of projections in the Hilbert space,
and by Gleason's theorem any state ($=$~probability measure) on such a
system must be stochastic, random.

As we saw in Section \ref{sc:1.3} (see Figure~\ref{fig:4.12y}), the
Boolean deterministic logic of classical mechanics is only a particular
case of the orthomodular quantum logic with its probabilities. Thus,
even in formal reasoning, it is the \emph{particular} classical theory that
needs a special explanation and interpretation, and not the
\emph{general} quantum mechanics.
%
\begin{quote}
$\ldots$ classical mechanics is loaded with metaphysical hypotheses which
clearly exceed our everyday experience. Since quantum mechanics is based
on strongly relaxed hypotheses of this kind, classical mechanics is less
intuitive and less plausible than quantum mechanics. Hence classical
mechanics, its language and its logic cannot be the basis of an adequate
interpretation of quantum mechanics -- P. Mittelstaedt
\cite{Mittelstaedt}.\index{Mittelstaedt, P.}
\end{quote}

%s6.6 #&#
\subsection{Quantum randomness and limits of knowledge}\label{ss:collapse2}
So, we came to the conclusion that quantum probability, its collapse and
the existence of superpositions of states are inevitable consequences
of the special orthomodular nature of the logic of experimental
propositions. The laws of probability, built on this logic, differ from
the classical laws of probability that are familiar to us. In
particular, any state ($=$~a probability measure on logical propositions)
must be stochastic, i.\,e., it is impossible to get rid of the element of
chance in measurements. This also means that there is no mystery in the
collapse of the wave function, and there is no need to introduce an
artificial boundary between the physical system and the measuring
apparatus.

The imaginary paradox of the quantum formalism is connected, on the one
hand, with the weirdness of quantum logic, and on the other hand with
unrealistic expectations about the power of science. Theoretical
physicists experience an internal protest when faced with real
physically measurable effects,\footnote{Such as random hits of
electrons on the screen or decay of nuclei.} which they are powerless
to control and/or predict. These are facts without explanations, effects
without causes. It seems that microparticles are subject to some
annoying mysterious random force. But in our view, instead of grieving,
physicists should have celebrated their success.

To us, the idea of the fundamental, irreducible and fundamentally
inexplicable nature of quantum probabilities seems very attractive,
because it may signal the fulfillment of the centuries-old dream of
scientists searching for deep laws of nature. Perhaps, in such an
elegant way, nature has evaded the need to answer our endless questions
``why?'' Indeed, if at the fundamental level nature were deterministic,
then we would face a terrifying prospect of unraveling the endless
sequences of cause--effect relationships. Each phenomenon would have its
own cause, which, in turn, would have a deeper reason, and so on,
\emph{ad infinitum}. Quantum mechanics breaks this chain and at some point
gives us the full right to answer: ``I don't know. It's just an
accident.'' And if some phenomenon is truly random, then there is no
need to seek an explanation for it. The chain of questions ``why?''
breaks. The quest for understanding ends in a logical, natural and
satisfying way.

So, perhaps, the apparent ``incompleteness'' of quantum theory is not
a problem to be solved, but an accurate reflection of the fundamental
essence of nature, in particular, its inherent unpredictability? In this
connection, the following quote from Einstein seems suitable:
%
\begin{quote}
I now imagine a quantum theoretician who may even admit that the
quantum-theoretical description refers to ensembles of systems and not
to individual systems, but who, nevertheless, clings to the idea that
the type of description of the statistical quantum theory will, in its
essential features, be retained in the future. He may argue as follows:
True, I admit that the quantum-theoretical description is an incomplete
description of the individual system. I even admit that a complete
theoretical description is, in principle, thinkable. But I consider it
proven that the search for such a complete description would be aimless.
For the lawfulness of nature is thus constructed that the laws can be
completely and suitably formulated within the framework of our
incomplete description. To this I can only reply as follows: Your point
of view -- taken as theoretical
possibility~--~is incontestable~--~A.~Einstein \cite{incontestable}.\index{Einstein, A.}
\end{quote}
%
The most important philosophical lesson of quantum mechanics is the call
to abandon speculations about unobservable things and their use in the
foundations of the theory. Quantum mechanics does not know whether the
moon is there or not. Quantum mechanics says that the moon will be there
when we look.
\end{document}
