%BeginFileInfo
%%Publisher=DGBOOK
%%Project=STEFAN1
%%Manuscript=STEFAN107
%%MS position=
%%Stage=203
%%TID=juraten
%%Pages=0
%%Format=2016
%%Distribution=vtex
%%Destination=PDF
%%PDF type=print
%%PDF.Maker=pdfwf_luatex
%%Spelled=Dictionary: American, Computer: 1GSRED478, 2018.04.09 14:58
%%History1=2018.04.09 14:38
%EndFileInfo
\documentclass[]{stefan1}
\usepackage{book-aux-toc}
\usepackage{mathrsfs}
\usepackage{bbm}
\makeindex

\startlocaldefs
\newcommand{\lleft}{\left}
\newcommand{\rrvert}{\vert}
\newcommand{\rright}{\right}
\newcommand{\llvert}{\vert}
%author defs here:
\allowdisplaybreaks
\newtheorem{theorem}{Theorem}[chapter]

\endlocaldefs

%\HPROOF\SETGRID
\PROOF
%\CRC

\pubyear{2019}
\firstpage{161}
\lastpage{177}
%\openaccess

\begin{document}

\thisischapter{7}
\chapter{Scattering}\label{sc:scattering}
\begin{epigraph}
{Physics is becoming so unbelievably complex that it is taking
longer and longer to train a physicist. It~is taking so long, in fact,
to train a physicist to the place where he understands the nature of
physical problems that he is already too old to solve them. }
\author{Eugene P. Wigner}
\end{epigraph}
%
Finding solutions of the time-dependent Schr\"{o}dinger equation
(\ref{eq:time-dep}) is incredibly difficult, even for the simplest
models. However, nature had mercy on us and created a very important
class of experiments, where a description of dynamics by
equation~(\ref{eq:time-dep}) is completely unnecessary, because it is
too detailed. Here we are talking about scattering experiments, which
are the topic of this chapter.

A typical scattering experiment is designed in such a way that free
particles (or their bound states, such as atoms or nuclei) are prepared
at a great distance from each other and directed into collision. Then
experimentalists study the properties of free particles or stable bound
states leaving the collision region. In these experiments, as a rule,
it is impossible to see evolution during the interaction process:
particle reactions occur almost instantaneously, and we can observe only
reactants and products moving freely before and after the collision. In
such situations, the theory is not required to describe the true
dynamics of the particles during the short interval of interaction. It
is sufficient to understand only the mapping of free states before the
collision to free states after the collision. This mapping is described
by the so-called $ S $-operator, which we are going to discuss in the
next section.

In Section \ref{sc:scatt-equiv} we will consider the situation of
scattering equivalence when two different Hamiltonians have exactly the
same scattering properties.

%s1 #&#
\section{Scattering operators}\label{ss:scattering}

%s1.1 #&#
\subsection{Physical meaning of $S$-operator}\label{sc:scatter}
Let us consider a scattering experiment in which the free states of the
reactants were prepared in the distant past, so $ t = - \infty $. The
collision itself occurred within a short interval $  [ \eta ',
\eta  ]  $ near time zero.\footnote{The short interaction time
interval (and the applicability of the scattering theory) is guaranteed
if three conditions are satisfied. First, the interaction between
particles is short-range or, in a more general setting,
cluster-separable. Second, the states of particles are describable by
localized wave packets, for example, as in Section \ref{ss:limit}.
Third, the velocities (or momenta) of the particles are high enough.}
Free states of the collision products are registered in the distant
future, $ t = \infty , $ so that the inequalities $ - \infty \ll
\eta '<0 <\eta \ll \infty $ are satisfied. For simplicity, we assume
that the two colliding particles do not form bound states either before
or after the collision. Therefore, at asymptotically distant times, the
exact time evolution of the system is well approximated by the
noninteracting operators $ U_{0}  ( \eta '\gets - \infty  )  $
and $ U_{0} (\infty \gets \eta ) $.\footnote{Here we denote by
$ U_{0} (t \gets t_{0}) \equiv \exp   ( - \frac{i}{\hbar } H_{0} (t-t
_{0})  )  $ the time evolution operator associated with the
noninteracting Hamiltonian $ H_{0} $. The interaction evolution operator
will be denoted by $ U (t \gets t_{0}) \equiv \exp   ( - \frac{i}{
\hbar } H (t-t_{0})  )  $. In the Schr\"{o}dinger representation,
this operator acts on state vectors, as in (\ref{eq:8.60}).} Then the
full evolution operator from the remote past to the distant future is
(here we use properties (\ref{eq:property-A}) and (\ref{eq:property-B}))
%
%e1 #&#
\begin{align}
U(\infty \gets -\infty ) &\approx U_{0}(\infty \gets \eta )
U \bigl( \eta \gets \eta ' \bigr) U_{0} \bigl( \eta
' \gets -\infty \bigr)
\nonumber
\\
&= U_{0}(\infty \gets \eta ) U_{0}(\eta \gets 0) \bigl[
U_{0}(0 \gets \eta ) U \bigl( \eta \gets \eta ' \bigr)
U_{0} \bigl( \eta ' \gets 0 \bigr) \bigr]
\nonumber
\\
&\quad  \times U_{0} \bigl( 0 \gets \eta ' \bigr) U_{0}
\bigl( \eta ' \gets -\infty \bigr)
\nonumber
\\
& = U_{0}(\infty \gets 0)
S_{\eta , \eta '} U_{0}(0 \gets - \infty ), \label{eq:8.63}
\end{align}
where we denote
%
%e2 #&#
\begin{align}
S_{\eta , \eta '} \equiv U_{0}(0 \gets \eta ) U \bigl( \eta \gets \eta
' \bigr) U_{0} \bigl( \eta ' \gets 0 \bigr)
. \label{eq:8.64}
\end{align}
Equation (\ref{eq:8.63}) means that it is possible to formulate a
simplified description for the time evolution in collision processes.
In this description, the evolution is always free, except for a sudden
change of state at the time $ t = 0 $. This change is described by the
unitary operator $ S_{\eta , \eta '} $. Approximation (\ref{eq:8.63})
becomes more accurate if we increase the time interval $  [ \eta ',
\eta  ]  $, during which the exact time evolution is taken into
account, i.\,e., in the limits $ \eta '\to - \infty $ and $\eta \to
\infty $.\footnote{Of course, we assume that the right-hand side of
(\ref{eq:8.64}) converges in these limits. The question of convergence
will be discussed briefly in Section \ref{ss:adiaba}.} Therefore, the
exact formula for the time evolution from $ -\infty $ to $ \infty $ has
the form
%
%e3 #&#
\begin{align}
U(\infty \gets -\infty ) = U_{0}(\infty \gets 0) S U_{0}(0
\gets - \infty ), \label{eq:8.65}
\end{align}
where the $S$-operator \index{S-operator} (or \emph{scattering} \emph{operator})
is defined by
\label{lb:s-oper}
%
%e4 #&#
\begin{align}
S &= \lim_{\eta ' \to -\infty , \eta \to \infty } S_{\eta , \eta '} = \lim_{\eta ' \to -\infty , \eta \to \infty }
U_{0}(0 \gets \eta ) U \bigl( \eta \gets \eta ' \bigr)
U_{0} \bigl( \eta ' \gets 0 \bigr)
\nonumber
\\
&= \lim_{\eta ' \to -\infty , \eta \to \infty } e^{\frac{i}{\hbar }H
_{0}
\eta } e^{-\frac{i}{\hbar }H  ( \eta - \eta ' ) } e^{-\frac{i}{
\hbar }H_{0} \eta '}
\label{eq:8.73a}
\\
&= \lim_{\eta \to \infty } S(\eta ),
\nonumber
\end{align}
where
%
%e5 #&#
\begin{align}
S(\eta ) &\equiv \lim_{\eta ' \to - \infty } e^{\frac{i}{\hbar }H_{0}
\eta } e^{-\frac{i}{\hbar }H ( \eta -\eta ' ) }
e^{-\frac{i}{
\hbar }H_{0}\eta '}. \label{eq:seta}
\end{align}


\begin{SCfigure}
\includegraphics{07f01}
\caption{Schematic representation of the scattering process.}
\label{fig:6.1}
\end{SCfigure}

To better understand how scattering theory describes the time evolution,
we turn to Figure \ref{fig:6.1}. In this figure, we have plotted the
state of the physical system (represented abstractly as a point on the
vertical axis) as a function of time. The exact development of the
system, governed by the complete evolution operator $ U $, is shown by
the thick line $ A \to D $. In asymptotic regions (when the time $ t $
is either very negative or very positive), the interaction between the
colliding parts of the system is weak. In these regions, the exact time
evolution can be rather well approximated by the free
operator~$ U_{0} $. These free ``trajectories'' are shown in the figure by two
thin straight lines with arrows: one for very positive times
$ C \to D $ and the other for very negative times $ A \to B $. The thick
line (the exact interacting time evolution) asymptotically approaches
these thin lines (free evolution) in the remote past (near $ A $) and
in the distant future (near~$ D $).

If we extrapolate the future and past free evolutions to the time
$ t = 0 $, we will realize that there is a gap $ B-C $ between these
extrapolated states. The $ S $-operator is defined precisely in such a
way as to close this gap, i.\,e., to connect the free extrapolated states
$ B $ and $ C $, as shown by the dashed arrow in the figure.

So, in the theory of scattering, the exact time evolution $ A \to D $
is approximated in three stages: first the system develops freely up to
the time instant $ t = 0 $, i.\,e., from $ A $ to $ B $. Then there is a
sharp jump $ B \to C $, represented by the $ S $-operator. Finally, the
system again goes into the free evolution mode $ C \to D $. As can be
seen from the figure, this description of the scattering process is
absolutely accurate, as long as we are interested only in the mapping
of asymptotically free states from the remote past ($ A $) into
asymptotically free states in the distant future ($ D $).

It should also be clear that the scattering operator $ S $ contains
information about particle interactions in an averaged form integrated
over the infinite time interval $ t \in (- \infty , \infty ) $. This
operator is not designed to describe the interacting time evolution
during the short interval of collision ($ t \approx 0 $). For these
purposes, we would need the complete interacting time evolution operator
$ U $.

In applications, we are mainly interested in matrix elements of the
$ S $-operator
%
%e6 #&#
\begin{align}
S_{i \to f} = \langle f | S | i \rangle , \label{eq:8.62}
\end{align}
where $| i \rangle $ is the initial state of the colliding particles and
$|f \rangle $ is their final state. Such matrix elements are called the
$ S$-\emph{matrix}. \index{S-matrix} Formulas relating the $ S $-matrix to
observable quantities, such as scattering cross sections, can be found
in any textbook on scattering theory \cite{Goldberger,Taylor}.
\index{scattering }

An important property of the $ S $-operator is its ``Poincar\'{e}
invariance,'' i.\,e., zero commutators with generators of the
noninteracting representation of the Poincar\'{e} group
\cite{book,Kazes},
%
%e7 #&#
\begin{align}
[S,H_{0}] = [S, \boldsymbol{P}_{0}] = [S,
\boldsymbol{J}_{0}] = [S, \boldsymbol{K}_{0}] = 0.
\label{eq:S-rel-inv}
\end{align}
In particular, the commutator $ [S, H_{0}] = 0 $ implies that in
(\ref{eq:8.65}) we can interchange $ U_{0} $ and $ S $, so that the full
interacting time evolution can be represented as the following product
of the free evolution and the $ S $-operator:
%
%e8 #&#
\begin{align}
U(\infty \gets -\infty ) = S U_{0}(\infty \gets -\infty ) =
U_{0}( \infty \gets -\infty ) S. \label{eq:8.65a}
\end{align}

%s1.2 #&#
\subsection{$S$-operator in perturbation theory}\label{ss:perturbation}
There are many methods for calculating the $ S $-operator. In this book,
we will mainly use perturbation theory. To derive the perturbation
theory series for the $ S $-operator, we first note that the operator
$ S (t) $ in (\ref{eq:seta}) satisfies the equation
%
%e9 #&#
\begin{align}
\frac{d}{dt} S(t)
&= \frac{d}{dt} \lim_{t' \to - \infty } e^{\frac{i}{\hbar }H_{0}t} e
^{-\frac{i}{\hbar }H ( t-t' ) } e^{-\frac{i}{\hbar }H_{0}t'}
\nonumber
\\
&= \lim_{t' \to - \infty } \biggl(e^{\frac{i}{\hbar }H_{0}t} \biggl(
\frac{i}{
\hbar }H_{0} \biggr) e^{-\frac{i}{\hbar }H ( t-t' ) } e^{-\frac{i}{
\hbar }H_{0}t'} +
e^{\frac{i}{\hbar }H_{0}t} \biggl( -\frac{i}{\hbar }H \biggr) e^{-\frac{i}{\hbar }H ( t-t' ) }
e^{-\frac{i}{\hbar }H_{0}t'} \biggr)
\nonumber
\\
&= -\frac{i}{\hbar } \lim_{t' \to - \infty } e^{\frac{i}{\hbar }H
_{0}t} (H-
H_{0}) e^{-\frac{i}{\hbar }H ( t-t' ) } e^{-\frac{i}{
\hbar }H_{0}t'}
\nonumber
\\
&= -\frac{i}{\hbar } \lim_{t' \to - \infty } e^{\frac{i}{\hbar }H
_{0}t} V
e^{-\frac{i}{\hbar }H ( t-t' ) } e^{-\frac{i}{\hbar }H
_{0}t'}
\nonumber
\\
&= -\frac{i}{\hbar } \lim_{t' \to - \infty } e^{\frac{i}{\hbar }H
_{0}t} V
e^{-\frac{i}{\hbar }H_{0}t} e^{\frac{i}{\hbar }H_{0}t}e^{-\frac{i}{
\hbar }H ( t-t' ) } e^{-\frac{i}{\hbar }H_{0}t'}
\nonumber
\\
&= -\frac{i}{\hbar } \lim_{t' \to - \infty } V(t) e^{\frac{i}{\hbar
}H_{0}t}e^{-\frac{i}{\hbar }H ( t-t' ) }
e^{-\frac{i}{\hbar }H
_{0}t'}
\nonumber
\\
&= -\frac{i}{\hbar } V(t) S(t), \label{eq:8.66}
\end{align}
where we denote\footnote{Note that the $ t $-dependence
\index{t-dependence} in $ V (t) $ does not mean that we are considering
time-dependent interactions. The argument $ t $ has no relation with the
true \emph{time} dependence of operators in the Heisenberg representation.
The latter must be generated by the full interacting Hamiltonian $ H $,
not by the free Hamiltonian $ H_{0} $, as in equation (\ref{eq:8.67}).
To emphasize this difference, in cases like (\ref{eq:8.67}) we will talk
about ``$ t $-dependence'' instead of ``time-dependence''.}
%
%e10 #&#
\begin{align}
V(t) = e^{\frac{i}{\hbar }H_{0}t} V e^{-\frac{i}{\hbar }H_{0}t}. \label{eq:8.67}
\end{align}

By a direct substitution, one can verify that a solution of equation
(\ref{eq:8.66}) with the natural initial condition $ S (- \infty ) = 1
$ is given by the Dyson perturbation series \index{Dyson series}
%
\begin{align*}
S(t) = 1 - \frac{i}{\hbar }\int_{-\infty }^{t} V \bigl(
t' \bigr) dt' - \frac{1}{\hbar^{2}} \int
_{-\infty }^{t} V \bigl( t' \bigr)
dt' \int_{-\infty }^{t'} V \bigl(
t'' \bigr) dt'' + \cdots .
\end{align*}
Therefore, the $ S $-operator can be calculated by substituting
$ t = + \infty $ as the upper limit of $ t $-integrals, so we have
%
%e11 #&#
\begin{align}
S = 1 - \frac{i}{\hbar }\int_{-\infty }^{+\infty } V \bigl(
t' \bigr) dt' - \frac{1}{\hbar^{2}} \int
_{-\infty }^{+\infty } V \bigl( t' \bigr)
dt' \int_{-\infty }^{t'} V \bigl(
t'' \bigr) dt'' + \cdots .
\label{eq:8.68}
\end{align}

As a rule, only the first few terms of this series are used in
calculations, assuming, therefore, that the interaction $ V $ is so weak
that it can be regarded as a small perturbation; and the scattering
itself is just a small correction to the free propagation of particles.
We shall say that a term in the perturbation theory series has
\emph{order} \index{order of perturbation theory} $ n $ if it contains a
product of $ n $ factors $ V $. Thus, in (\ref{eq:8.68}) we have
explicit terms in the zero, first and second perturbation orders. We do
not want to dwell on mathematical details related to (nontrivial)
convergence properties of the expansion (\ref{eq:8.68}). Throughout this
book, we will tacitly assume that all relevant perturbation series do
converge.

%s1.3 #&#
\subsection{Convenient notation for $t$-integrals}\label{ss:convenient}
We shall often use the following symbols for $ t $-integrals:
\label{lb:underline}
\label{lb:underbrace}
%
%e12 #&#
%e13 #&#
\begin{align}
\underline{Y(t)} &\equiv -\frac{i}{\hbar } \int_{-\infty }^{t}
Y \bigl( t' \bigr) d t'. \label{eq:underline}
\\
\underbrace{Y} &\equiv -\frac{i}{\hbar } \int_{-\infty }^{+\infty }
Y \bigl( t' \bigr) d t' = \underline{Y(\infty )}.
\label{eq:underbrace}
\end{align}
In this notation, the perturbation theory series for the $ S $-operator
(\ref{eq:8.68}) has a compact form. We have
%
%e14 #&#
%e15 #&#
\begin{align}
S &= 1 + \underbrace{\Sigma }, \label{eq:8.69}
\\
\Sigma (t) &= V(t) + V(t) \underline{V \bigl( t' \bigr) } + V(t)
\underline{V \bigl( t' \bigr) \underline{V \bigl(
t'' \bigr) }} + V(t) \underline{V \bigl(
t' \bigr) \underline{V \bigl( t'' \bigr)
\underline{V \bigl( t''' \bigr) }}} +
\cdots . \label{eq:8.70}
\end{align}

Equations (\ref{eq:8.69})--(\ref{eq:8.70}) do not represent the only
perturbative expansion of the $ S $-operator and perhaps not even the
most convenient one. In quantum field theory, the preference is given
to the time-ordered perturbation series \index{time-ordered series}
\cite{book,Peskin}, which uses time ordering of interaction operators
in the following integrand\footnote{When applied to a product of
several $ t $-dependent boson operators, the time ordering symbol
\index{time ordering} $ T $ changes the order of the operators so that
their $ t $ arguments increase from right to left, e.\,g.,
%
%e16 #&#
\begin{align}
T\bigl[A(t_{1})B(t_{2})\bigr] = \lleft\{ \begin{array}{@{}ll}
A(t_{1})B(t_{2}), &\mbox{if  } t_{1} > t_{2},
\\
B(t_{2})A(t_{1}), &\mbox{if  } t_{1} < t_{2}.
\end{array}
\rright. \label{eq:time-order}
\end{align}
}:
%
%e17 #&#
\begin{align}
S &= 1 - \frac{i}{\hbar }\int
_{-\infty }^{+\infty } dt_{1} V(t
_{1}) - \frac{1}{2!\hbar^{2}} \int
_{-\infty }^{+\infty } dt
_{1} dt_{2} T\bigl[V(t_{1}) V(t_{2})
\bigr]
\nonumber
\\
&\quad  +\frac{i}{3! \hbar^{3}} \int
_{-\infty }^{+\infty } dt_{1} dt
_{2} dt_{3}T\bigl[V(t_{1}) V(t_{2})
V(t_{3})\bigr]
\nonumber
\\
&\quad  +\frac{1}{4!\hbar^{4}} \int
_{-\infty }^{+\infty } dt_{1} dt
_{2} dt_{3} dt_{4}T\bigl[V(t_{1})
V(t_{2}) V(t_{3}) V(t_{4})\bigr] + \cdots .
\label{eq:F-D}
\end{align}
In the second volume of our book, we will find another useful
perturbative expression, proposed by Magnus
\cite{Magnus,Magnus2,Blanes08}, i.\,e., \index{Magnus series}
%
%e18 #&#
\begin{align}
S = \exp ( \underbrace{\Phi } ) \label{eq:8.71}.
\end{align}
In this formula, the Hermitian operator $ \Phi (t) $ is called the
\emph{scattering} \emph{phase}.\label{lb:phi-oper}
\index{scattering phase operator} It is represented by a series of
multiple commutators with $ t $-integrals,
%
%e19 #&#
\begin{align}
\Phi (t) &= V(t) - \frac{1}{2
} \Bigl[ \underline{V \bigl( t'
\bigr) },V(t) \Bigr] +\frac{1}{6} \Bigl[ \underline{\underline{V \bigl(
t'' \bigr) },\bigl[V \bigl( t' \bigr)
},V(t)\bigr] \Bigr]
\nonumber
\\
&\quad  +\frac{1}{6} \Bigl[ \underline{ \bigl[ \underline{V \bigl(
t'' \bigr) },V \bigl( t' \bigr) \Bigr]
},V(t) \bigr] -\frac{1}{12} \biggl[ \underline{\underline{\underline{V
\bigl( t''' \bigr) },\bigl[\bigl[V \bigl(
t'' \bigr) },V \bigl( t' \bigr)
\bigr]},V(t)\bigr] \biggr]
\nonumber
\\
&
\quad  -\frac{1}{12} \biggl[ \underline{\bigl[\underline{ \underline{V \bigl(
t''' \bigr) },\bigl[V \bigl(
t'' \bigr) },V \bigl( t' \bigr) \bigr]
\bigr]},V(t) \biggr]
\nonumber
\\
&\quad  -\frac{1}{12} \biggl[ \underline{\underline{\bigl[ \underline{V \bigl(
t''' \bigr) },V \bigl(
t'' \bigr) \bigr]},\bigl[V \bigl( t'
\bigr) },V(t)\bigr] \biggr] +\cdots . \label{eq:8.72}
\end{align}
An important advantage of equation (\ref{eq:8.71}) is that it explicitly
preserves the unitarity of the $ S $-operator in each perturbation
order.\footnote{The argument of the exponent is an \xch{anti-Hermitian}{antihermitian}
operator $ \underbrace{\Phi } $; therefore the exponent itself is
unitary.} The three listed perturbation theory series (Dyson,
time-ordered and Magnus) are equivalent in the sense that in the limit
$ n \to \infty $ they converge to the same $ S $-operator. However, in
each finite order $ n $ their terms can differ.

For brevity, we will often omit $ t $-arguments in operator expressions.
Then equations (\ref{eq:8.70}) and (\ref{eq:8.72}) simplify, even more,
to
%
%e20 #&#
%e21 #&#
\begin{align}
\Sigma &= V + V \underline{V} + V\underline{V \underline{V}} + \cdots ,
\label{eq:7.63c}
\\
\Phi &= V - \frac{1}{2} [ \underline{V}, V ] +\frac{1}{6} \Bigl[
\underline{\underline{V},[V},V] \Bigr] + \cdots . \label{eq:7.63b}
\end{align}

%s1.4 #&#
\subsection{Adiabatic switching of interaction}\label{ss:adiaba}
In formulas for scattering operators (\ref{eq:8.70}) and
(\ref{eq:8.72}), we encounter $ t $-integrals $ \underline{V (t)} $.
Straightforward calculation of such integrals leads to a rather
depressing result. To understand, let us introduce the complete basis
$ | n \rangle $ of eigenvectors of the free Hamiltonian,\vspace*{-6pt}
%
%e22 #&#
%e23 #&#
\begin{align}
H_{0} |n\rangle &= E_{n}|n\rangle , \label{eq:basis1}
\\
\sum_{n}|n \rangle \langle n | &= 1,
\label{eq:basis2}
\end{align}
and calculate matrix elements of $\underline{V(t)}$ in this basis. Then
we have
%
%e24 #&#
\begin{align}
\langle n \vert \underline{V(t)} \vert m \rangle &\equiv
-\frac{i}{\hbar } \int
_{-\infty }^{t} \langle n \vert e
^{\frac{i}{\hbar }H_{0}t'} V e^{-\frac{i}{\hbar }H_{0}t'} \vert m \rangle dt' =
-\frac{i}{\hbar } V_{nm} \int
_{-\infty } ^{t}
e^{\frac{i}{\hbar }(E_{n}-E_{m})t'} dt'
\nonumber
\\
&= - V_{nm} \biggl( \frac{
e^{\frac{i}{\hbar }(E_{n}-E_{m})t}}{E_{n}-E
_{m}} - \frac{
e^{\frac{i}{\hbar }(E_{n}-E_{m})(-\infty )}}{ E_{n}-E
_{m}} \biggr) .
\label{eq:9.54}
\end{align}
What can we do with the meaningless term on the right-hand side that
contains $(-\infty )$?

This term can be made harmless if we take into account the important
fact that the $ S $-operator cannot be applied to all states in the
Hilbert space. According to our discussion in Section
\ref{sc:scatter}, scattering theory is applicable in its entirety only
to \emph{scattering} \emph{states} $ | \Psi \rangle $,
\index{scattering states} in which free particles are far apart in the
asymptotic limits $ t \to \pm \infty $, so that the time evolution of
these states coincides with free evolution in the remote past and in the
distant future. Naturally, these assumptions are inapplicable to all
states in the Hilbert space. For example, the time evolution of bound
states of the full Hamiltonian $ H $ does not resemble the free
evolution in any time interval. It turns out that if we restrict our
theory only to scattering states,\footnote{Usually, such states can
be constructed from asymptotic wave packets, which are well localized
in both the position and momentum spaces; see, for example, Section
\ref{ss:limit}.} then there are no problems with $ t $-integrals.

Indeed, for scattering states $|\Psi \rangle $, the interaction operator
effectively vanishes in asymptotic regions, so we can write\vadjust{\goodbreak}
%
%e25 #&#
\begin{align}
\lim_{t \to \pm \infty } V e^{-\frac{i}{\hbar }H_{0}t}|\Psi \rangle &= 0,
\nonumber
\\*
\lim_{t \to \pm \infty }V(t)|\Psi \rangle &= \lim_{t \to \pm \infty
}e^{\frac{i}{\hbar }H_{0}t}
\bigl( V e^{-\frac{i}{\hbar }H_{0}t}|\Psi \rangle \bigr) = 0. \label{eq:9.55}
\end{align}
How can we apply this condition to calculations of integrals like
(\ref{eq:9.54})?

One approach to a rigorous formulation of scattering theory is the
explicit consideration of localized wave packets
\cite{Goldberger}. Then, the cluster separability ($=$ short range) of the
interaction $ V $ ensures the correct asymptotic behavior of colliding
particles and the validity of equation (\ref{eq:9.55}). However, this
approach is rather complicated, and we prefer to keep away from wave
packets.

There is another, less rigorous, but shorter way to achieve the same
goal -- to use a trick known as the \emph{adiabatic} \emph{switching}.
\index{adiabatic switching} The idea is to add the property
(\ref{eq:9.55}) ``by hands.'' To do this, we multiply $ V (t) $ by a
real nonnegative function of $ t $ that grows slowly from zero ($=$
interaction is ``off'') at $ t = - \infty $ to 1 in the vicinity of
$ t \approx 0 $ (interaction is ``on'') and then slowly decreases back
to zero at $ t = +\infty $ (interaction ``switches off'' again). For
example, one convenient choice for such a function is the exponent
%
%e26 #&#
\begin{align}
V(t) = e^{\frac{i}{\hbar }H_{0}t} V e^{-\frac{i}{\hbar }H_{0}t}e^{-
\epsilon |t|}. \label{eq:adiaba}
\end{align}
If the parameter $ \epsilon $ is small and positive, then such a
modification of the interaction operator will have no effect on the
movement of wave packets and on the $ S $-matrix.\footnote{Indeed,
when the interaction is ``off,'' the wave packets are far from each
other, anyway.} For the integral (\ref{eq:9.54}), we then get
%
\begin{align*}
\langle n \vert \underline{V(t)} \vert m \rangle &
\approx - V_{nm} \biggl( \frac{
e^{\frac{i}{\hbar }(E_{n}-E_{m})t - \epsilon |t|}}{E
_{n}-E_{m}} - \frac{
e^{\frac{i}{\hbar }(E_{n}-E_{m})(-\infty ) -
\epsilon (\infty )}}{ E_{n}-E_{m}}
\biggr)
\\
&= - V_{nm} \frac{
e^{\frac{i}{\hbar }(E_{n}-E_{m})t - \epsilon |t|}}{E
_{n}-E_{m}}.
\end{align*}
At the end of the calculations, we have to go to the limit $ \epsilon
\to + 0 $. Then the $ t $-integral becomes
%
%e27 #&#
\begin{align}
\langle n \vert \underline{V(t)} \vert m \rangle
\longrightarrow - V_{nm} \frac{ e^{\frac{i}{\hbar }(E_{n}-E_{m})t }}{E_{n}-E_{m}} \label{eq:9.56}
\end{align}
and the unpleasant expression $e^{i \infty }$ vanishes.

The ``adiabatic switching'' trick makes possible an alternative
derivation of equation (\ref{eq:8.68}). Take equation
(\ref{eq:time-evoll}) with the initial time in the remote past
$ t_{0} = - \infty $ and the final time in the distant future
$ t = + \infty $. Then we have
%
\begin{align*}
|\Psi (+\infty ) \rangle
&=\lim_{t \to +\infty }e^{-\frac{i}{\hbar }H_{0}
(t-t_{0})} \Biggl(1-\frac{i}{
\hbar } \int
_{-\infty }^{\infty } e^{\frac{i}{\hbar }H_{0}  ( t'-t
_{0} ) } V e^{-\frac{i}{\hbar }H_{0}  ( t'-t_{0} ) }dt'
\\
&\quad  -\frac{1}{\hbar^{2}} \int_{-\infty }^{\infty }
e^{\frac{i}{\hbar
}H_{0}
 ( t'-t_{0} ) } V e^{-\frac{i}{\hbar }H_{0}  ( t'-t
_{0} ) }dt' \int_{-\infty }^{t'}
e^{\frac{i}{\hbar }H_{0}  ( t''-t
_{0} ) } V e^{-\frac{i}{\hbar }H_{0}  ( t''-t_{0} ) }dt'' + \cdots
\Biggr)
\\
&\quad  \times|\Psi (-\infty ) \rangle .
\end{align*}
Then change the integration variables $ t'-t_{0} = \tau '$ and
$ t''- t_{0} = \tau ''$ so that\footnote{For brevity, we do not show
the ``adiabatic switching'' factors explicitly. They force the
integrands to vanish at $\pm \infty $ asymptotics. So, they allow us to
leave the infinite integration limits ($ - \infty $ and $ \infty $)
unchanged.}
%
\begin{align*}
|\Psi (+\infty ) \rangle
&=U_{0}(\infty \gets -\infty )\lim_{t \to +\infty } \Biggl(1-
\frac{i}{
\hbar } \int_{-\infty }^{\infty } e^{\frac{i}{\hbar }H_{0} \tau '} V e
^{-\frac{i}{\hbar }H_{0} \tau '}d \tau '
\\
&\quad  -\frac{1}{\hbar^{2}} \int_{-\infty }^{\infty }
e^{\frac{i}{\hbar
}H_{0}
\tau '} V e^{-\frac{i}{\hbar }H_{0} \tau '}d\tau ' \int_{-
\infty }^{\tau '}
e^{\frac{i}{\hbar }H_{0} \tau ''} V e^{-\frac{i}{
\hbar }H_{0} \tau ''}d \tau '' + \cdots
\Biggr) |\Psi (-\infty ) \rangle
\\
&= U_{0}(\infty \gets -\infty ) S |\Psi (-\infty ) \rangle .
\end{align*}
Comparing this formula with equation (\ref{eq:8.65a}), we conclude that
the $ S $-factor is exactly as in (\ref{eq:8.68}).

%s1.5 #&#
\subsection{$T$-matrix}\label{ss:t-matrix}
In this subsection we will get acquainted with a useful concept of the
$ T $ matrix.\footnote{I am grateful to Cao Bin for online
communications that led to the writing of this subsection.} Let us
calculate matrix elements of the $ S $-operator (\ref{eq:8.68}) in the
basis of eigenvectors of the free Hamiltonian
(\ref{eq:basis1})--(\ref{eq:basis2}).\footnote{Summation over repeated
indices $ k $ and $ l $ is implied. Equation (\ref{eq:9.56}) is used for
$ t $-integrals.} We have
%
%e28 #&#
\begin{align}
\langle n|S|m \rangle
&=\delta_{nm} -\frac{i}{\hbar } \int_{-\infty }^{\infty }
\langle n \vert e^{\frac{i}{\hbar }H_{0} t'} V e^{-\frac{i}{\hbar }H_{0} t'} \vert m
\rangle dt'
\nonumber
\\
&\quad  -\frac{1}{\hbar^{2}} \int_{-\infty }^{\infty } \langle n
\vert e^{\frac{i}{\hbar }H_{0} t'} V e^{-\frac{i}{\hbar }H_{0} t'} \vert k \rangle
dt' \int_{-\infty }^{t'} \langle k \vert e^{\frac{i}{\hbar }H_{0} t''} V e^{-\frac{i}{\hbar }H_{0}
t''} \vert m \rangle
dt'' + \cdots
\nonumber
\\
&=\delta_{nm} -\frac{i}{\hbar } \int_{-\infty }^{\infty }
e^{\frac{i}{
\hbar }(E_{n} - E_{m}) t'} V_{nm} dt'
\nonumber
\\
&\quad  -\frac{1}{\hbar^{2}} \int_{-\infty }^{\infty }
e^{\frac{i}{\hbar
}(E_{n}
- E_{k}) t'} V_{nk} dt' \int_{-\infty }^{t'}
e^{\frac{i}{
\hbar }(E_{k} - E_{m}) t''} V_{km} dt'' + \cdots
\nonumber
\\
&=\delta_{nm} -2 \pi i \delta (E_{n} - E_{m})
V_{nm} +\frac{i}{
\hbar } \int_{-\infty }^{\infty }
e^{\frac{i}{\hbar }(E_{n} - E_{k}) t'} dt' \frac{e^{\frac{i}{\hbar }(E_{k} - E_{m}) t'}}{E_{m} - E_{k}} V _{nk}
V_{km} + \cdots
\nonumber
\\
&=\delta_{nm} -2 \pi i \delta (E_{n} - E_{m})
V_{nm} +2\pi i \delta (E_{n} - E_{m})
\frac{1}{E_{m} - E_{k}} V_{nk} V_{km} + \cdots
\nonumber
\\
&=\delta_{nm} -2 \pi i \delta (E_{n} - E_{m})
V_{nk}
\nonumber
\\
&\quad  \times\biggl( \delta_{km} + \frac{ 1}{E_{m} -
E_{k}}V_{km} +
\frac{ 1}{E
_{m} - E_{k}}V_{kl} \frac{ 1}{E_{m} - E_{l}} V_{lm} + \cdots
\biggr)
\nonumber
\\
&=\delta_{nm} -2 \pi i \delta (E_{n} - E_{m})
T_{nm}. \label{eq:Snm}
\end{align}

The first term is the unit matrix expressing the free propagation of
particles. The matrix in the second term is called the \emph{transition}
\emph{matrix} (or $T$-\emph{matrix}).\footnote{Taking into account the fact that
the $ T $-matrix enters the $ S $-matrix multiplied by the
delta-function of energy $ \delta (E_{n} - E_{m}) $, we have denoted
$ E = E_{m} = E_{n} $.}\index{T-matrix} We have
%
\begin{align*}
 T_{nm}
&\equiv V_{nk} \biggl( \delta_{km} + \frac{ 1}{E_{m} -
E_{k}}V_{km}
+ \frac{ 1}{E_{m} - E_{k}}V_{kl} \frac{ 1}{E_{m} - E_{l}} V_{lm} +
\cdots \biggr)
\\
&= \langle n \vert V \vert k \rangle \langle k |m \rangle +
\langle n \vert V(E_{m} - E_{k})^{-1}
\vert k \rangle \langle k |V |m \rangle
\\
&\quad  + \langle n \vert V ( E_{m} - E_{k})^{-1}
\vert k \rangle \langle k \vert V(E_{m} -
E_{l})^{-1} \vert l \rangle \langle l
\vert V \vert m \rangle + \cdots
\\
&= \langle n \vert V \vert k \rangle \langle k |m \rangle +
\langle n \vert V(E - H_{0})^{-1} \vert k
\rangle \langle k \vert V \vert m \rangle
\\
&\quad  + \langle n \vert V ( E - H_{0})^{-1}
\vert k \rangle \langle k \vert V(E - H_{0})^{-1}
\vert l \rangle \langle l \vert V \vert m \rangle +
\cdots
\\
&= \langle n \vert V \vert m \rangle + \langle n \vert
V(E - H_{0})^{-1}V \vert m \rangle
\\
&\quad  + \langle n \vert V ( E - H_{0})^{-1} V(E -
H_{0})^{-1} V \vert m \rangle + \cdots
\\
&= \langle n \vert V \biggl( 1 + \frac{1}{ E - H_{0}}V +
\frac{1}{E
-H_{0}
}V\frac{1}{E -H_{0} }V + \cdots \biggr) \vert m
\rangle .
\end{align*}
The infinite series in the parenthesis can be summed by the standard
formula $(1-x)^{-1} = 1 + x + x^{2} +\cdots $, so we have
%
\begin{align*}
T_{nm} &= \langle n \vert V \frac{1}{1 - (E - H_{0} )^{-1}V}
\vert m \rangle = \langle n \vert V (E - H_{0}) (E
- H_{0} - V)^{-1} \vert m \rangle
\nonumber
\\
&= \langle n \vert V (E - H_{0}) (E - H)^{-1}
\vert m \rangle = \langle n | T(E)|m \rangle .
\nonumber
\end{align*}
Thus, the $ T $-matrix is represented by matrix elements of the
energy-dependent $ T (E) $-\allowbreak operator and we have
%e29 #&#
\begin{align}
T(E) \equiv V (E - H_{0}) (E - H)^{-1}. \label{eq:SofE}
\end{align}
The beauty of this result is that it provides a closed expression for
the $ S $-operator that goes beyond perturbation theory. This result is
widely used in numerical calculations
\cite{Rescigno,Brown-Jackson,Korchin,Dubovyk}.

%s1.6 #&#
\subsection{$S$-matrix and bound states}\label{ss:s-matr-bound}
The formal expression (\ref{eq:SofE}) can be used to derive an important
connection between poles of the $ S $-matrix and energies of bound
states. Our derivation will be formally heuristic. More rigorous
reasoning can be found in textbooks on scattering theory
\cite{Goldberger,Taylor}.\looseness=-1

We already mentioned that scattering theory can be formulated only for
states that behave asymptotically as free ones. The energy $ E $ of such
states exceeds the energy $ E_{0} $ of separated reactants at rest, for
which we have
%
\begin{align*}
E_{0} =\sum_{a=1}^{N}
m_{a}c^{2}.
\end{align*}
Therefore, the operator $ T (E) $ introduced in (\ref{eq:SofE}) has a
well-defined meaning only in the energy interval
%
%e30 #&#
\begin{align}
E \in [E_{0}, \infty ). \label{eq:eine}
\end{align}

There are reasons to believe that this operator is an analytic function
of energy~$ E $. Therefore, it would be interesting to find out where
the poles of this function are located. We can expect the appearance of
poles at those values of $ E $, where the denominator of the expression
$ (E - H_{0}) (E - H) ^{- 1} $ in (\ref{eq:SofE}) vanishes. Hence (at
least some of) the poles $ E _{\alpha } $ can be found as solutions of
the eigenvalue equation
%
\begin{align*}
(H - E_{\alpha }) |\Psi_{\alpha } \rangle = 0.
\end{align*}
Obviously, this is the familiar stationary Schr\"{o}dinger equation
(\ref{eq:6.71a}) for bound states. This means that there is a connection
between poles of the $ T $-operator and energies of bound states
$ E _{\alpha } $ of the full Hamiltonian $ H $.\footnote{Similarly,
the $S$-operator can be also regarded as an analytic function
$S(E)$ on the complex energy plane with poles at positions $E=E_{
\alpha }$.} These energies are always lower than $ E_{0} $, i.\,e.,
formally they are outside the domain of the operator $ T (E) $.
Therefore, the above-mentioned connection (poles of the $ T $-operator)
$ \leftrightarrow $ (energies of bound states) can be established only
in the sense of analytic continuation of the operator $ T (E) $ from its
natural domain (\ref{eq:eine}) to values below $ E_{0} $.

It is important to emphasize that the possibility of finding the
energies of the bound states $ E _{\alpha } $ from the $ T $-operator
does not mean that the eigenvectors of these states $ | \Psi_{\alpha
} \rangle $ can be found in the same way. All bound states are
eigenstates of the $ T $-operator, corresponding only to one (infinite)
eigenvalue. Therefore, even knowing the exact $ T $-operator, the most
we can do is to find a subspace in $ \mathscr{H} $ that is the linear
span of all bound states. This ambiguity is closely related to the
scattering equivalence of Hamiltonians, which we shall consider in the
next section.

%s2 #&#
\section{Scattering equivalence}\label{sc:scatt-equiv}
The results of the previous section indicate that even exact knowledge
of the $ S $-opera\-tor does not allow us to completely reconstruct the
corresponding Hamiltonian $ H $. In other words, many different
Hamiltonians can have identical scattering properties. Here we will
discuss these issues in more detail, because they will play an important
role in \xch{Volume~3}{Volume $\mathbf{III}$}.

%s2.1 #&#
\subsection{Equivalent Hamiltonians}\label{ss:scatt-equiv}
The $ S $-operator and the Hamiltonian are two fundamentally different
ways of describing dynamics. From the Hamiltonian $ H $ one can obtain
the evolution operator $ U (t \gets t_{0}) \equiv e ^{- \frac{i}{
\hbar } H (t-t_{0})} $, which describes in detail the development of
states in all time intervals, both large and small. On the other hand,
the $ S $-operator represents only the ``integrated'' time evolution in
the infinite interval. In other words, if we know the state of the
system in the remote past $ | \Psi (- \infty ) \rangle $, the free
Hamiltonian $ H_{0} $ and the scattering operator $ S $, then we can
find the final state of the system in the far future (\ref{eq:8.65a}),
i.\,e.,
%
\begin{align*}
|\Psi (\infty ) \rangle &= U(\infty \gets -\infty ) |\Psi (-\infty ) \rangle =
U_{0}(\infty \gets -\infty ) S |\Psi (-\infty ) \rangle .
\end{align*}
However, we cannot say anything about the system's evolution in the
interacting regime.

Despite its incomplete nature, the information contained in the $ S
$-operator is fully sufficient for the analysis of most experiments in
high-energy physics. In particular, from the $ S $-operator one can
obtain accurate scattering cross sections as well as energies and
lifetimes of stable and metastable bound states. This situation gave the
impression that an exhaustive theory of elementary particles could be
constructed on the basis of the $ S $-operator alone without resorting
to the Hamiltonian and wave functions. However, this impression is
deceptive, because the description of physics by means of scattering
theory is incomplete, and such a theory is applicable only to a limited
class of experiments.

Knowing the full interacting Hamiltonian $ H $, we can calculate the
corresponding $ S $-operator by formulas (\ref{eq:8.69}), (\ref{eq:F-D})
or (\ref{eq:8.71}). However, the converse is not true: even if the
$ S $-operator is fully known, it is impossible to recover the unique
underlying Hamiltonian. The same $ S $-operator can be obtained from
many different Hamiltonians.

Suppose that two Hamiltonians $ H $ and $ H'$ are related to each other
by a unitary transformation $ e ^{i \Xi } $, i.\,e.,
%
%e31 #&#
\begin{align}
H' = e^{i\Xi } H e^{-i\Xi }. \label{eq:7.34a}
\end{align}
Then they have exactly the same scattering properties\footnote{We say
that such Hamiltonians are \emph{scattering}-\emph{equivalent}.}
\index{scattering equivalence} if the following condition is satisfied:\vadjust{\goodbreak}
%
%e32 #&#
\begin{align}
\lim_{t \to \pm \infty } e^{\frac{i}{\hbar }H_{0} t} \Xi e^{-\frac{i}{
\hbar }H_{0}t}= 0.
\label{eq:8.75}
\end{align}
Indeed, in the limits $\eta \to +\infty $, $\eta ' \to -\infty $, we
obtain from (\ref{eq:seta}) that the two scattering operators are equal
\cite{Ekstein},\footnote{In (\ref{eq:6.101b}) we use condition
(\ref{eq:8.75}) to make replacements
%
\begin{align*}
&\lim_{\eta \to \infty } \biggl[ \exp \biggl( \frac{i}{\hbar }H_{0}
\eta \biggr) \exp ( i\Xi ) \exp \biggl( -\frac{i}{\hbar }H _{0}\eta
\biggr) \biggr]
\\
&\quad = \lim_{\eta ' \to -\infty } \biggl[ \exp \biggl( \frac{i}{\hbar }H_{0}
\eta ' \biggr) \exp ( -i\Xi ) \exp \biggl( -\frac{i}{\hbar }H
_{0}\eta ' \biggr) \biggr] =1.
\end{align*}.\vspace*{-12pt}} i.\,e.,
%
%e33 #&#
%e34 #&#
\begin{align}
S' &= \lim_{\eta ' \to -\infty , \eta \to \infty } e^{\frac{i}{\hbar
}H_{0}\eta }
e^{-\frac{i}{\hbar }H' ( \eta - \eta ' ) } e^{-\frac{i}{
\hbar }H_{0} \eta '}
\nonumber
\\
&= \lim_{\eta ' \to -\infty , \eta \to \infty } e^{\frac{i}{\hbar }H
_{0}\eta } \bigl( e^{i\Xi }
e^{-\frac{i}{\hbar }H ( \eta -
\eta ' ) } e^{-i\Xi } \bigr) e^{-\frac{i}{\hbar }H_{0} \eta '}
\nonumber
\\
&=\lim_{\eta ' \to -\infty , \eta \to \infty } \bigl( e^{\frac{i}{
\hbar }H_{0}\eta } e^{i\Xi }
e^{-\frac{i}{\hbar }H_{0}\eta } \bigr) e ^{\frac{i}{\hbar }H_{0}\eta } e^{-\frac{i}{\hbar }H ( \eta -
\eta ' ) } e^{-\frac{i}{\hbar }H_{0} \eta '}
\nonumber
\\
&
\quad \times\bigl( e^{\frac{i}{\hbar }H_{0} \eta '} e^{-i\Xi } e^{-\frac{i}{
\hbar }H_{0}
\eta '} \bigr)
\label{eq:6.101b}
\\
&= \lim_{\eta ' \to -\infty , \eta \to \infty } e^{\frac{i}{\hbar }H
_{0}\eta } e^{-\frac{i}{\hbar }H ( \eta - \eta ' ) }
e^{-\frac{i}{
\hbar }H_{0}\eta '} = S. \label{eq:6.101a}
\end{align}

Note that, due to Lemma \ref{LemmaA.12}, the energy spectra of the two
equivalent Hamiltonians $ H $ and $ H '$ also coincide. However, their
eigenvectors could be different, and the corresponding description of
dynamics (e.\,g., by equation (\ref{eq:8.61a})) could also differ.
Therefore, two theories predicting the same scattering are not
necessarily equivalent in the full physical sense.

%s2.2 #&#
\subsection{Bakamjian construction of point-form dynamics}\label{bakam-point}
It turns out that the above statement about scattering equivalent
Hamiltonians can be generalized in the sense that even two different
forms of dynamics (for example, the instant and point forms) can have
the same $ S $-operators. This nontrivial fact will be discussed in
Section \ref{ss:form-equiv-point}. To prepare for this discussion, here
we will build a specific version of the point form of dynamics, using
a recipe due to Bakamjian \cite{Bakamjian}. This method is very
similar to the method of Bakamjian--Thomas from Section
\ref{ss:bakamjian}.

We consider a system of $ n \geq 2 $ massive free particles with
noninteracting operators of mass $ M_{0} $, linear momentum
$ \boldsymbol{P} _{0} $, angular momentum $ \boldsymbol{J} _{0} $,
center-of-energy position $ \boldsymbol{R} _{0} $ and spin $
\boldsymbol{S} _{0} = \boldsymbol{J} _{0} - [\boldsymbol{R} _{0}
\times \boldsymbol{P} _{0}] $. Then we introduce two new operators,
%
\begin{align*}
\boldsymbol{Q}_{0} &\equiv \frac{\boldsymbol{P}_{0}}{ M_{0} c^{2}},
\\
\boldsymbol{X}_{0} &\equiv M_{0} c^{2}
\boldsymbol{R}_{0},
\end{align*}
which satisfy the canonical commutation relations
%
\begin{align*}
[X_{0i}, Q_{0i}] &= [X_{0i}, X_{0j}]
= [Q_{0i}, Q_{0j}] = 0,
\\
[X_{0i}, Q_{0j}] &= i \hbar \delta_{ij}.
\end{align*}
Next, we express the generators $ \{H_{0}, \boldsymbol{P} _{0},
\boldsymbol{J} _{0}, \boldsymbol{K} _{0} \} $ of the noninteracting
representation of the Poincar\'{e} group through the alternative set of
operators $ \{M_{0}, \boldsymbol{Q} _{0}, \boldsymbol{X} _{0},
\boldsymbol{S} _{0} \} $ as follows (compare with (\ref{eq:8.24})--(\ref{eq:8.25})):
%
\begin{align*}
\boldsymbol{P}_{0} &= M_{0} c^{2}
\boldsymbol{Q}_{0},
\\
\boldsymbol{K}_{0} &= -\frac{1}{2} \Bigl( \sqrt{1+
Q_{0}^{2}c^{2}} \boldsymbol{X}_{0} +
\boldsymbol{X}_{0} \sqrt{1+ Q_{0}^{2}c^{2}}
\Bigr) - \frac{[\boldsymbol{Q}_{0} \times
\boldsymbol{S}_{0}]}{1 + \sqrt{1+
Q_{0}^{2}c^{2}}},
\\
\boldsymbol{J}_{0} &= [\boldsymbol{X}_{0} \times
\boldsymbol{Q}_{0}] + \boldsymbol{S}_{0},
\\
H_{0} &= M_{0}c^{2} \sqrt{1+
Q_{0}^{2}c^{2}}.
\end{align*}
Now, a point-form interaction can be introduced by modifying the mass
operator
%
%e35 #&#
\begin{align}
M_{0} \to M \label{eq:mass-mod}
\end{align}
so as to satisfy the following conditions\footnote{As in Section
\ref{ss:bakamjian}, these conditions can be fulfilled by defining
$ M = M_{0} + N $, where $ N $ is a rotationally invariant function of
operators of the relative position and momentum that commute with both
$ \boldsymbol{Q} _{0} $ and $ \boldsymbol{X} _{0} $.}:
%
\begin{align*}
[M, \boldsymbol{Q}_{0}] = [M, \boldsymbol{X}_{0}] = [M,
\boldsymbol{S}_{0}] = 0.
\end{align*}
These conditions guarantee, in particular, that the mass operator
$ M $ is invariant under transformations from the Lorentz subgroup, i.\,e.,
%
\begin{align*}
[M, \boldsymbol{K}_{0}] = [M, \boldsymbol{J}_{0}] = 0.
\end{align*}
The mass modification (\ref{eq:mass-mod}) introduces interaction in
generators of the translation subgroup,
%
%e36 #&#
\begin{align}
\boldsymbol{P} &= M c^{2} \boldsymbol{Q}_{0},
\nonumber
\\*
H &= M c^{2} \sqrt{1 + Q_{0}^{2}c^{2}},
\label{eq:6.3.5}
\end{align}
while Lorentz subgroup generators $\boldsymbol{K}_{0}$ \xch{and}{и} $
\boldsymbol{J}_{0}$ remain interaction-free. So, we succeeded in
defining a nontrivial interaction $ \{H, \boldsymbol{P},
\boldsymbol{J} _{0}, \boldsymbol{K} _{0} \} $ in the point form of
dynamics.

%s2.3 #&#
\subsection{Unitary link between point and instant forms of dynamics}\label{ss:form-equiv-point2}
The $ S $-matrix equivalence of Hamiltonians established in Section
\ref{ss:scatt-equiv} remains valid even if the transformation
$ e ^{i \Xi } $ (\ref{eq:7.34a}) changes the relativistic form of
dynamics \cite{Sokolov_Shatnii,Sokolov_Shatnii2}. Here we are
going to demonstrate this equivalence using the example of the Dirac
instant and point forms \cite{Sokolov_Shatnii}.

To begin with, suppose that we are given a Bakamjian point form of
dynamics with operators
%
\begin{align*}
M &\neq M_{0},
\\
\boldsymbol{P} &= \boldsymbol{Q}_{0} Mc^{2},
\\
\boldsymbol{J} &= \boldsymbol{J}_{0},
\\
\boldsymbol{R} &= \boldsymbol{X}_{0} M^{-1}c^{-2},
\end{align*}
specified in Section \ref{bakam-point}. Then we define a unitary operator
%
\begin{align*}
\Theta = \zeta_{0} \zeta^{-1},
\end{align*}
where
%
\begin{align*}
\zeta_{0} &\equiv \exp \bigl( -i \ln \bigl( M_{0}c^{2}
\bigr) \mathfrak{Z} \bigr) ,
\\
\zeta &\equiv \exp \bigl( -i \ln \bigl( Mc^{2} \bigr) \mathfrak{Z}
\bigr)
\end{align*}
and operator
%
\begin{align*}
\mathfrak{Z} &\equiv \frac{1}{2 \hbar } (\boldsymbol{R} \cdot \boldsymbol{P} +
\boldsymbol{P} \cdot \boldsymbol{R}) = \frac{1}{2
\hbar } (\boldsymbol{Q}_{0}
\cdot \boldsymbol{X}_{0} + \boldsymbol{X} _{0} \cdot
\boldsymbol{Q}_{0})
\end{align*}
was defined in (\ref{eq:4.51a}). Our goal is to show that the set of
operators $\Theta M \Theta^{-1}$, $\Theta \boldsymbol{P} \Theta^{-1}$,
$\Theta \boldsymbol{J}_{0} \Theta^{-1}$ and $\Theta \boldsymbol{R}
\Theta^{-1}$ generates a Poincar\'{e} group representation in the
Bakamjian--Thomas instant form.

Let us define
%
\begin{align*}
\boldsymbol{Q}_{0}(b) \equiv e^{i b \mathfrak{Z}} \boldsymbol{Q}_{0}
e ^{-i b
\mathfrak{Z}}, \quad  b \in \mathbb{R}.
\end{align*}
From the commutator
%
\begin{align*}
[\mathfrak{Z}, \boldsymbol{Q}_{0}] &= i \boldsymbol{Q}_{0},
\end{align*}
it follows that
%
\begin{align*}
\frac{d}{db} \boldsymbol{Q}_{0}(b) &= i [\mathfrak{Z},
\boldsymbol{Q} _{0}] = - \boldsymbol{Q}_{0},
\\
\boldsymbol{Q}_{0}(b) &= e^{- b}\boldsymbol{Q}_{0}.
\end{align*}
This formula remains valid even if $b$ is not a number but any Hermitian
operator commuting with both $\boldsymbol{Q}_{0}$ and $\boldsymbol{X}
_{0}$. For example, if $b = \ln  ( M_{0}c^{2} ) $, then
%
\begin{align*}
\zeta_{0}^{-1}\boldsymbol{Q}_{0}
\zeta_{0} = e^{i \ln  ( M_{0}c
^{2} )  \mathfrak{Z}} \boldsymbol{Q}_{0}
e^{-i
\ln  ( M_{0}c
^{2} )  \mathfrak{Z}} = e^{- \ln  ( M_{0}c^{2} ) } \boldsymbol{Q}_{0} =
\boldsymbol{Q}_{0}/ \bigl( M_{0}c^{2} \bigr) .
\end{align*}
Similarly, one can show
%
\begin{align*}
\zeta^{-1}\boldsymbol{Q}_{0} \zeta &= e^{i \ln  ( Mc^{2} )
\mathfrak{Z}}
\boldsymbol{Q}_{0} e^{-i
\ln  ( Mc^{2} )
\mathfrak{Z}} = \boldsymbol{Q}_{0}/
\bigl( Mc^{2} \bigr) ,
\\
\zeta_{0}^{-1}\boldsymbol{X}_{0}
\zeta_{0} &= e^{i \ln  ( M_{0}c
^{2} )  \mathfrak{Z}} \boldsymbol{X}_{0}
e^{-i
\ln  ( M_{0}c
^{2} )  \mathfrak{Z}} = \boldsymbol{X}_{0}M_{0}c^{2},
\\
\zeta^{-1}\boldsymbol{X}_{0} \zeta &= e^{i \ln  ( Mc^{2} )
\mathfrak{Z}}
\boldsymbol{X}_{0} e^{-i
\ln  ( Mc^{2} )
\mathfrak{Z}} = \boldsymbol{X}_{0}Mc^{2},
\end{align*}
which implies
%
\begin{align*}
\Theta \boldsymbol{P} \Theta^{-1} &= \zeta_{0}
\zeta^{-1} \bigl( \boldsymbol{Q} _{0} Mc^{2}
\bigr) \zeta \zeta_{0}^{-1} = \zeta_{0}
\boldsymbol{Q} _{0} \zeta_{0}^{-1} =
\boldsymbol{Q}_{0} M_{0}c^{2} =\boldsymbol{P}
_{0},
\\
\Theta \boldsymbol{J}_{0} \Theta^{-1} &=
\boldsymbol{J}_{0},
\\
\Theta \boldsymbol{R} \Theta^{-1} &= \zeta_{0}
\zeta^{-1} \bigl( \boldsymbol{X} _{0} / \bigl(
Mc^{2} \bigr) \bigr) \zeta \zeta_{0}^{-1} =
\zeta_{0} \boldsymbol{X}_{0} \zeta_{0}^{-1}
= \boldsymbol{X}_{0} / \bigl( M_{0}c ^{2} \bigr)
= \boldsymbol{R}_{0}.
\end{align*}
From these formulas, it is clear that the transformation $ \Theta $
really changes dynamics from the Bakamjian point form to the
Bakamjian--Thomas instant form.

%s2.4 #&#
\subsection{Scattering equivalence of forms of dynamics}\label{ss:form-equiv-point}
Let us now verify that the scattering operator $ S $, calculated with
the point-form Hamiltonian $ H = M c ^{2} \sqrt{1 + c ^{2}Q_{0}
^{2}} $, is the same as the operator $ S '$ calculated with the
instant-form Hamiltonian $ H' = \Theta H \Theta^{- 1} $. Notice that we
can rewrite equation (\ref{eq:8.73a}) as
%
\begin{align*}
S = \Omega^{+}(H, H_{0}) \Omega^{-}( H,
H_{0}),
\end{align*}
where operators
%
\begin{align*}
\Omega^{\pm }(H, H_{0}) \equiv \lim_{t \to \pm \infty }
e^{\frac{i}{
\hbar }H_{0} t} e^{-\frac{i}{\hbar }Ht}
\end{align*}
are called \emph{M\"{o}ller} \emph{wave} \emph{operators}.
\index{M\"{o}ller wave operators} Next we use the so-called
\emph{Birman}--\emph{Kato} \emph{invariance} \emph{principle}
\index{Birman--Kato invariance principle} \cite{Dollard}, which
states that $ \Omega^{\pm } (H, H_{0}) = \Omega^{ \pm } (f (H), f (H
_{0})) $, where $ f $ can be any smooth function with positive
derivative. Using the relationship between mass operators in the point
($ M $) and instant ($ M '$) forms
%
\begin{align*}
M = \zeta^{-1} M \zeta = \zeta^{-1} \Theta^{-1}
M' \Theta \zeta = \zeta^{-1} \zeta \zeta_{0}^{-1}
M' \zeta_{0} \zeta^{-1} \zeta = \zeta
_{0}^{-1} M' \zeta_{0},
\end{align*}
we obtain
%
\begin{align*}
\Omega^{\pm }(H, H_{0}) &\equiv \Omega^{\pm } \Bigl(
M c^{2}\sqrt{1+ Q_{0}^{2}c^{2}},
M_{0}c^{2} \sqrt{1+ Q_{0}^{2}c^{2}}
\Bigr) = \Omega^{\pm } \bigl( Mc^{2}, M_{0}c^{2}
\bigr)
\\
&= \Omega^{\pm } \bigl( \zeta_{0}^{-1}
M' \zeta_{0} c^{2} , M_{0}c
^{2} \bigr) = \zeta_{0} ^{-1}
\Omega^{\pm } \bigl( M'c^{2}, M_{0}c^{2}
\bigr) \zeta_{0}
\\
&= \zeta_{0} ^{-1} \Omega^{\pm } \Bigl( \sqrt
{\bigl(M'\bigr)^{2} c^{4} + P
_{0}^{2}c^{2}}, \sqrt{M_{0}^{2}
c^{4} + P_{0}^{2}c^{2}} \Bigr) \zeta
_{0}
\\
&= \zeta_{0} ^{-1} \Omega^{\pm } \bigl(
H', H_{0} \bigr) \zeta_{0},
\\[1.5ex]
S' &= \Omega^{+} \bigl( H',
H_{0} \bigr) \Omega^{-} \bigl( H',
H_{0} \bigr) = \zeta_{0} \Omega^{+}(H,
H_{0}) \zeta_{0}^{-1} \zeta _{0}
\Omega^{-}(H, H_{0}) \zeta_{0}^{-1}
\\
&= \zeta_{0} \Omega^{+}(H, H_{0})
\Omega^{-}(H, H_{0}) \zeta_{0}^{-1} =
\zeta_{0} S \zeta_{0}^{-1}.
\end{align*}
Then we notice that $S$ commutes with free generators
(\ref{eq:S-rel-inv}) and therefore with $\zeta_{0}$ as well. Hence,
$S' = S$ and the transformation $\Theta $ conserves the $S$-matrix. This
completes the proof.

In addition to the above results, Sokolov and Shatnii \cite{Sokolov_Shatnii,Sokolov_Shatnii2} established the mutual scattering equivalence of all
three basic forms of dynamics -- instant, point and front. Then it seems
reasonable to assume that the $ S $-operator is not sensitive to the
form of dynamics at all.

The scattering equivalence of Hamiltonians and forms of dynamics gives
us great advantages in calculations. If we are only interested in
scattering amplitudes, energies and lifetimes of bound
states,\footnote{That is, the properties directly related to the
scattering matrix.} then we can choose the Hamiltonian and the form of
dynamics from a wide selection, as convenient. However, as we have
already said, the scattering equivalence does not mean full physical
equivalence of different theories. In the third volume of our book, we
will see that an adequate description of the time evolution and other
inertial transformations is possible only within the instant-form
framework.
\end{document}
