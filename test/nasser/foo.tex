\documentclass[11pt]{book}
\raggedbottom                                  

\usepackage{amsmath,mathtools,amssymb}
\usepackage[activate={true,nocompatibility},final,tracking=true,factor=1100,stretch=10,shrink=10]{microtype}

\usepackage{mwe}
\usepackage{graphicx}
\DeclareGraphicsExtensions{.pdf,.PDF,.png,.PNG,.jpg,.jpeg,.JPG,.JPEG}

\usepackage{color}
\usepackage{listings}
\lstset{language=Matlab,%
    breaklines=true,%
    morekeywords={matlab2tikz},
    keywordstyle=\color{blue},%
    morekeywords=[2]{1}, keywordstyle=[2]{\color{black}},
    identifierstyle=\color{black},%
    showstringspaces=false,%without this there will be a symbol in the places where there is a space
    numbers=left,%
    numberstyle={\tiny \color{black}},% size of the numbers
    numbersep=9pt, % this defines how far the numbers are from the text
    emph=[1]{for,end,break},emphstyle=[1]\color{red}, %some words to emphasise
}


\usepackage{fancyvrb}

\usepackage{etex} %adds more registers
\usepackage{upquote} %to handle correct tex4ht to html conversion of `
\newcommand{\authorMe}[0]
 {\author{\footnotesize \href{mailto:user@comain}{John Doe}}}%

\usepackage[us,12hr]{datetime}

\usepackage{ntheorem}
\newtheorem{theorem}{Theorem}
\usepackage[english]{babel}
\usepackage{blindtext}
\usepackage{caption}

\usepackage{float}  
\usepackage{bera}
\usepackage[T1]{fontenc}
\usepackage{hyperref}

\usepackage[letterpaper,bindingoffset=0.2in,%
            left=1.2in,right=1.2in,top=.8in,bottom=.8in,%
            footskip=.25in]{geometry}

\DeclareMathOperator{\Res}{Res}

\begin{document}   

\begin{description}
\item[] \href{/index.htm}{home}
\item[] \href{../index.htm}{up}
\end{description}

\title{My big title}
\authorMe
\date{Spring 2018 \hspace{.2in} \tiny{Compiled on \today\ at \currenttime}}
\maketitle  

\tableofcontents 

\chapter{Introduction} 
This is main chapter. Here is a nice image

\includegraphics[width=0.7\textwidth]{example-image}

\section{some math}

\blindtext
\pagestyle{empty}
\begin{theorem}[Residue Theorem]
Let $f$ be analytic in the region $G$ except for the isolated 
singularities $a_1,a_2,\dots,a_m$. If $\gamma$ is a closed 
rectifiable curve in $G$ which does not pass through any of the 
points $a_k$ and if $\gamma\approx 0$ in $G$, then
\[
  \frac{1}{2\pi i}\int_\gamma\! f = \sum_{k=1}^m 
  n(\gamma;a_k)\Res(f;a_k)\,.
\]
\end{theorem}

\subsection{this is a subsection}

Here is some code in minipage

\begin{minipage}{0.9\textwidth}
\begin{lstlisting}[basicstyle=\footnotesize]
%check if we converged or not
if k>opt.MAX_ITER || gradientNormTol(k)<=opt.gradientNormTol ...
|| (k>1 && levelSets(k)>levelSets(k-1))% check for getting worst
   keepRunning = false;
else
  ....
end
\end{lstlisting}
\end{minipage}

Here is example using listings

\begin{lstlisting}
%Evaluate J(u) at u
function f = objectiveFunc(u)
 u=u(:);
 N = size(u,1);
 f = 0;
 for i = 1:N-1
     f  = f + 100*(u(i+1)-u(i)^2)^2 + (1-u(i))^2;
 end
end
\end{lstlisting}


\subsubsection{This is subsubsection with images}

These two images should be side by side
\begin{figure}[H]
\centering
\begin{minipage}{0.48\textwidth}
\centering
\captionsetup{width=.8\textwidth}
\includegraphics[width=1\textwidth]{example-image}
\caption{Contour $J(u)$}
\end{minipage}
\hfill\begin{minipage}{0.48\textwidth}
\centering
\captionsetup{width=.8\textwidth}
\includegraphics[width=1\textwidth]{example-image}
\caption{Zooming on $J(u)$}
\end{minipage}
\end{figure}

\chapter{This is a new chapter}
Here is some verbatim

\begin{Verbatim}
K>> gradientNormTol(end-6:end)
....
          16.1440020280613
           17.487837406306
           16.092991548592
          17.4442963174089
\end{Verbatim}

\section{This is a section for more math}

\subsection{problem 1}
\underline{problem} Transform the following problem or system to set of first
order ODE $t^{2}x^{\prime\prime}+tx^{\prime}+\left(  t^{2}-1\right)  x=0$

\underline{solution} Since this is second order ODE, we need two state
variables, say $x_{1},x$

Let $x_{1}=x,x_{2}=x^{\prime}$, hence%
\[
\left.
\begin{array}
[c]{c}%
x_{1}=x\\
x_{2}=x^{\prime}%
\end{array}
\right\}  \overset{\text{take derivative}}{\longrightarrow}\left.
\begin{array}
[c]{c}%
x_{1}^{\prime}=x^{\prime}\\
x_{2}^{\prime}=x^{\prime\prime}%
\end{array}
\right\}  \overset{\text{replace RHS}}{\longrightarrow}%
\begin{array}
[c]{c}%
x_{1}^{\prime}=x_{2}\\
x_{2}^{\prime}=-\frac{x^{\prime}}{t}-\frac{\left(  t^{2}-1\right)  x}%
{t}=-\frac{x_{2}}{t}-\frac{\left(  t^{2}-1\right)  x_{1}}{t}%
\end{array}
\]
Hence the two first order ODE's are (now coupled)%
\begin{align*}
x_{1}^{\prime}  &  =x_{2}\\
x_{2}^{\prime}  &  =-\frac{x_{2}}{t}-\frac{\left(  t^{2}-1\right)  x_{1}}{t}%
\end{align*}
The matrix form of the above is%
\begin{align*}
\mathbf{x}^{\prime}  &  =A\mathbf{x}\\%
\begin{pmatrix}
x_{1}^{\prime}\\
x_{2}^{\prime}%
\end{pmatrix}
&  =%
\begin{pmatrix}
0 & 1\\
-\frac{t^{2}-1}{t} & -\frac{1}{t}%
\end{pmatrix}%
\begin{pmatrix}
x_{1}\\
x_{2}%
\end{pmatrix}
\end{align*}

\subsection{Example on page 500, textbook (Edwards\&Penny, 3rd edition)}

\underline{problem} This problem was solved in textbook using matrix
exponential. Here is solved using the fundamental matrix only. Use the method
of variation of parameters to solve $\mathbf{x}^{\prime}=A\mathbf{x}%
+\mathbf{f}\left(  t\right)  $.%
\begin{align*}
A &  =%
\begin{pmatrix}
4 & 2\\
3 & -1
\end{pmatrix}
\\
\bar{f}\left(  t\right)   &  =%
\begin{pmatrix}
-15\\
4
\end{pmatrix}
te^{-2t}\\
\bar{x}\left(  0\right)   &  =%
\begin{pmatrix}
7\\
3
\end{pmatrix}
\end{align*}


\underline{Solution}

The homogeneous solution was found in the book as%

\[
\bar{x}_{h}=c_{1}%
\begin{pmatrix}
1\\
-2
\end{pmatrix}
e^{-2t}+c_{2}%
\begin{pmatrix}
2\\
1
\end{pmatrix}
e^{5t}%
\]
Following scalar case, the guess would be $\bar{x}_{p}=\left(  \bar{b}+\bar
{a}t\right)  e^{-2t}$ but since $e^{-2t}$ is in the homogeneous, we have to
adjust to be $\bar{x}_{p}=\left(  \bar{b}t+\bar{a}t^{2}\right)  e^{-2t}%
+\bar{c}e^{5t}$. Notice we had to add $\bar{c}e^{5t}$, else it will not work
if we just guessed $\bar{x}_{p}=\left(  \bar{b}t+\bar{a}t^{2}\right)
e^{-2t}\,$\ based on what we would do in scalar case, we will find we get
$\bar{a}=\bar{b}=0$. This seems to be a trial and error stage and one just
have to try to find out. This is why undermined coefficients for systems is
not as easy to use as with scalar case. Hence
\[
\bar{x}_{p}=\left(  \bar{b}t+\bar{a}t^{2}\right)  e^{-2t}+\bar{c}e^{5t}%
\]
Now we plug-in this back into the ODE\ and solve for $\bar{a},\bar{b},\bar{c}%
$. But an easier method is to use Variation of parameters. The fundamental
matrix is%
\begin{align*}
\Phi & =%
\begin{pmatrix}
\bar{x}_{1} & \bar{x}_{2}%
\end{pmatrix}
\\
& =%
\begin{pmatrix}
e^{-2t} & 2e^{5t}\\
-2e^{-2t} & e^{5t}%
\end{pmatrix}
\end{align*}
And
\[
\Phi^{-1}=\frac{%
\begin{pmatrix}
e^{5t} & 2e^{-2t}\\
-2e^{5t} & e^{-2t}%
\end{pmatrix}
^{T}}{\left\vert \Phi\right\vert }=\frac{%
\begin{pmatrix}
e^{5t} & -2e^{5t}\\
2e^{-2t} & e^{-2t}%
\end{pmatrix}
}{e^{3t}+4e^{3t}}=\frac{1}{5}%
\begin{pmatrix}
e^{2t} & -2e^{2t}\\
2e^{-5t} & e^{-5t}%
\end{pmatrix}
\]
Hence using
\begin{align*}
\bar{x}_{p}  & =\Phi\int\Phi^{-1}\bar{f}\left(  t\right)  dt\\
& =\frac{1}{5}\Phi\int%
\begin{pmatrix}
e^{2t} & -2e^{2t}\\
2e^{-5t} & e^{-5t}%
\end{pmatrix}%
\begin{pmatrix}
-15te^{-2t}\\
4te^{-2t}%
\end{pmatrix}
dt\\
& =\frac{1}{5}\Phi\int%
\begin{pmatrix}
-23t\\
-26te^{-7t}%
\end{pmatrix}
dt
\end{align*}
The integral of  $\allowbreak\int-23tdt=\frac{-23}{2}t^{2}$ and $\int%
-26te^{-7t}dt=\allowbreak\frac{26}{49}e^{-7t}\left(  7t+1\right)  $ (using
integration by parts) hence the above simplifies to%
\begin{align*}
\bar{x}_{p}  & =\Phi%
\begin{pmatrix}
\frac{-23}{10}t^{2}\\
\frac{26}{245}e^{-7t}+\frac{26}{35}te^{-7t}%
\end{pmatrix}
\\
& =%
\begin{pmatrix}
e^{-2t} & 2e^{5t}\\
-2e^{-2t} & e^{5t}%
\end{pmatrix}%
\begin{pmatrix}
\frac{-23}{10}t^{2}\\
\frac{26}{245}e^{-7t}+\frac{26}{35}te^{-7t}%
\end{pmatrix}
\\
& =%
\begin{pmatrix}
\frac{52}{245}e^{-2t}+\frac{52}{35}te^{-2t}-\frac{23}{10}t^{2}e^{-2t}\\
\frac{26}{245}e^{-2t}+\frac{26}{35}te^{-2t}+\frac{23}{5}t^{2}e^{-2t}%
\end{pmatrix}
\\
& =%
\begin{pmatrix}
\frac{1}{490}e^{-2t}\left(  -1127t^{2}+728t+104\right)  \\
\frac{1}{245}e^{-2t}\left(  1127t^{2}+182t+26\right)
\end{pmatrix}
\end{align*}
Hence the complete solution is%
\begin{align*}
\bar{x}  & =\bar{x}_{h}+\bar{x}_{p}\\
& =c_{1}%
\begin{pmatrix}
1\\
-2
\end{pmatrix}
e^{-2t}+c_{2}%
\begin{pmatrix}
2\\
1
\end{pmatrix}
e^{5t}+%
\begin{pmatrix}
\frac{1}{490}e^{-2t}\left(  -1127t^{2}+728t+104\right)  \\
\frac{1}{245}e^{-2t}\left(  1127t^{2}+182t+26\right)
\end{pmatrix}
\end{align*}
To find the constants, we apply initial conditions. At $t=0$%
\begin{align*}%
\begin{pmatrix}
7\\
3
\end{pmatrix}
& =c_{1}%
\begin{pmatrix}
1\\
-2
\end{pmatrix}
+c_{2}%
\begin{pmatrix}
2\\
1
\end{pmatrix}
+%
\begin{pmatrix}
\frac{52}{245}\\
\frac{26}{245}%
\end{pmatrix}
\\
c_{1}%
\begin{pmatrix}
1\\
-2
\end{pmatrix}
+c_{2}%
\begin{pmatrix}
2\\
1
\end{pmatrix}
& =%
\begin{pmatrix}
7\\
3
\end{pmatrix}
-%
\begin{pmatrix}
\frac{52}{245}\\
\frac{26}{245}%
\end{pmatrix}
\\%
\begin{pmatrix}
1 & 2\\
-2 & 1
\end{pmatrix}%
\begin{pmatrix}
c_{1}\\
c_{2}%
\end{pmatrix}
& =%
\begin{pmatrix}
\frac{1663}{245}\\
\frac{709}{245}%
\end{pmatrix}
\\%
\begin{pmatrix}
1 & 2\\
0 & 5
\end{pmatrix}%
\begin{pmatrix}
c_{1}\\
c_{2}%
\end{pmatrix}
& =%
\begin{pmatrix}
\frac{1663}{245}\\
\frac{807}{49}%
\end{pmatrix}
\end{align*}
Hence $5c_{2}=\frac{807}{49}$ or $c_{2}=\frac{807}{245}$ and $c_{1}%
+2c_{2}=\frac{1663}{245}$, hence $c_{1}=\frac{1663}{245}-2\left(  \frac
{807}{245}\right)  =\frac{1}{5}$. Therefore the solution becomes%
\[
\bar{x}=\frac{1}{5}%
\begin{pmatrix}
1\\
-2
\end{pmatrix}
e^{-2t}+\frac{807}{245}%
\begin{pmatrix}
2\\
1
\end{pmatrix}
e^{5t}+%
\begin{pmatrix}
\frac{1}{490}e^{-2t}\left(  -1127t^{2}+728t+104\right)  \\
\frac{1}{245}e^{-2t}\left(  1127t^{2}+182t+26\right)
\end{pmatrix}
\]

\end{document}

